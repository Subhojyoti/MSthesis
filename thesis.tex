\documentclass[PhD]{iitmdiss}
%\documentclass[MS]{iitmdiss}
%\documentclass[MTech]{iitmdiss}
%\documentclass[BTech]{iitmdiss}
\usepackage{times}
 \usepackage{t1enc}

\usepackage{graphicx}
\usepackage{epstopdf}
\usepackage{hyperref} % hyperlinks for references.
\usepackage{amsmath} % easier math formulae, align, subequations \ldots


%%%%%%%%%%%%%%%%%%%%%%%%%%%%%%%
%\usepackage{ijcai17}


\usepackage{macros}

\usepackage{latexsym} 


\begin{document}

%%%%%%%%%%%%%%%%%%%%%%%%%%%%%%%%%%%%%%%%%%%%%%%%%%%%%%%%%%%%%%%%%%%%%%
% Title page

\title{\LaTeX\ CLASS FOR DISSERTATIONS SUBMITTED TO IITM}

\author{Name}

\date{December 2017}
\department{Computer Science and Engineering}

%\nocite{*}
\maketitle

%%%%%%%%%%%%%%%%%%%%%%%%%%%%%%%%%%%%%%%%%%%%%%%%%%%%%%%%%%%%%%%%%%%%%%
% Certificate
\certificate

\vspace*{0.5in}

\noindent This is to certify that the thesis titled {\bf \LaTeX\ CLASS  
  FOR DISSERTATIONS SUBMITTED TO IIT-M}, submitted by {\bf Subhojyoti Mukherjee}, 
  to the Indian Institute of Technology, Madras, for
the award of the degree of {\bf Master of Science (Research)}, is a bona fide
record of the research work done by him under our supervision.  The
contents of this thesis, in full or in parts, have not been submitted
to any other Institute or University for the award of any degree or
diploma.

\vspace*{1.5in}

\begin{singlespacing}
\hspace*{-0.25in}
\parbox{2.5in}{
\noindent {\bf Prof.~1} \\
\noindent Research Guide \\ 
\noindent Professor \\
\noindent Dept. of Physics\\
\noindent IIT-Madras, 600 036 \\
} 
\hspace*{1.0in} 
%\parbox{2.5in}{
%\noindent {\bf Prof.~S.~C.~Rajan} \\
%\noindent Research Guide \\ 
%\noindent Assistant Professor \\
%\noindent Dept.  of  Aerospace Engineering\\
%\noindent IIT-Madras, 600 036 \\
%}  
\end{singlespacing}
\vspace*{0.25in}
\noindent Place: Chennai\\
Date: 22nd December 2017 


%%%%%%%%%%%%%%%%%%%%%%%%%%%%%%%%%%%%%%%%%%%%%%%%%%%%%%%%%%%%%%%%%%%%%%
% Acknowledgements
\acknowledgements

Thanks to all those who made \TeX\ and \LaTeX\ what it is today.

%%%%%%%%%%%%%%%%%%%%%%%%%%%%%%%%%%%%%%%%%%%%%%%%%%%%%%%%%%%%%%%%%%%%%%
% Abstract

\abstract

\noindent KEYWORDS: \hspace*{0.5em} \parbox[t]{4.4in}{\LaTeX ; Thesis;
  Style files; Format.}

\vspace*{24pt}

\noindent A \LaTeX\ class along with a simple template thesis are
provided here.  These can be used to easily write a thesis suitable
for submission at IIT-Madras.  The class provides options to format
PhD, MS, M.Tech.\ and B.Tech.\ thesis.  It also allows one to write a
synopsis using the same class file.  Also provided is a BIB\TeX\ style
file that formats all bibliography entries as per the IITM format.

The formatting is as (as far as the author is aware) per the current
institute guidelines.

\pagebreak

%%%%%%%%%%%%%%%%%%%%%%%%%%%%%%%%%%%%%%%%%%%%%%%%%%%%%%%%%%%%%%%%%
% Table of contents etc.

\begin{singlespace}
\tableofcontents
\thispagestyle{empty}

\listoftables
\addcontentsline{toc}{chapter}{LIST OF TABLES}
\listoffigures
\addcontentsline{toc}{chapter}{LIST OF FIGURES}
\end{singlespace}


%%%%%%%%%%%%%%%%%%%%%%%%%%%%%%%%%%%%%%%%%%%%%%%%%%%%%%%%%%%%%%%%%%%%%%
% Abbreviations
\abbreviations

\noindent 
\begin{tabbing}
xxxxxxxxxxx \= xxxxxxxxxxxxxxxxxxxxxxxxxxxxxxxxxxxxxxxxxxxxxxxx \kill
\textbf{IITM}   \> Indian Institute of Technology, Madras \\
\textbf{RTFM} \> Read the Fine Manual \\
\end{tabbing}

\pagebreak

%%%%%%%%%%%%%%%%%%%%%%%%%%%%%%%%%%%%%%%%%%%%%%%%%%%%%%%%%%%%%%%%%%%%%%
% Notation

\chapter*{\centerline{NOTATION}}
\addcontentsline{toc}{chapter}{NOTATION}

\begin{singlespace}
\begin{tabbing}
xxxxxxxxxxx \= xxxxxxxxxxxxxxxxxxxxxxxxxxxxxxxxxxxxxxxxxxxxxxxx \kill
\textbf{$r$}  \> Radius, $m$ \\
\textbf{$\alpha$}  \> Angle of thesis in degrees \\
\textbf{$\beta$}   \> Flight path in degrees \\
\end{tabbing}
\end{singlespace}

\pagebreak
\clearpage

% The main text will follow from this point so set the page numbering
% to arabic from here on.
\pagenumbering{arabic}


%%%%%%%%%%%%%%%%%%%%%%%%%%%%%%%%%%%%%%%%%%%%%%%%%%
% Introduction.
\chapter{Introduction}
\label{chap:intro}
This document provides a simple template of how the provided
\verb+iitmdiss.cls+ \LaTeX\ class is to be used.  Also provided are
several useful tips to do various things that might be of use when you
write your thesis.

Before reading any further please note that you are strongly advised
against changing any of the formatting options used in the class
provided in this directory, unless you are absolutely sure that it
does not violate the IITM formatting guidelines.  \emph{Please do not
  change the margins or the spacing.}  If you do change the formatting
you are on your own (don't blame me if you need to reprint your entire
thesis).  In the case that you do change the formatting despite these
warnings, the least I ask is that you do not redistribute your style
files to your friends (or enemies).

It is also a good idea to take a quick look at the formatting
guidelines.  Your office or advisor should have a copy.  If they
don't, pester them, they really should have the formatting guidelines
readily available somewhere.

To compile your sources run the following from the command line:
\begin{verbatim}
% latex thesis.tex
% bibtex thesis
% latex thesis.tex
% latex thesis.tex
\end{verbatim}
Modify this suitably for your sources.

To generate PDF's with the links from the \verb+hyperref+ package use
the following command:
\begin{verbatim}
% dvipdfm -o thesis.pdf thesis.dvi
\end{verbatim}

\section{Package Options}

Use this thesis as a basic template to format your thesis.  The
\verb+iitmdiss+ class can be used by simply using something like this:
\begin{verbatim}
\documentclass[PhD]{iitmdiss}  
\end{verbatim}

To change the title page for different degrees just change the option
from \verb+PhD+ to one of \verb+MS+, \verb+MTech+ or \verb+BTech+.
The dual degree pages are not supported yet but should be quite easy
to add.  The title page formatting really depends on how large or
small your thesis title is.  Consequently it might require some hand
tuning.  Edit your version of \verb+iitmdiss.cls+ suitably to do this.
I recommend that this be done once your title is final.

To write a synopsis simply use the \verb+synopsis.tex+ file as a
simple template.  The synopsis option turns this on and can be used as
shown below.
\begin{verbatim}
\documentclass[PhD,synopsis]{iitmdiss}                                
\end{verbatim}

Once again the title page may require some small amount of fine
tuning.  This is again easily done by editing the class file.

This sample file uses the \verb+hyperref+ package that makes all
labels and references clickable in both the generated DVI and PDF
files.  These are very useful when reading the document online and do
not affect the output when the files are printed.


\section{Example Figures and tables}

Fig.~\ref{fig:iitm} shows a simple figure for illustration along with
a long caption.  The formatting of the caption text is automatically
single spaced and indented.  Table~\ref{tab:sample} shows a sample
table with the caption placed correctly.  The caption for this should
always be placed before the table as shown in the example.


\begin{figure}[htpb]
  \begin{center}
    \resizebox{50mm}{!} {\includegraphics *{iitm.eps}}
    \resizebox{50mm}{!} {\includegraphics *{iitm.eps}}
    \caption {Two IITM logos in a row.  This is also an
      illustration of a very long figure caption that wraps around two
      two lines.  Notice that the caption is single-spaced.}
  \label{fig:iitm}
  \end{center}
\end{figure}

\begin{table}[htbp]
  \caption{A sample table with a table caption placed
    appropriately. This caption is also very long and is
    single-spaced.  Also notice how the text is aligned.}
  \begin{center}
  \begin{tabular}[c]{|c|r|} \hline
    $x$ & $x^2$ \\ \hline
    1  &  1   \\
    2  &  4  \\
    3  &  9  \\
    4  &  16  \\
    5  &  25  \\
    6  &  36  \\
    7  &  49  \\
    8  &  64  \\ \hline
  \end{tabular}
  \label{tab:sample}
  \end{center}
\end{table}

\section{Bibliography with BIB\TeX}

I strongly recommend that you use BIB\TeX\ to automatically generate
your bibliography.  It makes managing your references much easier.  It
is an excellent way to organize your references and reuse them.  You
can use one set of entries for your references and cite them in your
thesis, papers and reports.  If you haven't used it anytime before
please invest some time learning how to use it.  

I've included a simple example BIB\TeX\ file along in this directory
called \verb+refs.bib+.  The \verb+iitmdiss.cls+ class package which
is used in this thesis and for the synopsis uses the \verb+natbib+
package to format the references along with a customized bibliography
style provided as the \verb+iitm.bst+ file in the directory containing
\verb+thesis.tex+.  Documentation for the \verb+natbib+ package should
be available in your distribution of \LaTeX.  Basically, to cite the
author along with the author name and year use \verb+\cite{key}+ where
\verb+key+ is the citation key for your bibliography entry.  You can
also use \verb+\citet{key}+ to get the same effect.  To make the
citation without the author name in the main text but inside the
parenthesis use \verb+\citep{key}+.  The following paragraph shows how
citations can be used in text effectively.

More information on BIB\TeX\ is available in the book by
\cite{lamport:86}.  There are many
references~\citep{lamport:86,prabhu:xx} that explain how to use
BIB\TeX.  Read the \verb+natbib+ package documentation for more
details on how to cite things differently.

Here are other references for example.  \citet{viz:mayavi} presents a
Python based visualization system called MayaVi in a conference paper.
\citet{pan:pr:flat-fst} illustrates a journal article with multiple
authors.  Python~\citep{py:python} is a programming language and is
cited here to show how to cite something that is best identified with
a URL.

\section{Other useful \LaTeX\ packages}

The following packages might be useful when writing your thesis.

\begin{itemize}  
\item It is very useful to include line numbers in your document.
  That way, it is very easy for people to suggest corrections to your
  text.  I recommend the use of the \texttt{lineno} package for this
  purpose.  This is not a standard package but can be obtained on the
  internet.  The directory containing this file should contain a
  lineno directory that includes the package along with documentation
  for it.

\item The \texttt{listings} package should be available with your
  distribution of \LaTeX.  This package is very useful when one needs
  to list source code or pseudo-code.

\item For special figure captions the \texttt{ccaption} package may be
  useful.  This is specially useful if one has a figure that spans
  more than two pages and you need to use the same figure number.

\item The notation page can be entered manually or automatically
  generated using the \texttt{nomencl} package.

\end{itemize}

More details on how to use these specific packages are available along
with the documentation of the respective packages.


%%%%%%%%%%%%%%%%%%%%%%%%%%%%%%%%%%%%%%%%%%%%%%%%%%%%%%%%%%%%

%%%%%%%%%%%%%%%%%%%%%%%%%%%%%%%%%%%%%%%%%%%%%%%%%%%%%%%%%%%%
\chapter{Thresholding Bandits}
\label{chap:tbandit}
\section{Introduction}
\label{tbandit:intro}
Stochastic multi-armed bandit (MAB) problems are instances of the classic sequential decision-making scenario; specifically an MAB problem comprises of a learner and a collection of actions (or arms), denoted $\mathcal{A}$. In each trial the learner plays (or pulls) an arm $i\in\mathcal{A}$ which yields independent and identically distributed (i.i.d.) reward samples from a distribution (corresponding to arm $i$), whose expectation is denoted by $r_i$. 
%whose rewards are  samples from the distribution specific to the arm $i\in A$ and whose expected reward is denoted by $r_{i},\forall i\in A$. 
The learner's objective is to identify an arm corresponding to the maximum expected reward, denoted $r^{*}$. Thus, at each time-step the learner 
%selects an arm $i$ and hence
is faced with the \emph{exploration vs.\ exploitation dilemma}, where it can pull an arm which has yielded the highest mean reward (denoted $\hat{r}_{i}$) thus far (\emph{exploitation}) or continue to explore other arms with the prospect of finding a better arm 
%superior performance 
whose performance has not been observed sufficiently (\emph{exploration}).

%In the stochastic multi-armed bandit setting a learning agent is required to choose from a set of decisions or arms at every round. The agent is then presented with a reward for that round, which is an independent draw from a stationary distribution specific to the arm selected. The agent, however, does not know the mean of the distributions associated with each arm, denoted by $r_{i}$, including the optimal arm which will give it the best reward, denoted by $r^{*}$. The agent attempts to make arm choices that will maximize some performance measure by keeping track of the reward that has been gathered from previous selections of the arm, for each arm. This is called the estimated mean reward of an arm denoted by $\hat{r}_{i}$. The bandit problem can be conceptualized as a sequential decision making process where the agent is at each round presented with an \emph{exploration-exploitation dilemma}. The agent could pull the arm which has the highest observed mean reward till now (exploitation) or to explore other arms, with the prospect of finding superior performance which was previously unobserved (exploration). 

%
%	Formally, let $r_i$, $i=1,\ldots,K$ denote the mean rewards of the $K$ arms and $r^* = \max_i r_i$ the optimal mean reward. The objective in some of the stochastic bandit problem is to minimize the cumulative regret, which is defined as follows:
%\begin{align*}
%R_{T}=r^{*}T - \sum_{i\in A} r_{i}N_{i}(T),
%\end{align*}
%where $T$ is the number of rounds, $N_{i}(T)=\sum_{m=1}^T I(I_m=i)$ is the number of times the algorithm chose arm $i$ up to round $T$.
%The expected regret of an algorithm after $T$ rounds can be written as,
%
%\begin{align*}
%\E[R_{T}]= \sum_{i=1}^K \E[N_i(T)] \Delta_i,
%\end{align*}
%where $\Delta_{i}=r^{*}-r_{i}$ denotes the gap between the means of the optimal arm and of the $i$-th arm. 


Pure-exploration MAB problems are unlike their traditional (exploration vs.\ exploitation)  counterparts where the  objective is to minimize the cumulative regret (which is the total loss incurred by the learner for not playing the optimal arm throughout the time horizon $T$). Instead, in pure-exploration problems a learning algorithm, until time $T$, can invest entirely on exploring the arms without being concerned about the loss incurred while exploring; the objective is to minimize the probability that the arm recommended at time $T$ is not the best arm.  In this paper, we further consider a combinatorial version of the pure-exploration MAB, called the thresholding bandit problem (TBP).  Here, the learning algorithm is provided with a threshold $\tau$, and the objective, after exploring for $T$ rounds, is to  output all arms $i$ whose $r_{i}$ is above $\tau$. 
It is important to emphasize that the \emph{thresholding} bandit problem is different from the \emph{threshold} bandit setup studied in \cite{abernethy2016threshold}, where the learner receives an unit reward whenever the value of an observation is above a threshold. 

%%%%%%%%%%%%%%%
% Old para
%%%%%%%%%%%%%%%
%Pure-exploration problems are unlike their traditional (exploration vs.\ exploitation) counterparts where the  objective is to minimize the cumulative regret, which is the total loss incurred by the learner for not playing the optimal arm throughout the time horizon $T$. Instead, in the pure exploration setup the learning algorithm is provided with a threshold $\tau$, and the objective, after exploring for $T$ rounds, is to  output all arms $i$ whose $r_{i}$ is above $\tau$. Thus, the learning algorithm, until  time $T$, can invest entirely on exploring the arms  without being concerned about the loss incurred while exploring. It is important to emphasize that the \emph{thresholding bandit} problem is different from the \emph{threshold bandit} setup studied in \cite{abernethy2016threshold}, where the learner receives an unit reward whenever the value of an observation is above a threshold. 
%<<<<<<< Updated upstream
%Pure-exploration problems are unlike their traditional (exploration vs.\ exploitation) counterparts where the  objective is to minimize the cumulative regret, which is the total loss incurred by the learner for not playing the optimal arm throughout the time horizon $T$. In this paper we study the fixed-budget setting of a specific combinatorial pure-exploration problem, called the thresholding bandit problem (TBP), in the context of (MAB) setting. In this pure-exploration setup the learning algorithm is provided with a threshold $\tau$, and the objective, after exploring for $T$ rounds, is to  output all arms $i$ whose $r_{i}$ is above $\tau$. Thus, the learning algorithm, until time $T$, can invest entirely on exploring the arms  without being concerned about the loss incurred while exploring. It is important to emphasize that the \emph{thresholding} bandit problem is different from the \emph{threshold} bandit setup studied in \cite{abernethy2016threshold}, where the learner receives an unit reward whenever the value of an observation is above a threshold. 
%=======

%Pure-exploration problems are unlike their traditional (exploration vs.\ exploitation) counterparts where the  objective is to minimize the cumulative regret, which is the total loss incurred by the learner for not playing the optimal arm throughout the time horizon $T$. Instead, in the pure exploration setup the learning algorithm is provided with a threshold $\tau$, and the objective, after exploring for $T$ rounds, is to  output all arms $i$ whose $r_{i}$ is above $\tau$. Thus, the learning algorithm, until  time $T$, can invest entirely on exploring the arms  without being concerned about the loss incurred while exploring. It is important to emphasize that the \emph{thresholding bandit} problem is different from the \emph{threshold bandit} setup studied in \cite{abernethy2016threshold}, where the learner receives an unit reward whenever the value of an observation is above a threshold. 
%>>>>>>> Stashed changes

% reward on each time-step depends on a threshold value and the learner receives the reward only if the reward is above the threshold.


%This is a specific instance of combinatorial pure exploration where the learning algorithm can explore as much as possible given a fixed horizon $T$ and not be concerned with the usual exploration-exploitation dilemma. 

Formally, the problem we consider is the following. First, we define the set $S_{\tau}=\lbrace i\in \mathcal{A}: r_{i}\geq \tau \rbrace$. Note that, $S_\tau$ is the set of all arms whose reward mean is greater than $\tau$. Let 
$S_\tau^c$ % =\mathcal{A}\backslash S_\tau$
 denote the complement of $S_\tau$, i.e.,  $S_{\tau}^{c}=\lbrace i\in \mathcal{A}: r_{i} < \tau \rbrace$. Next, let $\hat{S}_{\tau}=\hat{S}_{\tau}(T)\subseteq \mathcal{A}$ denote the recommendation of a learning algorithm (under consideration) after $T$ time units of exploration, while $\hat{S}_{\tau}^c$ denotes its complement.
%  Also we define $\hat{S}_{\tau}=\hat{S}_{\tau}(T)\subset \mathcal{A}$ and its complementary set $\hat{S}_{\tau}^{C}$ as the recommendation of the learning algorithm after $T$ rounds. 
% Given such sets exists, 
The performance of the learning agent is measured by the accuracy with which it can classify the arms into $S_{\tau}$ and $S_{\tau}^{c}$ after time horizon $T$. Equivalently, using $\mathbb{I}(E)$ to denote the indicator of an event $E$, the \emph{loss} $\mathcal{L}(T)$ is defined as
\begin{align*}
\Ls (T) = \mathbb{I}\big(\lbrace S_{\tau}\cap \hat{S}_{\tau}^{c}\neq \emptyset\rbrace    \cup    \lbrace\hat{S}_{\tau}\cap S_{\tau}^{c}\neq \emptyset\rbrace\big).
\end{align*}			
Finally, the goal of the learning agent is to minimize the expected loss:
% So, the expected loss after $T$ rounds is,
\begin{align*}
\E[\Ls(T)] = \Pb\big(\lbrace S_{\tau}\cap \hat{S}_{\tau}^{c} \neq \emptyset \rbrace  \cup   \lbrace \hat{S}_{\tau}\cap S_{\tau}^{c} \neq \emptyset\rbrace\big).
\end{align*}
Note that the expected loss is simply the \emph{probability of mis-classification} (i.e., error), that occurs either if a good arm is rejected or a bad arm is accepted as a good one.
% (represented by the events $\lbrace S_{\tau}\cap \hat{S}_{\tau}^{c} \neq \emptyset \rbrace$ and $\lbrace \hat{S}_{\tau}\cap S_{\tau}^{c} \neq \emptyset\rbrace$, respectively).

%which we can say is the probability of making mistake, that is whether the learning agent at the end of round $T$ rejects arms from $S_{\tau}$ or accepts arms from $S_{\tau}^{C}$ in its final recommendation. 

%Also, we are looking at an anytime algorithm, so the knowledge of $T$ may not be known to the learner.





The above TBP formulation has several applications, for instance, from areas ranging from anomaly detection and classification (see  \citet{locatelli2016optimal}) to industrial application. Particularly in industrial applications a learners objective is to choose (i.e., keep in  operation) all machines whose productivity is above a threshold. The TBP also finds applications in mobile communications (see \citet{audibert2010best})  where the users are to be allocated only those channels whose quality is above an acceptable threshold.
% In some cases the TBP problem is more relevant than the variants of TopM problem (identifying the best $M$ arms from $K$ given arms). where the learner has to keep all those workers active whose productivity is above a particular threshold $\tau$, or allocating channels whose quality is above a threshold for Mobile Communications 
% or in crowd-sourcing while hiring workers the TBP problem 

%
%	1. \emph{Product Selection:} A company wants to introduce a new product in market and there is a clear separation of the test phase from the commercialization phase. In this case the company tries to minimize the loss it might incur in the commercialization phase by testing as much as possible in the test phase. So from the several variants of the product that are in the test phase the learning agent must suggest the product variant(s) that are above a particular threshold $\tau$ at the end of the test phase that have the highest probability of minimizing loss in the commercialization phase. A similar problem has been discussed for single best product variant identification without threshold in \cite{bubeck2011pure}. 
%
%	2. \emph{Mobile Phone Channel Allocation:} Another similar problem as above concerns channel allocation for mobile phone communications (\cite{audibert2009exploration}). Here there is a clear separation between the allocation phase and communication phase whereby in the allocation phase a learning algorithm has to explore as many channels as possible to suggest the best possible set of channel(s) that are above a particular threshold $\tau$. The threshold depends on the subscription level of the customer. With higher subscription the customer is allowed better channel(s) with the $\tau$ set high. Each evaluation of a channel is noisy and the learning algorithm must come up with the best possible suggestion within a very small  number of attempts.
%
%	3. \emph{Anomaly Detection and Classification:} Thresholding bandit can also be used for anomaly detection and classification where we define a cutoff level $\tau$ and for any samples above this cutoff gets classified as an anomaly. For further reading we point the reader to section 3 of \cite{locatelli2016optimal}.
%
%

\subsection{Related Work}
\label{tbandit:prevRes}
Significant amount of literature is available on the stochastic MAB setting with respect to minimizing the cumulative regret. While the seminal work of \cite{robbins1952some}, \cite{thompson1933likelihood},  and \cite{lai1985asymptotically} prove asymptotic lower bounds on the cumulative regret, the more recent work of \cite{auer2002finite} propose the UCB1 algorithm that provides finite time-horizon guarantees. Also, subsequent work such as \cite{audibert2009minimax} and \cite{auer2010ucb} have improved the upper bounds on the cumulative regret. The authors in \cite{auer2010ucb} have proposed a \emph{round-based}\footnote{An algorithm is said to be \textit{round-based} if it pulls all the arms equal number of times in each round, and then proceeds to eliminate one or more arms that it identifies to be sub-optimal.} version of the UCB algorithm, referred to as UCB-Improved. Of special mention is the work of \cite{audibert2009exploration} where the authors have introduced a \emph{variance-aware} UCB algorithm, referred to as UCB-V; it is shown that the algorithms that take into account variance estimation along with mean estimation tends to perform better than the algorithms that solely focuses on mean estimation, for instance, such as UCB1.
For a more detail survey of literature on UCB algorithms, we refer the reader to \cite{bubeck2012regret}. 


%An early work involving a bandit setup is \cite{thompson1933likelihood}, where the author deals with the problem of choosing between two treatments to administer on patients who come in sequentially. Following the seminal work of  \cite{robbins1952some}, bandit algorithms have been extensively studied in a variety of applications. From a theoretical standpoint, an asymptotic lower bound for the regret was established in \cite{lai1985asymptotically}. Several other works such as \cite{auer2002finite},  \cite{audibert2009minimax} and \cite{auer2010ucb} have shown results for minimizing cumulative regret in stochastic bandit setup whereas works such as \cite{auer2002nonstochastic} have concentrated on adversarial bandit setup.
	
%	In the pure exploration setup, a significant amount of research has been done on finding the best arm(s) from a set of arms. 

In this work we are particularly interested in \emph{pure-exploration MABs},  where the focus in primarily on simple regret rather than the cumulative regret. The relationship between cumulative regret and simple regret is proved in \cite{bubeck2011pure} where the authors prove that minimizing the simple regret necessarily results in maximizing the cumulative regret.
The pure exploration problem has been explored  mainly under the following two settings:
	
	\emph{1. Fixed Budget setting:} Here the learning algorithm has to suggest the best arm(s) within a fixed time-horizon $T$, that is usually given as an input. The objective is to maximize the probability of returning the best arm(s).  This is the scenario we consider in our paper. In \cite{audibert2010best} the authors propose the  UCBE and the Successive Reject (SR) algorithm, and prove simple-regret guarantees for the problem of identifying the single best arm.  In the combinatorial fixed budget setup \cite{gabillon2011multi} propose the GapE and GapE-V algorithms that suggest, with high probability, the best $m$ 
	% (given as input)
	 arms at the end of the time budget. Similarly, \cite{bubeck2013multiple} introduce the  Successive Accept Reject (SAR) algorithm, which is an extension of the SR algorithm; SAR is a round based algorithm whereby at the end of each round an arm is either accepted or rejected (based on certain confidence conditions) until the top $m$ arms are suggested at the end of the budget with high probability. A similar combinatorial setup was explored in \cite{chen2014combinatorial} where the authors propose the Combinatorial Successive Accept Reject (CSAR) algorithm, which is similar in concept to SAR but with a more general setup. 

	\emph{2. Fixed Confidence setting:} In this setting the learning algorithm has to suggest the best arm(s) with a fixed confidence (given as input) with as fewer number of attempts as possible. The single best arm identification has been studied in \cite{even2006action}, while for the combinatorial setup \cite{kalyanakrishnan2012pac} have proposed the LUCB algorithm which, on termination, returns  $m$ arms which are at least $\epsilon$ close to the true top-$m$ arms with probability at least $1-\delta$. For a detail survey of this setup we refer the reader to \cite{jamieson2014best}. 

Apart from these two settings some unified approaches has also been suggested in \cite{gabillon2012best} which proposes the algorithms UGapEb and UGapEc which can work in both the above two settings. The thresholding bandit problem is a specific instance of the pure-exploration setup of \cite{chen2014combinatorial}. In the latest work of \cite{locatelli2016optimal} Anytime Parameter-Free Thresholding (APT) algorithm comes up with an improved anytime guarantee than CSAR for the thresholding bandit problem.	
	
	
%	\emph{1. Fixed Budget setting:} In this setting the learning algorithm has to suggest the best arm(s) within a fixed number of attempts that is given as an input. The objective here is to maximize the probability of returning the best arm(s). One of the foremost papers to deal with single best arm identification is \cite{audibert2009exploration} where the authors come up with the algorithm UCBE and Successive Reject(SR) with simple regret guarantees. The relationship between cumulative regret and simple regret is proved in \cite{bubeck2011pure} where the authors prove that minimizing the simple regret necessarily results in maximizing the cumulative regret. In the combinatorial fixed budget setup \cite{gabillon2011multi} come up with Gap-E and Gap-EV algorithm which suggests the best $m$ (given as input) arms at the end of the budget with high probability. Similarly, \cite{bubeck2013multiple} comes up with the algorithm Successive Accept Reject(SAR) which is an extension of the SR algorithm. SAR is a round based algorithm whereby at the end of round an arm is either accepted or rejected based on certain conditions till the required top $m$ arms are suggested at the end of the budget with high probability. 
%
%	\emph{2 Fixed Confidence setting:} In this setting the the learning algorithm has to suggest the best arm(s) with a fixed (given as input) confidence with as less number of attempts as possible. The single best arm identification has been handled in \cite{even2006action} where they come up with an algorithm called Successive Elimination (SE) which comes up with an arm that is $\epsilon$ close to the optimal arm. In the combinatorial setup recently \cite{kalyanakrishnan2012pac} have suggested the LUCB algorithm which on termination returns $m$ arms which are atleast $\epsilon$ close to the true top $m$ arms with $1-\delta$ probability.
%
%	Apart from these two settings some unified approach has also been suggested in \cite{gabillon2012best} which proposes the algorithms UGapEb and UGapEc which can work in both the above two settings. A similar combinatorial setup was also explored in \cite{chen2014combinatorial} where the authors come up with more similarities and dissimilarities between these two settings in a more general setup. In their work, the learning algorithm, called Combinatorial Successive Accept Reject (CSAR) is similar to SAR with a more general setup. The thresholding bandit problem is a specific instance of the pure exploration setup of \cite{chen2014combinatorial}. In the latest work in \cite{locatelli2016optimal} the algorithm Anytime Parameter-Free Thresholding (APT) algorithm comes up with a better anytime guarantee than CSAR for the thresholding bandit problem.

\subsection{Our Contribution}
\label{tbandit:contribution}
In this paper we propose the Augmented UCB (AugUCB) algorithm for the fixed-budget setting of a specific combinatorial, pure-exploration, stochastic MAB called the thresholding bandit problem.
%In this paper we propose AugUCB algorithm for the fixed-budget, comp thresholding bandit problem.
 AugUCB essentially combines the approach of UCB-Improved, CCB \citep{liu2016modification} and APT algorithms. Our algorithm takes into account the empirical variances of the arms along with mean estimates; to the best of our knowledge this is the first variance-based algorithm for the considered TBP. 
Thus, we also address an open problem discussed in \cite{auer2010ucb} of designing an algorithm that can eliminate arms based on variance estimates. In this regard, note that both CSAR and APT are not variance-based algorithms. 

\begin{table}[b]
\caption{AugUCB vs.\ State of the art}
\label{tab:regret-bds}
\begin{center}
\begin{tabular}{|p{2.3cm}|p{8.4cm}|}
% \toprule
\hline
Algorithm  & Upper Bound on Expected Loss \\
% \midrule
\hline
\hline
AugUCB      &$ \exp\left(- \dfrac{T}{4096 \log(K\log K)H_{\sigma,2}} + \log\left(2KT\right) \right) $ \\
\hline
\hline
UCBEV		&$\exp\left(-\dfrac{1}{512}\frac{T-2K}{H_{\sigma,1}} + \log\left(6KT\right)\right)$ \\
%\midrule
\hline
\hline
APT         &$\exp\left(-\dfrac{T}{64 H_1}+2\log((\log(T)+1)K)\right)$ \\
% \midrule
\hline
\hline
CSAR		&$\exp\left(-\dfrac{T-K}{72\log(K)H_{CSAR,2}}+2\log(K)\right)$ \\
%\midrule
\hline

%\bottomrule
\end{tabular}
\end{center}
\end{table}

Our theoretical contribution comprises 
 proving an upper bound on the expected loss incurred by AugUCB (Theorem~\ref{tbandit:Result:Theorem:1}).
In Table \ref{tab:regret-bds} we compare the upper bound on the losses incurred by the various algorithms, including AugUCB. The terms $H_1, H_2$, $H_{CSAR,2}, H_{\sigma,1}$ and $H_{\sigma,2}$ represent various problem complexities, and are as defined in Section~\ref{tbandit:results}. From Section~\ref{tbandit:results} we note that, for all $K\ge8$, we have
\begin{align*}
\log\left(K\log K\right) H_{\sigma,2} > \log(2K) H_{\sigma,2} \ge H_{\sigma,1}.
\end{align*}
%; relation between these quantities are also given in Section~\ref{results} 
%The term containing $H_{\sigma,2}$ is comparable to the similar terms (containing $H_{\sigma,1}$) for the error probability of GapE-V \cite{gabillon2011multi} algorithm which we modify to perform in the TBP problem and name it as UCBEV.
Thus, it follows that the upper bound for UCBEV is better than that for AugUCB.
 %The error probability of UCBEV for single bandit multi-armed case is given in Table \ref{tab:regret-bds}. We see that $\log(\frac{3}{16} K\log K) H_2^{\sigma} > \log(2K) H_2^{\sigma} \ge H_1^{\sigma}$ and hence our algorithm is weaker with respect to UCBEV for single  multi-armed bandit scenario.
 However, implementation of UCBEV algorithm requires $H_{\sigma,1}$ as input, whose computation is not realistic in practice. In contrast, our AugUCB algorithm requires no such complexity factor as input. 
%Theoretically, we can compare the first term (containing $H_2$) of our expected loss and see that for all $K\geq 4$, $ H_2 \log(\frac{3}{16} K\log K) > (\log K)H_{CSAR,2}\geq H_1 $ and hence our result is weaker than CSAR and APT.

Proceeding with the comparisons, we emphasize that the upper bound for  AugUCB is, in fact, not comparable with that of APT and CSAR; this is because the complexity term $H_{\sigma,2}$ is not explicitly comparable with either $H_1$ or $H_{CSAR,2}$. However, through extensive simulation experiments we find that AugUCB significantly outperforms both APT, CSAR and other non variance-based algorithms. AugUCB also outperforms UCBEV under explorations where non-optimal values of $H_{\sigma,1}$  are used. In particular, we consider experimental scenarios comprising large number of arms, with the variances of arms in $S_\tau$ being large. AugUCB, being variance based, exhibits superior performance under these settings.  
%


%Empirically we show that for a large number of arms when the variance of the arms lying above $\tau$ are high, our algorithm performs better than all other algorithms, except the algorithm UCBEV which has access to the underlying problem complexity and also is a variance-aware algorithm. 
%
%AugUCB requires one input parameter and the exact choice for the parameter is derived in Theorem \ref{Result:Theorem:1}. Also, unlike SAR or CSAR, AugUCB does not have explicit accept or reject sets rather the arm elimination condition simply removes arm(s) if it is sufficiently sure that the mean of the arms are very high or very low about the threshold based on mean and variance estimation thereby re-allocating the remaining budget among the surviving arms. This although is a tactic similar to SAR or CSAR, but here at any round, an arbitrary number of arms can be accepted or rejected thereby improving upon SAR and CSAR which accepts/rejects one arm in every round. Also their round lengths are non-adaptive and they pull all the arms equal number of times in each round. 
%At every timestep AugUCB pulls the arm that minimizes thereby making this an anytime algorithm whereby we need not finish every round. 
%Irrespective of this case AugUCB also employs elimination of arms based on mean estimation only and is the first such algorithm which uses elimination by both mean and variance estimation simultaneously.

The remainder of the paper is organized as follows. In section \ref{tbandit:algorithm} we present our AugUCB algorithm. 
Section \ref{tbandit:results} contains our main theorem on expected loss, while section \ref{tbandit:expt} contains simulation experiments. We finally draw our conclusions in section \ref{tbandit:conclusion}.
%in section \ref{notation} we introduce the notations and the


\section{Augmented-UCB Algorithm}
\label{tbandit:algorithm}
%The algorithm is presented below:-

%%%%%%%%%%%%%%%% alg-custom-block %%%%%%%%%%%%
%\algblock{ArmElim}{EndArmElim}
%\algnewcommand\algorithmicArmElim{\textbf{\em Arm Elimination by Mean Estimation}}
% \algnewcommand\algorithmicendArmElim{}
%\algrenewtext{ArmElim}[1]{\algorithmicArmElim\ #1}
%\algrenewtext{EndArmElim}{\algorithmicendArmElim}
%\algtext*{EndArmElim}
%
%\algblock{ArmElimV}{EndArmElimV}
%\algnewcommand\algorithmicArmElimV{\textbf{\em Arm Elimination by Mean and Variance Estimation}}
% \algnewcommand\algorithmicendArmElimV{}
%\algrenewtext{ArmElimV}[1]{\algorithmicArmElimV\ #1}
%\algrenewtext{EndArmElimV}{\algorithmicendArmElimV}
%\algtext*{EndArmElimV}
%
%\algblock{ResetParam}{EndResetParam}
%\algnewcommand\algorithmicResetParam{\textbf{\em Reset Parameters}}
% \algnewcommand\algorithmicendResetParam{}
%\algrenewtext{ResetParam}[1]{\algorithmicResetParam\ #1}
%\algrenewtext{EndResetParam}{\algorithmicendResetParam}
%\algtext*{EndResetParam}

%%%%%%%%%%%%%%%%%%%%%%%%%%%%%%%%%%%%%%%%%%%%%%%%%%%%%%%%%%%%%%%%%%%%%%%%
%Old Algorithm
%%%%%%%%%%%%%%%%%%%%%%%%%%%%%%%%%%%%%%%%%%%%%%%%%%%%%%%%%%%%%%%%%%%%%%%%

%\begin{algorithm}[th!]
%\caption{AugmentedUCB}
%\label{alg:augucb}
%\begin{algorithmic}
%\State {\bf Input:} Time horizon $T$, exploration parameters $\rho_{\mu}$, $\rho_v$ and $\psi$, threshold $\tau$.
%\State {\bf Initialization:} Set $B_{0}:=A$, $M=\left\lfloor \frac{1}{2}\log_{2} \frac{T}{e}\right\rfloor $, $m:=0$, $\epsilon_{0}:=1$, $\ell_{0}=\left\lceil \frac{2\psi\log( T\epsilon_{0}^{2})}{\epsilon_{0}} \right\rceil$ and $N_{0}=K\ell_{0} $.
%\State Pull each arm once
%\State \For{$t=K+1,..,T$}
%\State Pull arm $i\in\argmin_{j\in B_{m}}\bigg\lbrace |\hat{r}_{j} - \tau | - 2s_{j}\bigg\rbrace$
%\State $t:=t+1$ 
%\ArmElim
%\State For each arm $i \in B_{m}$, remove arm ${i}$ from $B_{m}$ if
%\begin{align*}
%\hat{r}_{i} + c_i  < \tau - c_i \mbox{ or } \hat{r}_{i} - c_i  > \tau + c_i \\
%\text{where $c_i=\sqrt{\frac{\rho_{\mu}\psi\log{( T\epsilon_{m}^{2})}}{2 n_{i}}}$}
%\end{align*}
%\EndArmElim
%\ArmElimV
%\State For each arm $i \in B_{m}$, remove arm ${i}$ from $B_{m}$ if
%\begin{align*}
%\hat{r}_{i} + s_i  < \tau - s_i \mbox{ or } \hat{r}_{i} - s_i  > \tau + s_i \\
%\text{where $s_i=\sqrt{\frac{\rho_v\psi\hat{V}_{i}\log{( T\epsilon_{m}^{2})}}{4 n_{i}} + \frac{\rho_v\psi \log{(T\epsilon_{m}^{2})}}{4 n_{i}}}$}
%\end{align*}
%\EndArmElimV
%\State \If{$t\geq N_{m}$ and $m \leq M$}
%\ResetParam
%\State $\epsilon_{m+1}:=\frac{\epsilon_{m}}{2}$
%\State $B_{m+1} := B_{m}$
%\State $\ell_{m+1}:=\left\lceil \frac{2\psi\log( T\epsilon_{m+1}^{2})}{\epsilon_{m+1}} \right\rceil$
%\State $N_{m+1} := t + |B_{m+1}|\ell_{m+1}$
%\State $m := m+1$
%\EndResetParam
%\EndIf
%\EndFor
%\State Output $\hat{S}_{\tau}=\lbrace i: \hat{r}_{i}\geq \tau \rbrace$.
%\end{algorithmic}
%\end{algorithm}

%%%%%%%%%%%%%%%%%%%%%%%%%%%%%%%%%%%%%%%%%%%%%%%%%%%%%%%%%%%%%%%%%%%%%%%%%%%%%%%%%%%%%

%%%%%%%%%%%%%%%%%%%%%%%%%%%%%%%%%%%%%%%%%%%%%%%%%%%%%%%%%%%%%%%%%%%%%%%%%%%%%%%%%
%New 	Algorithm
%%%%%%%%%%%%%%%%%%%%%%%%%%%%%%%%%%%%%%%%%%%%%%%%%%%%%%%%%%%%%%%%%%%%%%%%%%%%%%%%%


%%%%%%%%%%%%%%%%%%%%%%
%Notations moved here
%%%%%%%%%%%%%%%%%%%%%%
\label{tbandit:notation}
\textbf{Notation and assumptions:} $\mathcal{A}$ denotes the set of arms, and $|\mathcal{A}|=K$ is the number of arms in $\mathcal{A}$. 
%Arms generic arm is indexed by $i,j\in\mathcal{A}$. 
For arm $i\in\mathcal{A}$, we use $r_{i}$ to denote the true mean of the distribution from which the rewards are sampled, while $\hat{r}_{i}(t)$ denotes the estimated mean at time $t$. Formally, using $n_i(t)$ to denote the number of times arm $i$ has been pulled until time $t$, we have $\hat{r}_{i}(t)=\frac{1}{n_{i}(t)}\sum_{z=1}^{n_i(t)} X_{i,z}$, where $X_{i,z}$ is the reward sample received when arm $i$ is pulled for the $z$-th time. %
Similarly, we use $\sigma_{i}^{2}$ to denote the true variance of the reward distribution corresponding to arm $i$, while $\hat{v}_{i}(t)$ is the estimated variance, i.e., $\hat{v}_{i}(t)=\frac{1}{n_i(t)}\sum_{z=1}^{n_{i}(t)}(X_{i,z}-\hat{r}_{i})^{2}$. Whenever there is no ambiguity about the underlaying  time index $t$, for simplicity we neglect $t$ from the notations and simply use  $\hat{r}_i, \hat{v}_i,$ and $n_i, $ to denote the respective quantities.  Let  $\Delta_{i}=|\tau-r_{i}|$ denote the distance of the true mean from the threshold $\tau$. Also, the rewards are assumed to take values in $[0,1]$.

%%%%%%%%%%
%1-sub-gaussian assumption removed
%%%%%%%%%%
%Along the lines of \cite{locatelli2016optimal} we assume that all the reward distributions are $1$-sub-Gaussian (note that,  $1$-sub-Gaussian includes Gaussian distributions with variance less than $1$, distributions supported on an interval of length less than 2, etc).


%%%%%%%%%%%%%%%%%%%%%%
\textbf{The Algorithm:} The Augmented-UCB (AugUCB) algorithm is presented in Algorithm~\ref{alg:augucb}.
AugUCB is essentially based on the arm elimination method of the UCB-Improved \cite{auer2010ucb}, but adapted to the thresholding bandit setting proposed in \cite{locatelli2016optimal}. However, unlike the UCB improved (which is based on mean estimation) our algorithm employs \emph{variance estimates} (as in \cite{audibert2009exploration}) for arm elimination; to the best of our knowledge this is the first variance-aware  algorithm for the thresholding bandit problem. Further, we allow for arm-elimination at each time-step, which is in contrast to the earlier work (e.g., \cite{auer2010ucb,chen2014combinatorial}) where the arm elimination task is deferred to the end of the respective exploration rounds. The details are presented below.

% In algorithm \ref{alg:augucb}, hence referred to as AugUCB, we have two exploration parameters, $\rho_{\mu}$ and $\rho_v$ which are the arm elimination parameters. $\psi_{m}$ is the exploration regulatory factor. 
%The main approach is based on the UCB-Improved algorithm with modifications suited for the thresholding bandit problem. 
The active set $B_{0}$ is initialized with all the arms from $\mathcal{A}$. We divide the entire budget $T$ into rounds/phases like in UCB-Improved, CCB, SAR and CSAR. At every time-step AugUCB checks for arm elimination conditions, while updating parameters at the end of each round. As suggested by \cite{liu2016modification} to make AugUCB to overcome too much early exploration, we no longer pull all the arms equal number of times in each round. Instead, we choose an arm in the active set $B_m$ that minimizes $(|\hat{r}_{i} - \tau |-2s_i)$ where 
%$\min_{i\in B_{m}}\big\lbrace |\hat{r}_{i} - \tau | - 2\sqrt{\frac{\rho_v\psi_m \hat{V}_{i} \log ( T \epsilon_{m})}{4 n_{i}} + \frac{\rho_v\psi_m \log{( T\epsilon_{m})}}{4 n_{i}}} \big\rbrace $
\begin{small}
\begin{align*}
s_i & = \sqrt{\frac{\rho\psi_m (\hat{v}_{i}+1) \log ( T \epsilon_{m})}{4 n_{i}}} %+ \frac{\rho\psi_m \log{( T\epsilon_{m})}}{4 n_{i}}}.
\end{align*}
\end{small} 
with $\rho$ being the arm elimination parameter and $\psi_{m}$ being the exploration regulatory factor.
%  in the active set $B_{m}$. 
The above condition ensures that an arm closer to the threshold $\tau$ is pulled; 
%and with suitable choice of $\rho_{\mu}$ and $\rho_v$ we can fine tune the exploration. 
parameter $\rho$ can be used to fine tune the elimination interval.
The choice of exploration factor, $\psi_m$,
% $\psi_m=\frac{T\epsilon_m}{(\log(\frac{3}{16} K\log K))^{2}}$ 
comes directly from \cite{audibert2010best} and \cite{bubeck2011pure} where it is  stated that in pure exploration setup, the exploring factor must be linear in $T$ (so that an exponentially small probability of error is achieved) rather than being logarithmic in $T$ (which is more suited for minimizing cumulative regret).

\begin{algorithm}[t!]
\caption{AugUCB}
\label{alg:augucb}
\begin{algorithmic}
\State {\bf Input:} Time budget $T$; parameter $\rho$; 
% $\rho_{\mu}$, $\rho_v$ 
  threshold $\tau$
\State {\bf Initialization:} $B_{0}=\mathcal{A}$; $m=0$; $\epsilon_{0}=1$;
\begin{small}
\begin{align*}
M&=\left\lfloor \frac{1}{2}\log_{2} \frac{T}{e}\right\rfloor; 
\hspace{2mm}\psi_{0}=\frac{T\epsilon_{0}}{128\Big(\log(\frac{3}{16}K\log K)\Big)^2}; \\
\ell_{0}&=\left\lceil \frac{2\psi_0\log( T\epsilon_{0})}{\epsilon_{0}} \right\rceil;
\hspace{2mm}N_{0}=K\ell_{0}
\end{align*}
\end{small}
%$M=\left\lfloor \frac{1}{2}\log_{2} \frac{T}{e}\right\rfloor $,  
%$\psi_{0}=\frac{T\epsilon_{0}}{(\log(\frac{3}{16}K\log K)^2}$,
% $\ell_{0}=\left\lceil \frac{2\psi\log( T\epsilon_{0})}{\epsilon_{0}} \right\rceil$ and 
% $N_{0}=K\ell_{0} $. Pull each arm once.
\State Pull each arm once
\vspace{-2mm}
\State \For{$t=K+1,..,T$}
\State Pull arm $j\in\argmin_{i\in B_{m}}\Big\lbrace |\hat{r}_{i} - \tau | - 2s_{i}\Big\rbrace$
% \State where $s_j=\sqrt{\frac{\rho\psi_{m}\hat{v}_{j}\log{( T\epsilon_{m})}}{4 n_{j}} + \frac{\rho\psi_{m} \log{(T\epsilon_{m})}}{4 n_{j}}}$
\State $t\leftarrow t+1$ 
\vspace{-4mm}
%\ArmElim
%\State For each arm $i \in B_{m}$, remove arm ${i}$ from $B_{m}$ if
%\begin{align*}
%\hat{r}_{i} + c_i  < \tau - c_i \mbox{ or } \hat{r}_{i} - c_i  > \tau + c_i \\
%\text{where $c_i=\sqrt{\frac{\rho_{\mu}\psi_{m}\log{( T\epsilon_{m})}}{2 n_{i}}}$}
%\end{align*}
%\EndArmElim
%\ArmElimV
%\State \For{$i\in B_m$}
%\State For each arm $i \in B_{m}$, remove arm ${i}$ from $B_{m}$ if
\State \For{$i\in B_m$}
\vspace{-4mm}
\State \If{$(\hat{r}_{i} + s_i  < \tau - s_i)$ or $(\hat{r}_{i} - s_i > \tau + s_i)$}
\State $B_m\leftarrow B_m\backslash\{i\}$\hspace{4mm} (Arm deletion)
\EndIf
\EndFor
%\begin{align*}
%\hat{r}_{i} + s_i  < \tau - s_i,\hspace{1mm} \mbox{ or } \hspace{1mm}\hat{r}_{i} - s_i  > \tau + s_i \\
%% \text{where $s_i=\sqrt{\frac{\rho\psi_{m}\hat{v}_{i}\log{( T\epsilon_{m})}}{4 n_{i}} + \frac{\rho\psi_{m} \log{(T\epsilon_{m})}}{4 n_{i}}}$}
%\end{align*}
%\EndFor
%\EndArmElimV
\vspace{-2mm}
\State \If{$t\geq N_{m}$ and $m \leq M$}
%\ResetParam
\State \textbf{Reset Parameters}
\State $\epsilon_{m+1}\leftarrow\frac{\epsilon_{m}}{2}$
\State $B_{m+1} \leftarrow B_{m}$
\State $\psi_{m+1}\leftarrow \frac{T\epsilon_{m+1}}{128(\log(\frac{3}{16}K\log K))^{2}}$
\State $\ell_{m+1}\leftarrow\left\lceil \frac{2\psi_{m+1}\log( T\epsilon_{m+1})}{\epsilon_{m+1}} \right\rceil$
\State $N_{m+1} \leftarrow t + |B_{m+1}|\ell_{m+1}$
\State $m \leftarrow m+1$
%\EndResetParam
\EndIf
\EndFor
\State \textbf{Output:} $\hat{S}_{\tau}=\lbrace i: \hat{r}_{i}\geq \tau \rbrace$.
\end{algorithmic}
\end{algorithm}


%Also because of the said condition, like \cite{liu2016modification} we also claim that AugUCB is an anytime algorithm.



\section{Theoretical Results}
\label{tbandit:results}
% \subsection{Problem Complexity}

Let us begin by recalling the following definitions of the  \emph{problem complexity} as introduced in \cite{locatelli2016optimal}:
\begin{align*}
H_{1} = \sum_{i=1}^{K}\dfrac{1}{\Delta_{i}^{2}} \hspace{1mm}\text{     and }  \hspace{1mm}
H_{CSAR,2} =\min_{i\in \mathcal{A}}\dfrac{i}{{\Delta_{(i)}^{2}}} 
\end{align*}
where $(\Delta_{(i)}: i\in\mathcal{A})$ is obtained by arranging $(\Delta_i:i\in\mathcal{A})$ in an increasing order. Also, from \cite{chen2014combinatorial} we have
\begin{align*}
H_{CSAR,2}=\max_{i\in\mathcal{A}}\frac{i}{\Delta_{(i)}^2}.
\end{align*}
$H_{CSAR,2}$ is the complexity term appearing in the bound for the CSAR algorithm. The relation between the above complexity terms are as follows (see \cite{locatelli2016optimal}):
%
%$H_1$ and $H_2$ is same as the problem complexity defined in \cite{locatelli2016optimal} for the thresholding bandit problem while $H_{CSAR,2}=\max_{i}\frac{i}{\Delta_{(i)}^2}$ is defined in \cite{chen2014combinatorial}. Also we know from \cite{locatelli2016optimal} that,
\begin{align*}
H_{1}\leq \log(2K)H_{2} \mbox{ and }
 H_1 \leq \log(K)H_{CSAR,2}.
\end{align*}

As ours is a variance-aware algorithm, we require $H_{1}^{\sigma}$ (as defined in \cite{gabillon2011multi}) that incorporates reward variances into its expression as given below:
\begin{align*}
 H_{\sigma,1}=\sum_{i=1}^{K}\frac{\sigma_{i}+\sqrt{\sigma_{i}^{2}+(16/3)\Delta_{i}}}{\Delta_{i}^{2}}.
\end{align*}
Finally, analogous to $H_{CSAR,2}$, in this paper we introduce the complexity term $H_{\sigma,2}$, which is given by
%and $H_{2}^{\sigma}$ (introduced in this paper) as,
\begin{align*}
%& H_{1}^{\sigma}=\sum_{i=1}^{K}\frac{\sigma_{i}+\sqrt{\sigma_{i}^{2}+(16/3)\Delta_{i}}}{\Delta_{i}^{2}}\\
H_{\sigma,2}=\max_{i\in \mathcal{A}} \frac{i}{\tilde{\Delta}_{(i)}^{2}}%& H_{2}^{\sigma}=\min_{i\in \mathcal{A}} i\frac{12\sigma_{(i)}^{2} + \Delta_{(i)}}{12\Delta_{(i)}^{2}}
\end{align*}
where $\tilde{\Delta}_{i}^{2}=\frac{\Delta_{i}^{2}}{\sigma_{i}+\sqrt{\sigma_{i}^{2}+(16/3)\Delta_{i}}}$, and $(\tilde{\Delta}_{(i)})$ is an increasing ordering of $(\tilde{\Delta}_{i})$. Following the results in \cite{audibert2010best}, we can show that
\begin{align*}
H_{\sigma,2}\le H_{\sigma,1}\le\overline{\log}(K) H_{\sigma,2} \le \log(2K) H_{\sigma,2}.
\end{align*}


%Similar to the relation between $H_1$ and $H_2$, it can be shown that
%%which also gives us that 
%$H_{2}^{\sigma} \leq H_{1}^{\sigma} \leq \log(2K) H_{2}^{\sigma}$.
%
%Also, from \cite{audibert2010best} we know that,
%\begin{align*}
%\sum_{i=1}^{K}\tilde{\Delta}_{i}^{-2} = \tilde{\Delta}_{(2)}^{-2} + \sum_{i=2}^{K}\frac{1}{i}i\tilde{\Delta}_{(i)}^{-2} &\leq \bar{\log K}\min_{i}i\tilde{\Delta}_{(i)}^{-2}\\
%& \leq \log(2K) H_{2}^{\sigma}, \text{ as $\bar{\log K} \leq \log(2K)$}
%\end{align*}


%\subsection{Theorem 1}
Our main result is summarized in the following theorem where we prove an  upper bound on the expected loss. 
\begin{theorem}
\label{tbandit:Result:Theorem:1}
For $K\geq 4$ and
%with $\rho_{\mu}=\frac{1}{8}$ and 
$\rho={1}/{3}$,
the expected loss of the AugUCB algorithm is given by,
%\begin{small}
\begin{align*}
\E[\Ls(T)]
%\exp\bigg( -\frac{T\log (2 K\sqrt{\log K})}{2H_2 K (\log K)^{3/2}} + \log\bigg(K\big(\log_2\frac{T}{e}+1\big)\bigg)\bigg)\\
%& + \exp\bigg(- \frac{5T\log ( K\sqrt{\log K})}{H_{2}^{\sigma} K(\log K)^{3/2}}  + \log\bigg(K\big(\log_2\frac{T}{e}+1\big)\bigg)\bigg).
& \leq 2KT
% \bigg\lbrace\exp\bigg( -\frac{T}{ 64 H_2 a}\bigg)
% + 2
 \exp\bigg(- \frac{T}{4096 \log( K\log K) H_{\sigma,2}} \bigg).
 %\bigg\rbrace
\end{align*}
%where $a=\log(\frac{3}{16} K\log K)$.
%\end{small}
%For every $0<\eta <1$ and $\gamma > 1$, there exists time $t$ such that for all $T>t$ the simple regret of AugUCB is upper bounded by,
%\begin{small}
%\begin{align*}
%& SR_{AugUCB} \leq \sum_{i=1}^{K} \Delta_{i}\bigg\lbrace \exp\bigg(-4\rho\log (\psi T\frac{\Delta_{i}^{4}}{16\rho^{2}})-\dfrac{c_{0}\sqrt{T}}{16\rho H_{2}}\\
%& + \log \big( 16\gamma C_1\log_{2}\dfrac{T}{e} \big) \bigg) + \exp\bigg(- \dfrac{3\rho_v}{2} \bigg(\dfrac{2\sigma_{i}^{2}+\Delta_{i}+2}{6\sigma_{i}^{2}+\Delta_{i}}\bigg)\log(\psi T\frac{\Delta_{i}^{4}}{16\rho_{v}^{2}})\\
%& -\dfrac{c_{0}\sqrt{T}}{16\rho_v H_{2}} + \log\big ( 32\gamma C_2\log_{2}\dfrac{T}{e} \big)  \bigg)\bigg\rbrace
%\end{align*}
%\end{small}
%with probability at least $1-\eta$, where $c_{0}>0$ is a constant and $C_1=\dfrac{K\rho\log (\psi T \frac{\Delta_{i}^{4}}{16\rho^{2}})}{T\Delta_{i}^{2}}$ and $C_2= \dfrac{K\rho_v\log (\psi T \frac{\Delta_{i}^{4}}{16\rho_{v}^{2}})}{T\Delta_{i}^{2}}$.
\end{theorem}

\begin{proof}
The proof comprises of two modules. In the first module we investigate the necessary conditions for arm elimination within a specified number of rounds, which is motivated by the technique in \cite{auer2010ucb}. Bounds on the arm-elimination probability is then obtained; however, since we use variance estimates, we invoke the Bernstein inequality (as in \cite{audibert2009exploration}) rather that the Chernoff-Hoeffding bounds (which is appropriate for the UCB-Improved \citep{auer2010ucb}). In the second module, as in \cite{locatelli2016optimal}, we first define a favourable event that will yield an upper bound on the expected loss. Using union bound, we then incorporate the result from module-1 (on the arm elimination probability), and finally derive the result through a series of simplifications.
%In the final module we conclude by combining the results for the first two modules. 
The details are as follows. 


\textbf{Arm Elimination:} Recall the notations used in the algorithm, Also, for each arm $i\in\mathcal{A}$, define $m_{i}=\min\left\lbrace m| \sqrt{\rho\epsilon_{m}}<\frac{\Delta_{i}}{2}\right\rbrace$. In the $m_i$-th round, whenever $n_i=\ell_{m_i}\ge\frac{2\psi_{m_i}\log{(T\epsilon_{m_{i}})}}{\epsilon_{m_{i}}}$, we obtain (as $\hat{v}_i\in[0,1]$)
%
%\begin{align*}
%s_{i}&=\sqrt{\dfrac{\rho \psi_{m_i} \hat{v}_{i} \epsilon_{m_{i}}\log ( T\epsilon_{m_{i}})}{4 n_{i}} + \dfrac{\rho \psi_{m_i}\log{( T\epsilon_{m_{i}})}}{4 n_{i}}} \\
%&\leq \sqrt{\dfrac{\rho\psi_{m_i} \epsilon_{m_{i}}\log ( T\epsilon_{m_{i}})}{4*2 \log(\psi_{m_i} T\epsilon_{m_{i}})} + \dfrac{\rho\psi_{m_i}\epsilon_{m_{i}} \log{( T\epsilon_{m_{i}})}}{4*2\psi_{m_i} \log( T\epsilon_{m_{i}})} } \text{, as }\hat{V}_{i}\in [0,1].\\
%& \leq \sqrt{\dfrac{\rho_v \epsilon_{g_{i}}}{8} + \dfrac{\rho_v \epsilon_{g_{i}}}{8} } \leq \dfrac{\sqrt{\rho_v \epsilon_{g_{i}}}}{2}< \dfrac{\Delta_{i}}{4} \text{, as }\rho_v\in (0,1].
%%& \leq \sqrt{\rho_v \epsilon_{g_{i}+1}} < \dfrac{\Delta_{i}}{4} \text{, as }\rho_v\in (0,1].
%\end{align*}
%
\begin{align}
\label{si_bound_equn}
s_i 
&\le \sqrt{\frac{\rho(\hat{v}_i+1)\epsilon_{m_i}}{8}}
% +\frac{\rho\epsilon_{m_i}}{8}}
  \le \frac{\sqrt{\rho\epsilon_{m_i}}}{2} < \frac{\Delta_i}{4}.
\end{align}

First, let us consider a bad arm $i\in\mathcal{A}$ (i.e., $r_i<\tau$). We note that, in the $m_i$-th round  whenever 
$\hat{r}_i \le r_i +2s_i$, then arm $i$ is eliminated as a bad arm. This is easy to verify as follows: using (\ref{si_bound_equn}) we obtain,
\begin{align*}
\hat{r}_{i}\leq r_{i} + 2s_{i} 
%&= r_{i} + 4s_{i} - 2s_{i} \\
< r_{i} + \Delta_{i} - 2s_{i} 
= \tau - 2s_{i} % \geq \tau + s_{i}
\end{align*}
which is precisely one of the elimination conditions in Algorithm~\ref{alg:augucb}. Thus, the probability that a bad arm is not eliminated correctly in the $m_i$-th round (or before) is given by

%%%%%%%%%%%%%%%%% Favorable event is defined here
%We note that in the $g_i$-th round arm $i$ can be pulled no more than $\ell_{g_i}$ number of times. 
%
%
%According to the algorithm, the number of rounds is $m=\lbrace 0,1,2,.. M\rbrace $ where $M=\bigg\lfloor \frac{1}{2}\log_{2} \frac{T}{e}\bigg\rfloor$. So, $\epsilon_{m}\geq 2^{-M}\geq \sqrt{\frac{e}{T}}$. Also each round $m$ consists of $|B_{m}|\ell_{m}$ timesteps where $\ell_{m} = \left\lceil\frac{2\psi_{m}\log( T \epsilon_{m})}{\epsilon_{m}}\right\rceil$, $B_{m}$ is the set of all surviving arms and let $a=(\log(\frac{3}{16} K\log K))$.
%
%
%Let $c_{i} = \sqrt{\frac{\rho_{\mu}\psi_{m} \log{(T\epsilon_{m})}}{2 n_{i}}}$ denote the confidence interval, where $n_{i}$ is the number of times an arm $i$ is pulled. Let $\mathcal{A}^{'}=\lbrace i\in \mathcal{A}|\Delta_{i}\geq b\rbrace$, for $b\geq \sqrt{\frac{e}{T}}$. Define $m_{i}=\min\lbrace m| \sqrt{\rho_{\mu}\epsilon_{m}}<\frac{\Delta_{i}}{2}\rbrace$.
%% Let $m_{i}$ be the minimum round such that an arm $i$ gets eliminated such that. 
%
%% Let $s_{i}=\sqrt{\frac{\rho_v\psi_{g} \hat{V_{i}} \log{( T\epsilon_{g})}}{4 n_{i}} + \frac{\rho_v\psi_{g} \log{( T\epsilon_{g})}}{4 n_{i}}}$ and 
%% $g_{i}$ be the minimum round that an arm $i$ gets eliminated such that $g_{i}=min\lbrace g| \sqrt{\rho_{v}\epsilon_{g}}<\frac{\Delta_{i}}{2}\rbrace$. 
%%In this proof sub-optimal arms refer to the arms whose $r_{i}$ is lower than the threshold $\tau$.
%
%%At the end of any round $\max\lbrace m_{i},g_{i}\rbrace$, for any arm $i$, two cases are possible.
%
%Let $\xi_{1}$ and $\xi_{2}$ be the favorable event such that,
%\begin{align*}
%\xi_{1}&=\bigg\lbrace \forall i\in \mathcal{A}, \forall m=0,1,2,..,M: |\hat{r_i} - r_i| \leq 2c_i\bigg\rbrace\\
%\xi_{2}&=\bigg\lbrace \forall i\in \mathcal{A}, \forall m=0,1,2,..,M: |\hat{r_i} - r_i| \leq  2s_i\bigg\rbrace
%\end{align*}
%
%So, $\xi_{1}$ and $\xi_{2}$ signifies the event any arm $i$ will get eliminated from $B_m$.
%%%%%%%%%%%%%%%%%%%%%%







%%%%%%%%%%%%%%%%%
%\subsubsection{\textit{Arm i is not eliminated on or before round $\max\lbrace m_{i},g_{i}\rbrace$}}
%
%For any arm $i$, if it is eliminated from active set $B_{m_{i}}$ then one of the below two events has to occur,
%%\begin{small}
%\begin{align}
%\hat{r}_{i} + c_{i} < \tau - c_{i}, \label{eq:armelim-casea}\\
%\hat{r}_{i} - c_{i} > \tau + c_{i}, \label{eq:armelim-caseb}
%\end{align}
%%\end{small}
%For (\ref{eq:armelim-casea}) we can see that it eliminates arms that have performed poorly and removes them  from $B_{m_{i}}$. Similarly, (\ref{eq:armelim-caseb}) eliminates arms from $B_{m_{i}}$ that have performed very well compared to threshold $\tau$.
%
%%Each round consists of $|B_{m_{i}}|\ell_{m_{i}}$ timesteps. 
%In the $m_{i}$-th round an arm $i$ can be pulled no more than $\ell_{m_{i}}$ times. So when $n_{i}=\ell_{m_{i}}$, putting the value of $\ell_{m_{i}}\ge\frac{2\psi_{m_i}\log{( T\epsilon_{m_{i}})}}{\epsilon_{m_{i}}}$ in $c_{i}$ we get, 
%%\begin{small}
%\begin{align*}
%c_{i}
%&=\sqrt{\frac{\rho_{\mu}\psi_{m_i}\epsilon_{m_{i}}\log ( T\epsilon_{m_{i}})}{2 n_{i}}}
%\le\sqrt{\frac{\rho_{\mu}\psi_{m_i}\epsilon_{i}\log ( T\epsilon_{m_{i}})}{2*2 \psi_{m_i} \log( T\epsilon_{m_{i}})}}\\
%& \le\frac{\sqrt{\rho_{\mu}\epsilon_{m_{i}}}}{2}
%% % \leq \sqrt{\rho_{\mu}\epsilon_{m_{i}+1}} 
%< \frac{\Delta_{i}}{4} \text{, as }\rho_{\mu}\in (0,1].
%\end{align*}
%%\end{small}
%Again, for ${i} \in \mathcal{A}^{'}$ for the  elimination condition in (\ref{eq:armelim-casea}), 
%%\begin{small}
%%\begin{align*}
%%\hat{r}_{i} + c_{i}&\leq r_{i} + 2c_{i} = r_{i} + 4c_{i} - 2c_{i} \\
%%&< r_{i} + \Delta_{i} - 2c_{i} = \tau -2c_{i} \leq \tau - c_{i}
%%\end{align*}
%%\end{small}
%%\begin{small}
%\begin{align*}
%\hat{r}_{i} &\leq r_{i} + 2c_{i} = r_{i} + 4c_{i} - 2c_{i} \\
%&< r_{i} + \Delta_{i} - 2c_{i} = \tau -2c_{i}.
%\end{align*}
%%\end{small}
%Similarly, for ${i} \in \mathcal{A}^{'}$ for the  elimination condition in (\ref{eq:armelim-caseb}), 
%%\begin{small}
%\begin{align*}
%\hat{r}_{i} &\geq r_{i} - 2c_{i} = r_{i} - 4c_{i} + 2c_{i} \\
%&> r_{i} - \Delta_{i} + 2c_{i}= \tau + 2c_{i}.
%\end{align*}
%%\end{small}
%
%
%%Now, arm elimination condition is being checked at every timestep, in the $m_{i}$-th round as soon as $n_{i}=\ell_{m_{i}}$, arm $i$ gets eliminated. 
%Applying Chernoff-Hoeffding bound and considering independence of complementary of the event in (\ref{eq:armelim-casea}),
%%\begin{small}
%\begin{align*}
%%\mathbb{P}\lbrace\hat{r}_{i}\geq r_{i} - 2c_{i}\rbrace &\leq exp(-2(\tau + 2c_{i})^{2}n_{i})\\
%&\mathbb{P}\lbrace\hat{r}_{i}> r_{i} + 2c_{i}\rbrace \leq \exp(-4 c_{i}^{2}n_{i})\\
%&\leq \exp(-8 * \dfrac{\rho_{\mu}\psi_{m_i}\log ( T\epsilon_{m_{i}})}{2 n_{i}} *n_{i})\\
%&\leq \exp\big(-4\rho_{\mu}\psi_{m_i}\log ( T\epsilon_{m_{i}})\big)\\
%&\leq \exp\left(-\rho_{\mu}\frac{T\epsilon_{m_{i}}}{32 a^2}\log ( T\epsilon_{m_{i}})\right),\\
%&\text{putting the value of $\psi_{m_i}=\frac{T\epsilon_{m_i}}{128(\log(\frac{3}{16} K\log K))^{2}}$}
%\end{align*}
%%\end{small}
%Similarly for the condition in (\ref{eq:armelim-caseb}), $\mathbb{P}\lbrace\hat{r}_{i}< r_{i} - 2c_{i}\rbrace\leq \exp\left(-\frac{T\rho_{\mu}\epsilon_{m_{i}}}{32 a^2 }\log ( T\epsilon_{m_{i}})\right)$.
%
%Summing the above two expressions, the probability that arm ${i}$ is not eliminated on or before $m_{i}$-th is $\left(2\exp\left(-\frac{T\rho_{\mu}\epsilon_{m_{i}}}{32 a^2 }\log ( T\epsilon_{m_{i}})\right)\right)$. 
%%%%%%%%%%%%%%%%%%%%%

%%%%%%%%%%%%
%Again for any arm $i$, if it is eliminated from active set $B_{g_{i}}$ then the below two events have to come true,
%%\begin{small}
%\begin{align}
%\hat{r}_{i} + s_{i} < \tau - s_{i}, \label{eq:armelim-var-casea}\\
%\hat{r}_{i} - s_{i} > \tau + s_{i}, \label{eq:armelim-var-caseb}
%\end{align}
%%\end{small}
%%
%% For \ref{eq:armelim-var-casea} we can see that it eliminates arms that have performed poorly and removes them them from $B_{g_{i}}$. Similarly, \ref{eq:armelim-var-caseb} eliminates arms from $B_{g_{i}}$ that have performed very well compared to threshold $\tau$.
%%But, we know that $\epsilon_{m_{i}}=\epsilon_{g_{i}}$ and round consist of $|B_{g_{i}}|\ell_{g_{i}}$ timesteps. 
%In the $g_{i}$-th round an arm $i$ can be pulled no more than $\ell_{g_{i}}$ times. So when $n_{i}=\ell_{g_{i}}$, putting the value of $\ell_{g_{i}}\ge\frac{2\psi_{m_i}\log{( T\epsilon_{g_{i}})}}{\epsilon_{g_{i}}}$ in $s_{i}$ we get, 
%%\begin{small}
%\begin{align*}
%s_{i}&=\sqrt{\dfrac{\rho_v \psi_{g_i} \hat{V}_{i} \epsilon_{g_{i}}\log ( T\epsilon_{g_{i}})}{4 n_{i}} + \dfrac{\rho_v \psi_{g_i}\log{( T\epsilon_{g_{i}})}}{4 n_{i}}} \\
%&\leq \sqrt{\dfrac{\rho_v\psi_{g_i} \epsilon_{g_{i}}\log ( T\epsilon_{g_{i}})}{4*2 \log(\psi_{g_i} T\epsilon_{g_{i}})} + \dfrac{\rho_v \psi_{g_i}\epsilon_{g_{i}} \log{( T\epsilon_{g_{i}})}}{4*2\psi_{g_i} \log( T\epsilon_{g_{i}})} } \text{, as }\hat{V}_{i}\in [0,1].\\
%& \leq \sqrt{\dfrac{\rho_v \epsilon_{g_{i}}}{8} + \dfrac{\rho_v \epsilon_{g_{i}}}{8} } \leq \dfrac{\sqrt{\rho_v \epsilon_{g_{i}}}}{2}< \dfrac{\Delta_{i}}{4} \text{, as }\rho_v\in (0,1].
%%& \leq \sqrt{\rho_v \epsilon_{g_{i}+1}} < \dfrac{\Delta_{i}}{4} \text{, as }\rho_v\in (0,1].
%\end{align*}
%%\end{small}
%
%Again, for ${i} \in \mathcal{A}^{'}$ for the elimination condition in (\ref{eq:armelim-var-casea}),
%%\begin{small}
%\begin{align*}
%\hat{r}_{i} &\leq r_{i} + 2s_{i} = r_{i} + 4s_{i} - 2s_{i} \\
%&< r_{i} + \Delta_{i} - 2s_{i} = \tau -2s_{i} % \leq \tau - s_{i}
%\end{align*}
%%\end{small} 
%
%
%Also, for ${i} \in \mathcal{A}^{'}$ for the elimination condition in (\ref{eq:armelim-var-caseb}), 
%%\begin{small}
%\begin{align*}
%\hat{r}_{i}&\geq r_{i} - 2s_{i} = r_{i} - 4s_{i} + 2s_{i} \\
%&> r_{i} - \Delta_{i} + 2s_{i}\geq \tau + 2s_{i} % \geq \tau + s_{i}
%\end{align*}
%%\end{small}
%%%%%%%%%%%%%%%%


%Since, arm elimination condition is being checked at every timestep, in the $g_{i}$-th round as soon as $n_{i}=\ell_{g_{i}}$, arm $i$ gets eliminated. 
% Applying Bernstein inequality and considering independence of complementary of the event in (\ref{eq:armelim-var-casea}),
%\begin{small}
\noindent
\begin{align}
\mathbb{P}(\hat{r}_{i}> r_{i} + 2s_{i})
% &= \mathbb{P}\bigg( \hat{r}_{i} > r_{i}+ 2\sqrt{\dfrac{\rho\psi_{m_i} \hat{v}_{i}\log( T\epsilon_{m_{i}}) + \rho\psi_{m_i} \log{( T\epsilon_{m_{i}})}}{4n_{i}} } \bigg)\nonumber\\
&\leq \mathbb{P}\left( \hat{r}_{i} > r_{i}+ 2\bar{s}_i\right)  % \label{eq:prob_eq1}\\ 
+ \mathbb{P}\left( \hat{v}_{i}\geq \sigma_{i}^{2}+\sqrt{\rho\epsilon_{m_{i}}}\right)\label{eq:prob_eq2}
\end{align}
where 
\begin{align*}
\bar{s}_i=\sqrt{\dfrac{\rho\psi_{m_i} (\sigma_{i}^{2}+\sqrt{\rho\epsilon_{m_{i}}} + 1)\log( T\epsilon_{m_{i}})}{4n_{i}}}
\end{align*}
%\end{small}
Note that, substituting $n_i=\ell_{m_i}\ge \frac{2\psi_{m_i}\log{(T\epsilon_{m_{i}})}}{\epsilon_{m_{i}}}$, $\bar{s}_i$ can be simplified to obtain,
\begin{align}
2\bar{s}_i
% &\le 2\sqrt{\dfrac{\rho\psi_{m_i} (\sigma_{i}^{2}+\sqrt{\rho\epsilon_{m_{i}}})\log( T\epsilon_{m_{i}})}{\frac{8\psi_{m_i}\log( T \epsilon_{m_{i}})}{\epsilon_{m_{i}}}} }
%+ \dfrac{\rho\psi_{m_i} \log{( T\epsilon_{m_{i}})}}{\frac{8\psi_{m_i}\log( T \epsilon_{m_{i}})}{\epsilon_{m_{i}}}}}
\leq \dfrac{\sqrt{\rho\epsilon_{m_{i}}(\sigma_{i}^{2}+\sqrt{\rho\epsilon_{m_{i}}} + 1)}}{2}\leq \sqrt{\rho \epsilon_{m_{i}}}.
\label{si_bar_equn}
\end{align}

%Now, we know that in the $g_{i}$-th round,
%%\begin{small}
%\begin{align*}
%& 2\sqrt{\dfrac{\rho_v\psi_{g_i} [\sigma_{i}^{2}+\sqrt{\rho_{v}\epsilon_{g_{i}}}]\log( T\epsilon_{g_{i}})}{4n_{i}} + \dfrac{\rho_v\psi_{g_i}  \log{(T\epsilon_{g_{i}})}}{4 n_{i}}}\\ &\leq  2\sqrt{\dfrac{\rho_v\psi_{g_i} [\sigma_{i}^{2}+\sqrt{\rho_{v}\epsilon_{g_{i}}}]\log( T\epsilon_{g_{i}})}{\frac{8\psi_{g_i}\log( T \epsilon_{g_{i}})}{\epsilon_{g_{i}}}} + \dfrac{\rho_v\psi_{g_i} \log{( T\epsilon_{g_{i}})}}{\frac{8\psi_{g_i}\log( T \epsilon_{g_{i}})}{\epsilon_{g_{i}}}}}\\
%& \leq \dfrac{\sqrt{\rho_v \epsilon_{g_{i}}[\sigma_{i}^{2}+\sqrt{\rho_{v}\epsilon_{g_{i}}} + 1]}}{2}\leq \sqrt{\rho_v \epsilon_{g_{i}}}
%\end{align*}
%%\end{small}
%--------------------

The first term in the LHS of (\ref{eq:prob_eq2}) can be bounded using the Bernstein inequality as below:
\begin{align}
&\mathbb{P}\left( \hat{r}_{i} > r_{i}+ 2\bar{s}_i\right)\nonumber \\
&\le \exp\left(- \dfrac{(2\bar{s}_i)^2 n_i}{2\sigma_i^2+\frac{4}{3}\bar{s}_i}\right)\nonumber \\
& \le \exp\left(- \dfrac{\rho\psi_{m_i} (\sigma_{i}^{2}+\sqrt{\rho\epsilon_{m_{i}}} + 1)\log( T\epsilon_{m_{i}})}{2\sigma_i^2+\frac{2}{3}\sqrt{\rho \epsilon_{m_{i}}}}\right)\nonumber \\
& \overset{(a)}{\leq} \exp\left(- \dfrac{3\rho T\epsilon_{m_i}}{256 a^2} \left(\dfrac{\sigma_{i}^{2}+\sqrt{\rho\epsilon_{m_{i}}}+1}{3\sigma_{i}^{2}+\sqrt{\rho \epsilon_{m_{i}}}}\right) \log( T\epsilon_{m_{i}}) \right) \nonumber \\
&:= \exp(-Z_i) 
\label{lhs1_equn}
\end{align}
where, for simplicity, we have used $\alpha_i$ to denoted the exponent in the inequality $(a)$.
Also, note that $(a)$ is obtained by using  $\psi_{m_i}=\frac{T\epsilon_{m_i}}{128a^{2}}$, where $a=(\log(\frac{3}{16} K\log K))$.
%For the term in (\ref{eq:prob_eq1}), by applying Bernstein inequality, we can write as,
%\begin{small}
%\begin{align*}
%&\mathbb{P}\bigg( \hat{r}_{i}> r_{i} + \bigg(2\sqrt{\frac{\rho_v\psi_{g_i} [\sigma_{i}^{2}+\sqrt{\rho_{v}\epsilon_{g_{i}}} + 1]\log( T\epsilon_{g_{i}})}{4n_{i}}  } \bigg)\bigg)\\
%%%%%%%%%%%%%%%%%%%%%%%%
% &\leq \exp\bigg(- \dfrac{\bigg(2\sqrt{\frac{\rho_v\psi_{g_i} [\sigma_{i}^{2}+\sqrt{\rho_{v}\epsilon_{g_{i}}} +1]\log( T\epsilon_{g_{i}})}{4n_{i}}}\bigg)^{2}n_{i}}{2\sigma_{i}^{2}+\frac{4}{3}\sqrt{\frac{\rho_v\psi_{g_i} [\sigma_{i}^{2}+\sqrt{\rho_{v}\epsilon_{g_{i}}}+1]\log( T\epsilon_{g_{i}})}{4n_{i}}}}\bigg) \\
%%%%%%%%%%%%%%%%%%%%%%%
%&\leq \exp\bigg(- \dfrac{\bigg(\rho_v\psi_{g_i} [\sigma_{i}^{2}+\sqrt{\rho_{v}\epsilon_{g_{i}}} + 1]\log( T\epsilon_{g_{i}})\bigg)}{2\sigma_{i}^{2}+\frac{2}{3}\sqrt{\rho_v \epsilon_{g_{i}}}} \bigg)\\
% &\leq \exp\bigg(- \dfrac{3\rho_v\psi_{g_i}}{2} \bigg(\dfrac{\sigma_{i}^{2}+\sqrt{\rho_{v}\epsilon_{g_{i}}}+1}{3\sigma_{i}^{2}+\sqrt{\rho_v \epsilon_{g_{i}}}}\bigg) \log( T\epsilon_{g_{i}}) \bigg)\\
%%%%%%%%%%%%%%%%%%%%%%%
% &\leq \exp\left(- \dfrac{3\rho_v T\epsilon_{g_i}}{256 a^2} \left(\dfrac{\sigma_{i}^{2}+\sqrt{\rho_{v}\epsilon_{g_{i}}}+1}{3\sigma_{i}^{2}+\sqrt{\rho_v \epsilon_{g_{i}}}}\right) \log( T\epsilon_{g_{i}}) \right),
% &\text{ putting the value of $\psi_{g_i}=\frac{T\epsilon_{g_i}}{128(\log(\frac{3}{16} K\log K))^{2}}$}
%%%%%%%%%%%%%%%%%%%%%%%%%%%%%%%%%%%%%%%%%%%%%%%%%%%%%%%%%%%%%%%%%%%%%%%%%%%%%%%%%%
%\begin{align*}
%&\mathbb{P}\bigg\lbrace \hat{r}_{i}> r_{i} + \bigg(2\sqrt{\frac{\rho_v\psi_{g_i} [\sigma_{i}^{2}+\sqrt{\rho_{v}\epsilon_{g_{i}}} + 1]\log( T\epsilon_{g_{i}})}{4n_{i}}  } \bigg)\bigg\rbrace\\
%&\leq \exp\bigg(- \dfrac{\bigg(2\sqrt{\frac{\rho_v\psi_{g_i} [\sigma_{i}^{2}+\sqrt{\rho_{v}\epsilon_{g_{i}}}]\log( T\epsilon_{g_{i}})}{4n_{i}} + \frac{\rho_v\psi_{g_i} \log{( T\epsilon_{g_{i}})}}{4 n_{i}}}\bigg)^{2}n_{i}}{2\sigma_{i}^{2}+\frac{4}{3}\sqrt{\frac{\rho_v\psi_{g_i} [\sigma_{i}^{2}+\sqrt{\rho_{v}\epsilon_{g_{i}}}]\log( T\epsilon_{g_{i}})}{4n_{i}}+\frac{\rho_v\psi_{g_i} \log{( T\epsilon_{g_{i}})}}{4 n_{i}}}}\bigg) \\
%&\leq \exp\bigg(- \dfrac{\bigg(\rho_v\psi_{g_i} [\sigma_{i}^{2}+\sqrt{\rho_{v}\epsilon_{g_{i}}} + 1]\log( T\epsilon_{g_{i}})\bigg)}{2\sigma_{i}^{2}+\frac{2}{3}\sqrt{\rho_v \epsilon_{g_{i}}}} \bigg)\\
%&\leq \exp\bigg(- \dfrac{3\rho_v\psi_{g_i}}{2} \bigg(\dfrac{\sigma_{i}^{2}+\sqrt{\rho_{v}\epsilon_{g_{i}}}+1}{3\sigma_{i}^{2}+\sqrt{\rho_v \epsilon_{g_{i}}}}\bigg) \log( T\epsilon_{g_{i}}) \bigg)\\
%&\leq \exp\left(- \dfrac{3\rho_v T\epsilon_{g_i}}{16 K\log K} \left(\dfrac{\sigma_{i}^{2}+\sqrt{\rho_{v}\epsilon_{g_{i}}}+1}{3\sigma_{i}^{2}+\sqrt{\rho_v \epsilon_{g_{i}}}}\right) \log( T\epsilon_{g_{i}}) \right),\\
%&\text{ putting the value of $\psi_{g_i}=\frac{T\epsilon_{g_i}}{128(\log(\frac{3}{16} K\log K))^{2}}$}
%%%%%%%%%%%%%%%%%%%%%%%%%%%%%%%%%%%%%%%%%%%%%%%%%%%%%%%%%%%%%%%%%%%%%%%%%%%%%%%%%%
%\end{align*}
%\end{small}
% where  the last inequality is obtained using 
% \begin{align*}
% \psi_{m_i}=\frac{T\epsilon_{m_i}}{128(\log(\frac{3}{16} K\log K))^{2}}.
% \end{align*}
 
 The second term in the LHS of (\ref{eq:prob_eq2}) can be simplified as follows:
% For the term in , by applying Bernstein inequality, we can write as,
%\begin{small}
\begin{align}
&\mathbb{P}\bigg\lbrace \hat{v}_{i}\geq \sigma_{i}^{2}+\sqrt{\rho\epsilon_{m_{i}}}\bigg\rbrace\nonumber\\
&\leq \mathbb{P}\bigg\lbrace \dfrac{1}{n_{i}}\sum_{t=1}^{n_{i}}(X_{i,t}-r_{i})^{2}-(\hat{r}_{i}-r_{i})^{2}\geq \sigma_{i}^{2}+\sqrt{\rho\epsilon_{m_{i}}}\bigg\rbrace\nonumber\\
&\leq \mathbb{P}\bigg\lbrace \dfrac{\sum_{t=1}^{n_{i}}(X_{i,t}-r_{i})^{2}}{n_{i}}\geq \sigma_{i}^{2}+\sqrt{\rho\epsilon_{m_{i}}} \bigg\rbrace\nonumber\\
&\overset{(a)}{\leq} \mathbb{P}\bigg\lbrace \dfrac{\sum_{t=1}^{n_{i}}(X_{i,t}-r_{i})^{2}}{n_{i}}\geq \sigma_{i}^{2} + 2\bar{s}_i\bigg\rbrace \nonumber\\
% &\bigg(2\sqrt{\dfrac{\rho_v\psi_{g_i} [\sigma_{i}^{2}+\sqrt{\rho_{v}\epsilon_{g_{i}}}]\log( T\epsilon_{g_{i}})}{4n_{i}}+\frac{\rho_v\psi_{g_i}  \log{(T\epsilon_{g_{i}})}}{4 n_{i}}}\bigg)\bigg\rbrace\\
&\overset{(b)}{\leq} \exp\bigg(- \dfrac{3\rho\psi_{m_i}}{2} \bigg(\dfrac{\sigma_{i}^{2}+\sqrt{\rho\epsilon_{m_{i}}}+1}{3\sigma_{i}^{2}+\sqrt{\rho \epsilon_{m_{i}}}}\bigg) \log( T\epsilon_{m_{i}}) \bigg)\nonumber \\
%&\leq \exp\bigg(- \dfrac{3\rho_vT\epsilon_{g_i}}{256 a^2 } \bigg(\dfrac{\sigma_{i}^{2}+\sqrt{\rho_{v}\epsilon_{g_{i}}}+1}{3\sigma_{i}^{2}+\sqrt{\rho_v \epsilon_{g_{i}}}}\bigg) \log( T\epsilon_{g_{i}}) \bigg)
& = \exp(-Z_i)
%&\text{ putting the value of $\psi_{g_i}=\frac{T\epsilon_{g_i}}{128(\log(\frac{3}{16} K\log K))^{2}}$}
\label{lhs2_equn}
\end{align}
%\end{small}
where inequality $(a)$ is obtained using (\ref{si_bar_equn}), while $(b)$ follows from the Bernstein inequality. 
  
Thus, using (\ref{lhs1_equn}) and (\ref{lhs2_equn}) in (\ref{eq:prob_eq2}) we obtain $\mathbb{P}(\hat{r}_{i}> r_{i} + 2s_{i})\le 2\exp(-Z_i)$.
% \begin{small}
% \begin{align*}
%& \mathbb{P}(\hat{r}_{i}> r_{i} + 2s_{i}) \le\\ 
%&2\exp\left(- \frac{3T\rho_v\epsilon_{g_{i}}}{256 a^2 } \left(\frac{\sigma_{i}^{2}+\sqrt{\rho_{v}\epsilon_{g_{i}}}+1}{3\sigma_{i}^{2}+\sqrt{\rho_v \epsilon_{g_{i}}}}\right) \log( T\epsilon_{g_{i}}) \right)
% \end{align*}
% \end{small}
 %
Proceeding similarly, for a good arm $i\in\mathcal{A}$, the probability that it is not correctly eliminated in the $m_i$-th round (or before) is also bounded by $\mathbb{P}(\hat{r}_{i}< r_{i} - 2s_{i})\le 2\exp(-Z_i)$. In general, for any $i\in\mathcal{A}$ we have
\begin{align}
\Pb(|\hat{r}_i-r_i|>2s_i) 
&\le4\exp(-Z_i).
\label{final_bound_equn}
\end{align}
  
  
%Similarly, the condition for the complementary event for the elimination case \ref{eq:armelim-var-caseb} holds such that $\mathbb{P}\lbrace\hat{r}_{i}< r_{i} - 2s_{i}\rbrace \leq 2\exp\left(- \frac{3T\rho_v\epsilon_{g_{i}}}{256 a^2 } \left(\frac{\sigma_{i}^{2}+\sqrt{\rho_{v}\epsilon_{g_{i}}}+1}{3\sigma_{i}^{2}+\sqrt{\rho_v \epsilon_{g_{i}}}}\right) \log( T\epsilon_{g_{i}}) \right)$.


\textbf{Favourable Event:} Following the notation in \cite{locatelli2016optimal} we define the event
\begin{align*}
\xi&=\bigg\lbrace \forall i\in \mathcal{A}, \forall t=1,2,..,T: |\hat{r_i} - r_i| \leq  2s_i\bigg\rbrace.
\end{align*}
Note that, on $\xi$ each arm $i\in \mathcal{A}$  is eliminated correctly in the $m_i$-th round (or before). Thus, it follows that $\mathbb{E}[\mathcal{L}(T)]\le P(\xi^c)$. Since $\xi^c$ can be expressed as an union of the events $(|\hat{r}_i-r_i|>2s_i)$ for all $i\in\mathcal{A}$ and all $t=1,2,\cdots,T$, using union bound we can write
\begin{align*}
&\mathbb{E}[\mathcal{L}(T)] 
\le \sum_{i\in\mathcal{A}}\sum_{t=1}^T \Pb(|\hat{r}_i-r_i|>2s_i) \\
&\le \sum_{i\in\mathcal{A}}\sum_{t=1}^T 4 \exp(-Z_i) \\
&\le 4T\sum_{i\in\mathcal{A}} \exp\left(- \dfrac{3\rho T\epsilon_{m_i}}{256 a^2} \left(\dfrac{\sigma_{i}^{2}+\sqrt{\rho\epsilon_{m_{i}}}+1}{3\sigma_{i}^{2}+\sqrt{\rho \epsilon_{m_{i}}}}\right) \log( T\epsilon_{m_{i}}) \right) \\
&\overset{(a)}{\le} 4T \sum_{i\in\mathcal{A}} \exp\left(- \frac{3T\Delta_{i}^{2}}{4096 a^2} \left(\frac{4\sigma_{i}^{2}+\Delta_{i}+4}{12\sigma_{i}^{2}+\Delta_{i}}\right) \log( \frac{3}{16} T\Delta_{i}^{2}) \right) \\
&\overset{(b)}{\le} 4T \sum_{i\in\mathcal{A}}\exp\bigg(- \frac{12T\Delta_{i}^{2}}{(12\sigma_{i}+ 12\Delta_{i})}\frac{\log (\frac{3}{16} K\log K)}{4096 a^2 } \bigg) \\
&\overset{(c)}{\le} 4T \sum_{i\in\mathcal{A}} \exp\bigg(- \frac{T\Delta_{i}^{2}\log ( \frac{3}{16} K\log K)}{4096 (\sigma_{i} + \sqrt{\sigma_{i}^{2} + (16/3)\Delta_{i}}) a^2} \bigg) \\
& \overset{(d)}{\le} 4T \sum_{i\in\mathcal{A}} \exp\bigg(- \frac{T\log ( \frac{3}{16} K\log K)}{4096 \tilde{\Delta}_i^{-2} a^2} \bigg) \\
& \overset{(e)}{\le}4T \sum_{i\in\mathcal{A}} \exp\bigg(- \frac{T\log ( \frac{3}{16} K\log K)}{4096 \max_{j}(j\tilde{\Delta}_{(j)}^{-2}) (\log(\frac{3}{16} K\log K))^{2}} \bigg) \\
& \overset{(f)}{\le}4KT \exp\bigg(- \frac{T}{4096 \log(K\log K)H_{\sigma,2}}\bigg).
\end{align*}
The justification for the above simplifications are as follows:
%\begin{itemize}

\noindent
 $\bullet$ $(a)$ is obtained by noting that in round $m_i$ we have 
 $\frac{\Delta_i}{4}\leq\sqrt{\rho\epsilon_{m_{i}}}<\frac{\Delta_i}{2}.$

\noindent
 $\bullet$ For $(b)$, we note that the function $x\mapsto x\exp(-Cx^2)$, where $x\in[0,1]$, is  decreasing on $[1/\sqrt{2C},1]$ for any $C>0$ (see \cite{bubeck2011pure,auer2010ucb}). Thus, using $C=\lfloor T/e\rfloor$ and $\min_{j\in \mathcal{A}}\Delta_j =\Delta =\sqrt{\frac{K\log K}{T}} > \sqrt{\frac{e}{T}}$,
%\forall i\in \mathcal{A}$ 
we obtain (b).

\noindent
 $\bullet$
To obtain $(c)$ we have used the inequality $\Delta_i\le \sqrt{\sigma_{i}^{2} + (16/3)\Delta_{i}}$ (which holds because $\Delta_i\in[0,1]$).

\noindent
 $\bullet$
 $(d)$ is obtained simply by substituting $\tilde{\Delta}_i=\frac{\Delta_{i}^{2}}{\sigma_{i}+\sqrt{\sigma_{i}^{2}+(16/3)\Delta_{i}}}$ and $a=\log(\frac{3}{16} K\log K)$.
 
 \noindent
 $\bullet$
 Finally, to obtain $(e)$ and $(f)$, note that 
%\begin{align*}
$\tilde{\Delta}_i^{-2}\le i\tilde{\Delta}_i^{-2} \le \max_{j\in\mathcal{A}}j\Delta_{(j)}^{-2}=H_{\sigma,2}.$
%\end{align*}
%\end{itemize}
\end{proof}

%Again  summing the above expressions, the probability that an arm ${i}$ is not eliminated on or before $g_{i}$-th round based on the (\ref{eq:armelim-var-casea}) and (\ref{eq:armelim-var-caseb}) elimination condition is  $4\exp\left(- \frac{3T\rho_v\epsilon_{g_{i}}}{256 a^2 } \left(\frac{\sigma_{i}^{2}+\sqrt{\rho_{v}\epsilon_{g_{i}}}+1}{3\sigma_{i}^{2}+\sqrt{\rho_v \epsilon_{g_{i}}}}\right) \log( T\epsilon_{g_{i}}) \right)$. 
  
%%%%%%%%%%%%%%%%%%%%%%%%%%%%%%%%%%%%%%%%%%%%%%%%%%%%%%%%%%%%%%%%%%%%%%%%%%%%%%%%%%%%%%
%Not Required for probability of error for AugUCB
%%%%%%%%%%%%%%%%%%%%%%%%%%%%%%%%%%%%%%%%%%%%%%%%%%%%%%%%%%%%%%%%%%%%%%%%%%%%%%%%%%%%%%

%We start with an upper bound on the number of plays $\delta_{\max\lbrace m_{i}, g_{i}\rbrace}$ in the $\max\lbrace m_{i}, g_{i}\rbrace$-th round. We know that the total number of arms surviving in the $\max\lbrace m_{i}, g_{i}\rbrace$-th arm is, 
%
%\begin{small}
%\begin{align*}
%&|B_{\max\lbrace m_{i}, g_{i}\rbrace}|=2K\exp\bigg(-4\rho_{\mu}\log (\psi T\epsilon_{m_{i}}^{2})\bigg)\\ 
%& + 4K\exp\bigg(- \frac{3\rho_v}{2} \big(\frac{\sigma_{i}^{2}+\sqrt{\rho_{v}\epsilon_{g_{i}}}+1}{3\sigma_{i}^{2}+\sqrt{\rho_v \epsilon_{g_{i}}}}\big) \log(\psi T\epsilon_{g_{i}}^{2}) \bigg)
%\end{align*}     
%\end{small}
%
%
%Again for AugUCB, we know that the number of pulls allocated for each surviving arm $i$ in the $m_{i}$-th round is $\ell_{m_{i}}=\frac{2\log (\psi T \epsilon_{m_{i}}^{2})}{\epsilon_{m_{i}}}$ or for the $g_{i}$-th round is $\ell_{g_{i}}=\frac{2\log (\psi T \epsilon_{g_{i}}^{2})}{\epsilon_{g_{i}}}$. Therefore, the proportion of plays $\delta_{\max\lbrace m_{i}, g_{i}\rbrace}$ in the $\max\lbrace m_{i}, g_{i}\rbrace$-th round can be written as,
%
%\begin{small}
%\begin{align*}
%&\delta_{\max\lbrace m_{i}, g_{i}\rbrace}=(|B_{m_{i}}|.\ell_{m_{i}}) + (|B_{g_{i}}|.\ell_{g_{i}})\\
%&\leq 2K\exp\bigg(-4\rho_{\mu}\log (\psi T\epsilon_{m_{i}}^{2})\bigg).\dfrac{2\log (\psi T \epsilon_{m_{i}}^{2})}{\epsilon_{m_{i}}}\\
% & + 4K\exp\bigg(- \dfrac{3\rho_v}{2} \bigg(\dfrac{\sigma_{i}^{2}+\sqrt{\rho_{v}\epsilon_{g_{i}}}+1}{3\sigma_{i}^{2}+\sqrt{\rho_v \epsilon_{g_{i}}}}\bigg) \log(\psi T\epsilon_{g_{i}}^{2})\bigg).\dfrac{2\log (\psi T \epsilon_{g_{i}}^{2})}{\epsilon_{g_{i}}} \\
%& \leq \dfrac{4K\log (\psi T \epsilon_{m_{i}}^{2})}{\epsilon_{m_{i}}}\exp\bigg(-4\rho_{\mu}\log (\psi T\epsilon_{m_{i}}^{2})\bigg)\\
%& + \dfrac{8K\log (\psi T \epsilon_{g_{i}}^{2})}{\epsilon_{g_{i}}}\exp\bigg(- \dfrac{3\rho_v}{2} \bigg(\dfrac{\sigma_{i}^{2}+\sqrt{\rho_{v}\epsilon_{g_{i}}}+1}{3\sigma_{i}^{2}+\sqrt{\rho_v \epsilon_{g_{i}}}}\bigg) \log(\psi T\epsilon_{g_{i}}^{2}) \bigg)
%\end{align*}
%\end{small}

%Now, in the $\max\lbrace m_{i}, g_{i}\rbrace$-th round $\sqrt{\rho_{\mu}\epsilon_{m_{i}}}\leq \frac{\Delta_{i}}{2}$ or $\sqrt{\rho_v\epsilon_{g_{i}}}\leq \frac{\Delta_{i}}{2}$. Hence,
%
%\begin{small}
%\begin{align*}
%&\delta_{\max\lbrace m_{i},g_{i}\rbrace} \leq \dfrac{4K\log (\psi T \frac{\Delta_{i}^{4}}{16\rho_{\mu}^{2}})}{\frac{\Delta_{i}^{2}}{4\rho_{\mu}}}\exp\bigg(-4\rho_{\mu}\log (\psi T\frac{\Delta_{i}^{4}}{16\rho_{\mu}^{2}})\bigg)\\
%& + \dfrac{8K\log (\psi T \frac{\Delta_{i}^{4}}{16\rho_{v}^{2}})}{\frac{\Delta_{i}^{2}}{4\rho_{v}}}\exp\bigg(- \dfrac{3\rho_v}{2} \bigg(\dfrac{\sigma_{i}^{2}+\frac{\Delta_{i}}{2}+1}{3\sigma_{i}^{2}+\frac{\Delta_{i}}{2}}\bigg) \log(\psi T\frac{\Delta_{i}^{4}}{16\rho_{v}^{2}}) \bigg)\\
%%%%%%%%%%%%%%%%%%%%%%%%%%%%%%%%%%%%%%%%
%&\leq 16 C_1\exp\bigg(-4\rho_{\mu}\log (\psi T\frac{\Delta_{i}^{4}}{16\rho_{\mu}^{2}})\bigg)\\
%& + 32C_2\exp\bigg(- \dfrac{3\rho_v}{2} \bigg(\dfrac{2\sigma_{i}^{2}+\Delta_{i}+2}{6\sigma_{i}^{2}+\Delta_{i}}\bigg) \log(\psi T\frac{\Delta_{i}^{4}}{16\rho_{v}^{2}}) \bigg)\\
%&\text{where $C_1=\frac{K\rho_{\mu}\log (\psi T \frac{\Delta_{i}^{4}}{16\rho_{\mu}^{2}})}{\Delta_{i}^{2}}$ and $C_2= \frac{K\rho_v\log (\psi T \frac{\Delta_{i}^{4}}{16\rho_{v}^{2}})}{\Delta_{i}^{2}}$}\\
%%%%%%%%%%%%%%%%%%%%%%%%%%%%%%%%%%%%%%%%
%&\leq 16 C_1\exp\bigg(-4\rho_{\mu}\log (\psi T\frac{\Delta_{i}^{4}}{16\rho^{2}})\bigg)
% + 32C_2\exp\bigg(- \dfrac{3\rho_v}{2} \log(\psi T\frac{\Delta_{i}^{4}}{16\rho_{v}^{2}}) \bigg)
%\end{align*}
%\end{small}
%
%%Summing over all rounds $m=0,1,..,M$,
%Now, putting the values of $\psi$, $\rho_{\mu}$, $\rho_v$ and taking $\Delta_{i}\geq\min_{i\in A}\Delta=\sqrt{\frac{K\log K}{T}}\geq \sqrt{\frac{e}{T}},\forall i\in A$( see \cite{auer2010ucb}), 
%
%\begin{small}
%\begin{align*}
%& \delta_{\max\lbrace m_{i}, g_{i}\rbrace}= \bigg\lbrace 16 C_1\exp\bigg(-4\rho_{\mu}\log (\psi T\frac{\Delta_{i}^{4}}{16\rho_{\mu}^{2}})\bigg)\\
%& + 32C_2\exp\bigg(- \frac{3\rho_v}{2} \log(\psi T\frac{\Delta_{i}^{4}}{16\rho_{v}^{2}}) \bigg) \bigg\rbrace\\
%%%%%%%%%%%%%%%%%%%%%
%&\leq \bigg\lbrace  \frac{2K\log ( T^2 \frac{4\Delta_{i}^{4}}{\log K})}{\Delta_{i}^{2}}\exp\bigg(-\frac{1}{2}\log ( T^2\frac{4\Delta_{i}^{4}}{\log K})\bigg)\\
%& + \frac{32K\log ( T^2 \frac{9\Delta_{i}^{4}}{\log K})}{3\Delta_{i}^{2}}\exp\bigg(- \frac{1}{2} \log( T^2 \frac{9\Delta_{i}^{4}}{\log K}) \bigg) \bigg\rbrace\\
%%%%%%%%%%%%%%%%%%%%%
%&\leq \bigg\lbrace  \frac{4K\log ( T \frac{2\Delta_{i}^{2}}{\sqrt{\log K}})}{\Delta_{i}^{2}}\exp\bigg(-\log ( T\frac{2\Delta_{i}^{2}}{\sqrt{\log K}})\bigg)\\
%& + \frac{64K\log ( T \frac{3\Delta_{i}^{2}}{\sqrt{\log K}})}{3\Delta_{i}^{2}}\exp\bigg(- \log( T \frac{3\Delta_{i}^{2}}{\sqrt{\log K}}) \bigg) \bigg\rbrace\\
%%%%%%%%%%%%%%%%%%%%%
%&\leq \bigg\lbrace  \frac{4KT\log ( \frac{2 K\log K}{\sqrt{\log K}})}{K\log K}\exp\bigg(-\log ( \frac{2K\log K}{\sqrt{\log K}})\bigg)\\
%& + \frac{64TK\log (\frac{3 K\log K}{\sqrt{\log K}})}{3 K\log K}\exp\bigg(- \log( \frac{3 K\log K}{\sqrt{\log K}}) \bigg) \bigg\rbrace\\
%%%%%%%%%%%%%%%%%%%%
%&\leq \bigg\lbrace  \frac{2T\log (2 K\sqrt{\log K})}{K (\log K)^{3/2}}
% + \frac{22T\log ( K\sqrt{\log K})}{ K(\log K)^{3/2}}\bigg) \bigg\rbrace\\
%\end{align*}
%\end{small}
%Now we know that till $m_i$-th round $2c_i > \frac{\Delta_i}{2}$  or till $g_i$ th round $2s_i > \frac{\Delta_i}{2}$. Hence, for the $i$-th arm we can bound the probability of error for any round $m$ by applying Chernoff-Hoeffding and Bernstein inequality,
%\begin{small}
%\begin{align*}
% \Pb\lbrace \xi_1\rbrace  + \Pb\lbrace \xi_2 \rbrace &\geq 1-(\Pb\lbrace |\hat{r}_i -r_i| > 2c_i \rbrace + \Pb\lbrace |\hat{r}_i -r_i| > 2s_i \rbrace)\\ 
%&\geq 1-\left(\Pb\lbrace |\hat{r}_i - r_i| > \frac{\Delta_i}{2} \rbrace + \Pb\lbrace |\hat{r}_i - r_i| > \frac{\Delta_i}{2} \rbrace\right) \\
%&\geq 1-\big(2\exp( -\frac{\Delta_{i}^{2}}{4}n_i ) + 2\exp(- \frac{\Delta_{i}^{2}}{8\sigma_{i}^{2}+ \frac{4}{3}\Delta_i}n_i)\big)\\
%&\geq 1-\bigg(2\exp( -\frac{\Delta_{i}^{2}}{4}\delta_{m_{i}} ) + 2\exp(- \frac{\Delta_{i}^{2}}{8\sigma_{i}^{2}+ \frac{4}{3}\Delta_i}\delta_{g_{i}})\bigg)
%\end{align*}
%\end{small}
%Now, we know that $\E[\Ls(T)]\le1- (\Pb\lbrace \xi_1\rbrace  + \Pb\lbrace \xi_2 \rbrace) $. Summing over all arms $K$ and over all rounds $m=0,1,2,..,M$ we get that,
%\begin{small}
%\begin{align*}
%&\E[\Ls(T)] \leq \sum_{i=1}^{K}\sum_{m=0}^{M}\bigg\lbrace 2\exp\bigg( -\frac{\Delta_{i}^{2}}{4}.\frac{2T\log (2 K\sqrt{\log K})}{K (\log K)^{3/2}}\bigg)\\
%& + 2\exp\bigg(- \frac{\Delta_{i}^{2}}{8\sigma_{i}^{2}+ \frac{4}{3}\Delta_i}.\frac{22T\log ( K\sqrt{\log K})}{ K(\log K)^{3/2}} \bigg)\bigg\rbrace\\
%%%%%%%%%%%%%%%%
%&\E[\Ls(T)] \leq K\left\lceil\log_2\frac{T}{e}\right\rceil\bigg\lbrace\exp\bigg( -\frac{1}{i\max_{i}\Delta_{i}^{-2}}.\frac{T\log (2 K\sqrt{\log K})}{2K (\log K)^{3/2}}\bigg)\\
%& + \exp\bigg(- \frac{3}{i\max_i(6\sigma_{i}^{2}+ \Delta_i)\Delta_{i}^{-2}}.\frac{5T\log ( K\sqrt{\log K})}{ K(\log K)^{3/2}} \bigg)\bigg\rbrace\\
%%%%%%%%%%%%%%%%
%&\E[\Ls(T)] \leq K\left(\log_2\frac{T}{e}+1\right)\bigg\lbrace\exp\bigg( -\frac{T\log (2 K\sqrt{\log K})}{2 H_2 K (\log K)^{3/2}}\bigg)\\
%& + \exp\bigg(- \frac{5T\log ( K\sqrt{\log K})}{H_{2}^{\sigma} K(\log K)^{3/2}} \bigg)\bigg\rbrace\\
%\end{align*}
%\end{small}
%%%%%%%%%%%%%%%%%%%%%%%%%%%%%%%%%%%%%%%%%%%%%%%%%%%%%%%%%%%%%%%%%%%%%%%%%%%%%%%%%%%%%%
%Not Required for probability of error for AugUCB
%%%%%%%%%%%%%%%%%%%%%%%%%%%%%%%%%%%%%%%%%%%%%%%%%%%%%%%%%%%%%%%%%%%%%%%%%%%%%%%%%%%%%%

%Hence, for the $i$-th arm we can bound the probability of it getting eliminated till $\max\lbrace m_i , g_i  \rbrace$-th round by,
%%\begin{small}
%\begin{align*}
% & \Pb\lbrace \text{$i\in \mathcal{A}^{'}$ getting eliminated on or before round $\max\lbrace m_i, g_i\rbrace$} \rbrace \\
%&\geq 1-(\Pb\lbrace |\hat{r}_i -r_i| > 2c_i \rbrace + \Pb\lbrace |\hat{r}_i -r_i| > 2s_i \rbrace)\\
%&\geq 1- \bigg( \left(2\exp\left(-\frac{T\rho_{\mu}\epsilon_{m_{i}}}{32 a^2}\log ( T\epsilon_{m_{i}})\right)\right)\\
%& + 4\exp\left(- \frac{3T\rho_v\epsilon_{g_{i}}}{256 a^2 } \left(\frac{\sigma_{i}^{2}+\sqrt{\rho_{v}\epsilon_{g_{i}}}+1}{3\sigma_{i}^{2}+\sqrt{\rho_v \epsilon_{g_{i}}}}\right) \log( T\epsilon_{g_{i}}) \right)\bigg)
%\end{align*}
%%\end{small}
%Now, in the $m_i$-th round or in the $g_i$-th round we know that $\frac{\Delta_i}{4}\leq\sqrt{\epsilon_{m_{i}}\rho_{\mu}}<\frac{\Delta_i}{2}$ or  $\frac{\Delta_i}{4}\leq\sqrt{\epsilon_{g_{i}}\rho_{v}}<\frac{\Delta_i}{2}$.
%%\begin{small}
%\begin{align*}
%&\Pb\lbrace \text{$i\in \mathcal{A}^{'}$ getting eliminated on or before round $\max\lbrace m_i, g_i\rbrace$} \rbrace\\
%%%%%%%%%%%%%%%%%%%%%%%%%%%%%%%%%%%%%%%%%%%%%%%%%%%%%%%
%& \geq 1- \bigg( 2\exp\left(-\frac{T\rho_{\mu}\frac{\Delta_{i}^{2}}{16\rho_{\mu}}}{32 a^2 }\log ( T\frac{\Delta_{i}^{2}}{16\rho_{\mu}})\right)\\
%& + 4\exp\left(- \frac{3T\rho_v\frac{\Delta_{i}^{2}}{16\rho_{v}}}{256 a^2} \left(\frac{\sigma_{i}^{2}+\frac{\Delta_{i}}{4}+1}{3\sigma_{i}^{2}+\frac{\Delta_{i}}{4}}\right) \log( T\frac{\Delta_{i}^{2}}{16\rho_{v}}) \right)\bigg)\\
%%%%%%%%%%%%%%%%%%%%%%%%%%%%%%%%%%%%%%%%%%%%%%%%%%%%%%%	
%&\geq 1-\bigg( 2\exp\left(-\frac{T\Delta_{i}^{2}}{64a}\log( \frac{T\Delta_{i}^{2}}{2})\right) \\
%& + 4\exp\left(- \frac{3T\Delta_{i}^{2}}{4096 a^2} \left(\frac{4\sigma_{i}^{2}+\Delta_{i}+4}{12\sigma_{i}^{2}+\Delta_{i}}\right) \log( \frac{3}{16} T\Delta_{i}^{2}) \right)\bigg),\\
%&\text{putting the values of $\rho_{\mu}$ and $\rho_{v}$.}
%\end{align*}
%%\end{small}
%Again, $\Pb\lbrace \xi_1 \cup \xi_2 \rbrace\geq 1- \sum_{i=1}^{K}\sum_{m=0}^{\max\lbrace m_{i} ,g_{i}\rbrace}\Pb\lbrace i\in \mathcal{A}^{'}$ getting eliminated on or before round $\max\lbrace m_i, g_i\rbrace \rbrace $.
%Also, $\E[\Ls(T)]\le 1- \Pb\lbrace \xi_1 \cup \xi_2 \rbrace $. We know from \cite{bubeck2011pure} and \cite{auer2010ucb} that the function $x\in [0,1]\mapsto x\exp(-Cx^2)$ is  decreasing on $[1/\sqrt{2C},1]$ for any $C>0$. So, taking $C=\lfloor \sqrt{e/T}\rfloor$ and putting $\min_{i\in \mathcal{A}}\Delta_i =\Delta =\sqrt{\frac{K\log K}{T}} > \sqrt{\frac{e}{T}},\forall i\in \mathcal{A}$ we get that,
%%and summing over all arms $K$ and over all rounds $m=0,1,2,..,\max\lbrace m_{i} ,g_{i}\rbrace$
%%\begin{small}
%\begin{align*}
%&\E[\Ls(T)] \leq \sum_{i=1}^{K}\sum_{m=0}^{\max\lbrace m_{i} ,g_{i}\rbrace}\bigg\lbrace \bigg( 2\exp\left(-\frac{T\Delta_{i}^{2} \log(\frac{T\Delta_{i}^{2}}{2})}{64 a^2 }\right) \\
%& + 4\exp\left(- \frac{3T\Delta_{i}^{2}}{4096 a^2 } \left(\frac{4\sigma_{i}^{2}+\Delta_{i}+4}{12\sigma_{i}^{2}+\Delta_{i}}\right) \log( \frac{3}{16} T\Delta_{i}^{2}) \right)\bigg\rbrace\\
%%%%%%%%%%%%%%%%%
%& \leq K\sum_{m=0}^{M}\bigg\lbrace 2\exp\bigg( -\frac{T}{\min_{i}i\Delta_{(i)}^{-2}}.\frac{\log (\frac{1}{2} K\log K)}{64 a^2 }\bigg)\\
%& + 4\exp\bigg(- \frac{12T\Delta_{i}^{2}}{(12\sigma_{i}+ 12\Delta_{i})}.\frac{\log (\frac{3}{16} K\log K)}{4096 a^2 } \bigg)\bigg\rbrace\\
%%%%%%%%%%%%%%%%
%&\leq K\left(\log_2\frac{T}{e}+1\right)\bigg\lbrace\exp\bigg( -\frac{T\log ( \frac{1}{2} K\log K)}{ 64 H_2 a^2}\bigg)\\
%& + 2\exp\bigg(- \frac{T\Delta_{i}^{2}\log ( \frac{3}{16} K\log K)}{4096 (\sigma_{i} + \sqrt{\sigma_{i}^{2} + (16/3)\Delta_{i}}) a^2} \bigg)\bigg\rbrace\\
%%%%%%%%%%%%%%%%
%&\leq K\left(\log_2\frac{T}{e}+1\right)\bigg\lbrace\exp\bigg( -\frac{T\log ( \frac{1}{2} K\log K)}{ 64 H_2 (\log(\frac{3}{16} K\log K))^{2}}\bigg)\\
%& + 2\exp\bigg(- \frac{T\log ( \frac{3}{16} K\log K)}{4096 \min_{i}i\tilde{\Delta}_{(i)}^{-2} (\log(\frac{3}{16} K\log K))^{2}} \bigg)\bigg\rbrace\\
%%%%%%%%%%%%%%%%
%&\leq K\left(\log_2\frac{T}{e}+1\right)\bigg\lbrace\exp\bigg( -\frac{T}{ 64 H_2 (\log(\frac{3}{16} K\log K))}\bigg)\\
%& + 2\exp\bigg(- \frac{T}{4096 H_{2}^{\sigma} (\log(\frac{3}{16} K\log K))} \bigg)\bigg\rbrace\\
%\end{align*}
%\end{small}


%	Next we specialize the result of Theorem \ref{Result:Theorem:1} in Corollary \ref{Result:Corollary:1}.
%
%\subsection{Corollary 2}
%
%
%\begin{corollary}
%\label{Result:Corollary:1}
%For $c_{0}=\sqrt{T}$, $\psi=\frac{T}{\log (K)}$, $\rho_{\mu}=\frac{1}{8}$ and $\rho_v=\frac{2}{3}$, the simple regret of AugUCB is given by,
%\begin{small}
%\begin{align*}
%& SR_{AugUCB} \leq \sum_{i=1}^{K} \Delta_{i}\bigg\lbrace\exp\bigg(-\log ( 2T\frac{\Delta_{i}^{2}}{\sqrt{\log K}})-\dfrac{T}{2 H_{2}}\\
%& + \log \big( \dfrac{4\gamma K\log ( 2T \frac{\Delta_{i}^{2}}{\sqrt{\log K}})}{T\Delta_{i}^{2}}\log_{2}\dfrac{T}{e} \big) \bigg)\\
%& +  \exp\bigg(- \bigg(\dfrac{2\sigma_{i}^{2}+\Delta_{i}+2}{6\sigma_{i}^{2}+\Delta_{i}}\bigg)\log( 3T\frac{\Delta_{i}^{2}}{8\sqrt{\log K}}) -\dfrac{3T}{32 H_{2}}\\
%& + \log\big ( \dfrac{64\gamma K\log ( 3T \frac{\Delta_{i}^{2}}{8\sqrt{\log K}})}{3T\Delta_{i}^{2}}\log_{2}\dfrac{T}{e} \big)  \bigg)\bigg\rbrace
%\end{align*}
%\end{small}
%\end{corollary}
%
%\begin{proof}
%Putting $c_{0}=\sqrt{T}$, $\psi=\frac{T}{\log (K)}$, $\rho_{\mu}=\frac{1}{8}$ and $\rho_v=\frac{2}{3}$ in the result obtained in Theorem \ref{Result:Theorem:1}, we get
%\begin{small}
%\begin{align*}
%& SR_{AugUCB} \leq \sum_{i=1}^{K} \Delta_{i}\bigg\lbrace \exp\bigg(-4\rho\log (\psi T\frac{\Delta_{i}^{4}}{16\rho^{2}})-\dfrac{c_{0}\sqrt{T}}{16\rho H_{2}}\\
%& + \log \big( 16\gamma C_1\log_{2}\dfrac{T}{e} \big) \bigg) + \exp\bigg(- \dfrac{3\rho_v}{2} \bigg(\dfrac{2\sigma_{i}^{2}+\Delta_{i}+2}{6\sigma_{i}^{2}+\Delta_{i}}\bigg)\log(\psi T\frac{\Delta_{i}^{4}}{16\rho_{v}^{2}})\\
%& -\dfrac{c_{0}\sqrt{T}}{16\rho_v H_{2}} + \log\big ( 32\gamma C_2\log_{2}\dfrac{T}{e} \big)  \bigg)\bigg\rbrace\\
%%%%%%%%%%%%%%%%%%
%&\leq \sum_{i=1}^{K} \Delta_{i}\bigg\lbrace\exp\bigg(-\dfrac{1}{2}\log ( T^{2}\frac{4\Delta_{i}^{4}}{\log K})-\dfrac{T}{2 H_{2}}\\
%& + \log \big( \dfrac{2\gamma K\log ( T^{2} \frac{4\Delta_{i}^{4}}{\log K})}{T\Delta_{i}^{2}}\log_{2}\dfrac{T}{e} \big) \bigg)\\
%& + \exp\bigg(-  \bigg(\dfrac{2\sigma_{i}^{2}+\Delta_{i}+2}{6\sigma_{i}^{2}+\Delta_{i}}\bigg)\log( T^{2}\frac{\Delta_{i}^{4}}{16.\frac{4}{9}\log K}) -\dfrac{c_{0}\sqrt{T}}{16.\frac{2}{3} H_{2}}\\
%& + \log\big ( \dfrac{32\gamma\rho_v K\log ( T^{2} \frac{\Delta_{i}^{4}}{16.\frac{2}{9}\log K})}{T\Delta_{i}^{2}}\log_{2}\dfrac{T}{e} \big)  \bigg)\bigg\rbrace\\
%%%%%%%%%%%%%%%%%%
%&\leq \sum_{i=1}^{K} \Delta_{i}\bigg\lbrace\exp\bigg(-\log ( 2T\frac{\Delta_{i}^{2}}{\sqrt{\log K}})-\dfrac{T}{2 H_{2}}\\
%& + \log \big( \dfrac{4\gamma K\log ( 2T \frac{\Delta_{i}^{2}}{\sqrt{\log K}})}{T\Delta_{i}^{2}}\log_{2}\dfrac{T}{e} \big) \bigg)\\
%& +  \exp\bigg(- \bigg(\dfrac{2\sigma_{i}^{2}+\Delta_{i}+2}{6\sigma_{i}^{2}+\Delta_{i}}\bigg)\log( 3T\frac{\Delta_{i}^{2}}{8\sqrt{\log K}}) -\dfrac{3T}{32 H_{2}}\\
%& + \log\big ( \dfrac{64\gamma K\log ( 3T \frac{\Delta_{i}^{2}}{8\sqrt{\log K}})}{3T\Delta_{i}^{2}}\log_{2}\dfrac{T}{e} \big)  \bigg)\bigg\rbrace
%\end{align*} 
%\end{small}
%\end{proof}


\section{Numerical Experiments}
\label{tbandit:expt}

In this section, we empirically compare the  performance of AugUCB against APT, UCBE, UCBEV, CSAR and the uniform-allocation (UA) algorithms. A brief note about these algorithms are as follows:
%\begin{itemize}

$\bullet$ APT: This algorithm is from \cite{locatelli2016optimal}; we set $\epsilon=0.05$, which is the margin-of-error within which APT suggests the set of good arms.

$\bullet$ AugUCB: This is the Augmented-UCB algorithm proposed in this paper; as in Theorem \ref{Result:Theorem:1} we set $\rho=\frac{1}{3}$.

$\bullet$ UCBE: This is a modification of the algorithm in \cite{audibert2009exploration} (as it was originally proposed for the best arm identification problem); here, we set $a=\frac{T-K}{H_1}$, and at each time-step an arm $i\in\argmin\left\lbrace |\hat{r}_{i} -\tau|-\sqrt{\frac{a}{n_{i}}} \right\rbrace$ is pulled.

$\bullet$ UCBEV: This is a modification of the algorithm in \cite{gabillon2011multi} (proposed for the TopM problem); its implementation is identical to UCBE, but with $a = \frac{T-2K}{H_{\sigma,1}}$. As mentioned earlier, note that UCBEV's implementation would not be possible in real scenarios, as it requires computing the problem complexity $H_{\sigma,1}$. However, for theoretical reasons we show the best performance achievable by UCBEV. In experiment 6 we perform further explorations of UCBEV with alternate settings of $a$.

$\bullet$ CSAR:  Modification of the successive-reject algorithm in \cite{chen2014combinatorial}; here, we reject the arm farthest from $\tau$ after each round. 

$\bullet$ UA: The naive strategy where at each time-step an arm is uniformly sampled from $\mathcal{A}$ (the set of all arms); however, UA is known to be optimal if all arms are equally difficult to classify. 
%\end{itemize}

%We also implement the uniform-allocation (labeled UA) strategy,  CSAR is modified for the TBP setting such that it behaves as a Successive Reject algorithm whereby it rejects the arm farthest from $\tau$ after each round.
%Similarly we modify the UCBE \cite{audibert2009exploration} and UCBEV \cite{gabillon2011multi} algorithms  (originally proposed for single best arm and TopM identification problems, respectively) to suit the TBP setting. 
%%to \cite{locatelli2016optimal} to suit the goal of finding arms above the threshold $\tau$. 
%Following  \cite{locatelli2016optimal} the exploration parameter $a$ in UCBE is set to $a=\frac{T-K}{H_1}$, while for UCBEV we set $a = \frac{T-2K}{H_{\sigma,1}}$. Then, at each time-step $t=1,2,..,T$ we pull the arm that minimizes
% $\lbrace |\hat{r}_{i} -\tau|-\sqrt{\frac{a}{n_{i}}} \rbrace$, where $a$ is set as mentioned above for UCBE and UCBEV respectively. Finally, for AugUCB we take $\rho=\frac{1}{3}$ as in  Theorem \ref{Result:Theorem:1}.

%Again, for UCBEV, following \cite{gabillon2011multi}, we modify it such that the exploration parameter $a = \frac{T-2K}{H_{\sigma,1}}$. Then for each timestep $t=1,2,..,T$ we pull the arm that minimizes $\lbrace |\hat{r}_{i} -\tau|-\sqrt{\frac{a}{n_{i}}} \rbrace$, where $n_{i}$ is the number of times the arm $i$ is pulled till $t-1$ timestep and $a$ is set as mentioned above for UCBE and UCBEV respectively. Also, APT is run with $\epsilon=0.05$, which denotes the precision with which the algorithm suggests the best set of arms and we modify CSAR as per \cite{locatelli2016optimal} such that it behaves as a Successive Reject algorithm whereby it rejects the arm farthest from $\tau$ after each phase. For AugUCB we take $\rho_{\mu}=\frac{1}{8}$ and $\rho_v=\frac{1}{3}$ as in Theorem \ref{Result:Theorem:1}. In all the testbeds AugUCB, APT, CSAR, Uniform Allocation, UCBE and UCBEV are run with the same settings as mentioned above.

\noindent
Motivated by the settings considered in \cite{locatelli2016optimal}, 
we design six different experimental scenarios that are obtained by varying the arm means and variances. 
%We conduct a set of six experiments with different reward means and variances. 
Across all experiments consists of $K=100$  arms (indexed $i=1,2,\cdots,100$) of which ${S}_\tau=\{6,7,\cdots,10\}$, where we have fixed $\tau=0.5$.
%in all the experiments, the threshold $\tau$ is set to $0.5$. %for all experiments. 
%Also, the number of arms in each experiment is $K=100$ , of which $\lbrace 6,7,8,9,10 \rbrace$ arms have their reward means above $\tau$. 
In all the experiments, each algorithm is run independently for $10000$ time-steps. At every time-step, the output set,  $\hat{S}_\tau$, suggested by each algorithm is recorded; the output is counted as an error if $\hat{S}_\tau\ne S_\tau$. In Figure~1, for each experiment, we have reported the percentage of error incurred by the different algorithms as a function of time; Error percentage is obtained by repeating each experiment independently  for $500$ iterations, and then respectively computing the fraction of errors. The details of the considered experiments are as follows.

%that is obtained by computing the fraction of errors when   The experiment is repeated for $500$ independent iterations, and the average error percentage is plotted against the $10000$ time-steps. 

 %The output is considered erroneous if the correct set of arms is not $i=\lbrace 6,7,8,9,10 \rbrace$ (true for all the experiments). The error percentage over $500$ runs is plotted against $10000$ timesteps. 
 
 %For the uniform allocation algorithm, for each $t=1,2,..,T$ the arms are sampled uniformly. 
 
 
%Also we run AugUCBM with arm elimination just by mean estimation and AugUCBV with arm elimination just by variance estimation. For AugUCBM, at every timestep we pull arm that minimizes $i\in\argmin_{j\in B_{m}}\bigg\lbrace |\hat{r}_{j} - \tau | - 2c_{j}\bigg\rbrace$ while for AugUCBV we pull arm that minimizes $i\in\argmin_{j\in B_{m}}\bigg\lbrace |\hat{r}_{j} - \tau | - 2s_{j}\bigg\rbrace$.

%	The first experiment is conducted on a testbed of $100$ arms involving 
	
\textbf{Experiment-1:} The reward distributions are Gaussian with  means  $r_{1:4}=0.2+(0:3)\cdot0.05$, $r_{5}=0.45$, $r_{6}=0.55$, $r_{7:10}=0.65+(0:3)\cdot0.05$ and $r_{11:100}=0.4$. Thus, the means of the first $10$ arms follow an arithmetic progression. The remaining arms have identical means; this setting is chosen because now a significant budget is required in exploring these arms, thus increasing the problem complexity.

 The corresponding variances are $\sigma_{1:5}^{2}=0.5$ and $\sigma_{6:10}^{2}=0.6$, while $\sigma_{11:100}^{2}$ is chosen independently and uniform in the  interval $[0.38,0.42]$;
% Then $\sigma_{11:100}^{2}$ is chosen uniform randomly between $0.38-0.42$.
note that, the variances of the arms in $S_\tau$ are higher than those of the other arms.
% The means in the testbed are chosen in such a way that any algorithm has to spend a significant amount of budget to explore all the arms and variance is chosen in such a way that the arms above $\tau$ have high variance whereas arms below $\tau$ have lower variance. 
 The corresponding  results are shown in Figure \ref{Fig:budgetExpt1},
 from where we see that UCBEV, which has access to the problem complexity while being variance-aware, outperforms all other algorithm (including UCBE which also has access to the problem complexity but does not take into account the variances of the arms).  Interestingly, the performance of our AugUCB (without requiring any complexity input) is comparable with UCBEV, while it 
   % with the said parameters 
 outperforms UCBE, APT and the other non variance-aware algorithms that we have considered. 	
%AugUCBM with just mean estimation performs worse than AugUCB or AugUCBV, which have a matching performance in this setup.
	
\begin{figure}[th!]
    \centering
    \begin{tabular}{cc}
    \subfigure[0.32\textwidth][Expt-$1$: Arithmetic Progression (Gaussian)]
    {
    		\pgfplotsset{
		tick label style={font=\Large},
		label style={font=\Large},
		legend style={font=\Large},
		}
        \begin{tikzpicture}[scale=0.6]
      	\begin{axis}[
		xlabel={Time-step},
		ylabel={Error Percentage},
		grid=major,
        %clip mode=individual,grid,grid style={gray!30},
        clip=true,
        %clip mode=individual,grid,grid style={gray!30},
  		legend style={at={(0.5,1.3)},anchor=north, legend columns=3} ]
      	% UCB
		\addplot table{tbandit/results/budgetTestAP/APT12_comp_subsampled.txt};
		\addplot table{tbandit/results/budgetTestAP/AugUCBV1_comp_subsampled.txt};
		%\addplot table{results/budgetTestAP/AugUCBV_1_13_comp_subsampled.txt};
		\addplot table{tbandit/results/budgetTestAP/UCBEM1_comp_subsampled.txt};
		\addplot table{tbandit/results/budgetTestAP/UCBEMV1_comp_subsampled.txt};
		\addplot table{tbandit/results/budgetTestAP/SR1_comp_subsampled.txt};
		\addplot table{tbandit/results/budgetTestAP/UA1_comp_subsampled.txt};
		%\addplot table{results/budgetTestAP/AugUCBM12_comp_subsampled.txt};
		%\addplot table{results/budgetTestAP/AugUCBV1_comp_subsampled.txt};
      	%\legend{APT,AugUCB,UCBE,UCBEV,CSAR,Unif Alloc,AugUCBM,AugUCBV}
      	\legend{APT,AugUCB,UCBE,UCBEV,CSAR,UA}
      	\end{axis}
      	\end{tikzpicture}
  		\label{Fig:budgetExpt1}
    }
    &
    \subfigure[0.32\textwidth][Expt-$2$: Geometric Progression (Gaussian)]
    {
    	\pgfplotsset{
		tick label style={font=\Large},
		label style={font=\Large},
		legend style={font=\Large},
		}
        \begin{tikzpicture}[scale=0.6]
        \begin{axis}[
		xlabel={Time-step},
		ylabel={Error Percentage},
        %clip mode=individual,grid,grid style={gray!30},
		grid=major,
		clip=true,
  		legend style={at={(0.5,1.3)},anchor=north, legend columns=3} ]
        % UCB
		\addplot table{tbandit/results/budgetTestGP/APT12_comp_subsampled.txt};
		\addplot table{tbandit/results/budgetTestGP/AugUCBV1_comp_subsampled.txt};
		%\addplot table{results/budgetTestGP/AugUCBV_1_13_comp_subsampled.txt};
		\addplot table{tbandit/results/budgetTestGP/UCBEM1_comp_subsampled.txt};
		\addplot table{tbandit/results/budgetTestGP/UCBEMV1_comp_subsampled.txt};
		\addplot table{tbandit/results/budgetTestGP/SR1_comp_subsampled.txt};
		\addplot table{tbandit/results/budgetTestGP/UA1_comp_subsampled.txt};
		%\addplot table{results/budgetTestGP/AugUCBM12_comp_subsampled.txt};
		%\addplot table{results/budgetTestGP/AugUCBV1_comp_subsampled.txt};
        %\legend{APT,AugUCB,UCBE,UCBEV,CSAR,Unif Alloc,AugUCBM,AugUCBV}
        \legend{APT,AugUCB,UCBE,UCBEV,CSAR,UA}
      	\end{axis}
      	\label{Fig:budgetExpt2}
        \end{tikzpicture}
    }
    \end{tabular}

	\begin{tabular}{cc}
	\centering
    \subfigure[0.32\textwidth][Expt-$3$: Three Group Setting (Gaussian)]
    {
    		\pgfplotsset{
		tick label style={font=\Large},
		label style={font=\Large},
		legend style={font=\Large},
		}
        \begin{tikzpicture}[scale=0.6]
        \begin{axis}[
		xlabel={Time-step},
		ylabel={Error Percentage},
        %clip mode=individual,grid,grid style={gray!30},
       	grid=major,
       	clip=true,
  		legend style={at={(0.5,1.3)},anchor=north, legend columns=3} ]
      	% UCB
		\addplot table{tbandit/results/budgetTestGR1/APT1_comp_subsampled.txt};
		\addplot table{tbandit/results/budgetTestGR1/AugUCB1_comp_subsampled.txt};
		\addplot table{tbandit/results/budgetTestGR1/UCBEM1_comp_subsampled.txt};
		\addplot table{tbandit/results/budgetTestGR1/UCBEMV1_comp_subsampled.txt};
		\addplot table{tbandit/results/budgetTestGR1/SR1_comp_subsampled.txt};
		\addplot table{tbandit/results/budgetTestGR1/UA1_comp_subsampled.txt};
        \legend{APT,AugUCB,UCBE,UCBEV,CSAR,UA}
      	\end{axis}
      	\end{tikzpicture}
   		\label{Fig:budgetExpt3} 
    }
    &
    \subfigure[0.32\textwidth][Expt-$4$: Two Group Setting (Gaussian) ]
    {
    	\pgfplotsset{
		tick label style={font=\Large},
		label style={font=\Large},
		legend style={font=\Large},
		}
        \begin{tikzpicture}[scale=0.6]
        \begin{axis}[
		xlabel={Time-step},
		ylabel={Error Percentage},
        %clip mode=individual,grid,grid style={gray!30},
		grid=major,
		clip=true,
  		legend style={at={(0.5,1.3)},anchor=north, legend columns=3} ]
        % UCB
		\addplot table{tbandit/results/budgetTestGR2/APT1_comp_subsampled.txt};
		\addplot table{tbandit/results/budgetTestGR2/AugUCBV1_comp_subsampled.txt};
		%\addplot table{results/budgetTestGP/AugUCBV_1_13_comp_subsampled.txt};
		\addplot table{tbandit/results/budgetTestGR2/UCBEM1_comp_subsampled.txt};
		\addplot table{tbandit/results/budgetTestGR2/UCBEMV1_comp_subsampled.txt};
		\addplot table{tbandit/results/budgetTestGR2/SR1_comp_subsampled.txt};
		\addplot table{tbandit/results/budgetTestGR2/UA1_comp_subsampled.txt};
		%\addplot table{results/budgetTestGP/AugUCBM12_comp_subsampled.txt};
		%\addplot table{results/budgetTestGP/AugUCBV1_comp_subsampled.txt};
        %\legend{APT,AugUCB,UCBE,UCBEV,CSAR,Unif Alloc,AugUCBM,AugUCBV}
        \legend{APT,AUgUCB,UCBE,UCBEV,CSAR,UA}
        %\legend{APT,AugUCB,UCBE,UCBEV,CSAR,Unif Alloc}
      	\end{axis}
      	\label{Fig:budgetExpt4}
        \end{tikzpicture}
    }
    \end{tabular}

	\begin{tabular}{cc}
    \subfigure[0.32\textwidth][Expt-$5$: Two Group Setting (Advance) ]
    {
    	\pgfplotsset{
		tick label style={font=\Large},
		label style={font=\Large},
		legend style={font=\Large},
		}
        \begin{tikzpicture}[scale=0.6]
        \begin{axis}[
		xlabel={Time-step},
		ylabel={Error Percentage},
        %clip mode=individual,grid,grid style={gray!30},
		grid=major,
		clip=true,
  		legend style={at={(0.5,1.3)},anchor=north, legend columns=3} ]
        % UCB
		\addplot table{tbandit/results/budgetTestGR4/APT1_comp_subsampled.txt};
		\addplot table{tbandit/results/budgetTestGR4/AugUCB1_comp_subsampled.txt};
		\addplot table{tbandit/results/budgetTestGR4/UCBEM1_comp_subsampled.txt};
		\addplot table{tbandit/results/budgetTestGR4/UCBEMV1_comp_subsampled.txt};
		\addplot table{tbandit/results/budgetTestGR4/SR1_comp_subsampled.txt};
		\addplot table{tbandit/results/budgetTestGR4/UA1_comp_subsampled.txt};
        \legend{APT,AUgUCB,UCBE,UCBEV,CSAR,UA}
      	\end{axis}
      	\label{Fig:budgetExpt5}
        \end{tikzpicture}
    }
    &
    \subfigure[0.32\textwidth][Expt-$6$: Two Group Setting (Advance) ]
    {
    	\pgfplotsset{
		tick label style={font=\Large},
		label style={font=\Large},
		legend style={font=\Large},
		}
        \begin{tikzpicture}[scale=0.6]
        \begin{axis}[
		xlabel={Time-step},
		ylabel={Error Percentage},
        %clip mode=individual,grid,grid style={gray!30},
		grid=major,
		clip=true,
  		legend style={at={(0.5,1.3)},anchor=north, legend columns=2} ]
        % UCB
		\addplot table{tbandit/results/budgetTestGR3/testUCBEMV1_0.25_comp_subsampled.txt};
		\addplot table{tbandit/results/budgetTestGR4/AugUCB1_comp_subsampled.txt};
		%\addplot table{results/budgetTestGP/AugUCBV_1_13_comp_subsampled.txt};
		%\addplot table{results/budgetTestGR3/testUCBEMV1_0.25_comp_subsampled.txt};
		\addplot table{tbandit/results/budgetTestGR3/testUCBEMV1_256_comp_subsampled.txt};
		\addplot table{tbandit/results/budgetTestGR4/UCBEMV1_comp_subsampled.txt};
		%\addplot table{results/budgetTestGR3/testUCBEMV1_64_comp_subsampled.txt};
		%\addplot table{results/budgetTestGP/AugUCBM12_comp_subsampled.txt};
		%\addplot table{results/budgetTestGP/AugUCBV1_comp_subsampled.txt};
        %\legend{APT,AugUCB,UCBE,UCBEV,CSAR,Unif Alloc,AugUCBM,AugUCBV}
        \legend{UCBEV($0.25$), AugUCB, UCBEV($256$), UCBEV($1$)}
        %\legend{UCBEV($0.06$),AUgUCB,UCBEV($0.25$),UCBEV($1$),UCBEV($64$),UCBEV($256$)}
        %\legend{APT,AugUCB,UCBE,UCBEV,CSAR,Unif Alloc}
      	\end{axis}
      	\label{Fig:budgetExpt6}
        \end{tikzpicture}
    }
    \end{tabular}
    \caption{Performances of the various TBP algorithms in terms of error percentage vs. time-step, for  six different experimental scenarios.}
    \label{fig:budgetExpt}
    \vspace{-6mm}
\end{figure}

	
\textbf{Experiment-2:} We again consider  Gaussian reward distributions. However, here the means of the first $10$ arms constitute a geometric progression. Formally, the reward means are $r_{1:4}=0.4-(0.2)^{1:4}$, $r_{5}=0.45$, $r_{6}=0.55$, $r_{7:10}=0.6+(0.2)^{5-(1:4)}$ and $r_{11:100}=0.4$; the arm variances are as in experiment-$1$. The corresponding results are shown in Figure \ref{Fig:budgetExpt2}.  We again observe AugUCB outperforming the other algorithms, except UCBEV. 
	

\textbf{Experiment-3:} Here, the first
$10$ arms are partitioned into three groups, with all arms in a group being assigned the same mean; the reward distributions are again Gaussian. Specifically, the reward means are $r_{1:3}=0.1$, $r_{4:7}=\lbrace 0.35, 0.45, 0.55, 0.65\rbrace$ and $r_{8:10}=0.9$; as before,  $r_{11:100}=0.4$ and all the variances are as in Experiment-$1$. The results for this scenario are presented in Figure \ref{Fig:budgetExpt3}. The observations are inline with those made in the previous experiments. 


	
\textbf{Experiment-4:} The setting is similar to that considered in Experiment-3, but with the first $10$ arms partitioned into two groups; the respective means are $r_{1:5}=0.45$, $r_{6:10}=0.55$. The corresponding results are shown in Figure \ref{Fig:budgetExpt4}, from where the good performance of AugUCB is again validated.


\textbf{Experiment-5:} This is again the two group setting involving Gaussian reward distributions. The reward means are as in Experiment-4, while the variances are  $\sigma_{1:5}^{2}=0.3$ and $\sigma_{6:10}^{2}=0.8$;  $\sigma_{11:100}^{2}$ are independently and uniformly chosen in the interval $[0.2,0.3]$.  The corresponding results are shown in Figure \ref{Fig:budgetExpt5}.
 We refer to this setup as \emph{Advanced} because here the chosen variance values are such that only  variance-aware algorithms will perform well.Hence, we see that UCBEV performs very well in comparison with the other algorithms. However,  it is interesting to note that the performance of  AugUCB catches-up with UCBEV as the time-step increases, while significantly outperforming the other non-variance aware algorithms.


\textbf{Experiment-6:} We use the same setting as in Experiment-5, but conduct more exploration of UCBEV with different values of the exploration parameter $a$. The corresponding results are shown in Figure \ref{Fig:budgetExpt6}. As studied in \cite{locatelli2016optimal}, we implement UCBEV with $ a_{i} = 4^{i} \frac{T-2K}{H_{\sigma,1}}$ for $i = -1,0,4$. Here, $a_{0}$ corresponds to UCBEV($1$) (in Figure \ref{Fig:budgetExpt6}) which is UCBEV run with the optimal choice of $H_{\sigma ,1}$. For other choices of $a_i$ we see that UCBEV($a_i$) is significantly outperformed by AugUCB. 
	
Finally, note that in all the above experiments, the CSAR algorithm, although performs well initially, quickly exhausts its budget and saturates at a higher error percentage. This is because it pulls all arms equally in each round, with the round lengths being non-adaptive.








\section{Conclusion}
\label{tbandit:conclusion}
We proposed the AugUCB algorithm for a fixed-budget, pure-exploration TBP. Our algorithm employs both mean and variance estimates for arm elimination. This, to our knowledge is the first variance-based algorithm for the specific TBP that we have considered. We first prove an upper bound on the expected loss incurred by AugUCB. We then conduct simulation experiments to validate the performance of AugUCB. In comparison with APT, CSAR and other non variance-based algorithms, we find that the performance of AugUCB is significantly better. Further, the performance of AugUCB is comparable with UCBEV (which is also variance-based), although the latter exhibits a slightly better performance.  However, UCBEV is not implementable in practice as it requires computing problem complexity, $H_{\sigma,1}$, while AugUCB (requiring no such inputs) can be easily deployed in real-life scenarios. It would be an interesting future work to design an anytime version of the AugUCB algorithm. 

%Although UCBEV provides a better guarantee, it is important to emphasize that UCBEV has access to the problem complexity, and is hence not realistic in practice. This is in contrast to AugUCB whose implementation does not require any such complexity inputs. 
%Finally, through extensive simulation experiments we have validated the performance of AugUCB.

%From a theoretical viewpoint we conclude the expected loss AugUCB is more than UCBEV (which has access to problem complexity). From the numerical experiments on settings with large number of arms with different mean and variance, we observed that AugUCB outperforms all the non-variance aware algorithms.
%This is also the first paper to apply elimination by variance estimation in the TBP problem by modifying UCB-Improved and CCB algorithms. 
% It would be interesting future research to come up with an anytime version of AugUCB algorithm. This is also the first paper to apply elimination by variance estimation in the TBP problem by modifying UCB-Improved and CCB algorithms. 


\section{Summary}
\label{tbandit:Summary}
In this chapter we looked at the Augmented-UCB (AugUCB) algorithm for a fixed-budget version of the thresholding bandit problem (TBP), where the objective is to identify a set of arms whose expected mean is above a threshold. A key feature of AugUCB is that it uses both mean and variance estimates to eliminate arms that have been sufficiently explored; to the best of our knowledge this is the first algorithm to employ such an approach for the considered TBP.  Theoretically, we obtain an upper bound on the loss (probability of mis-classification) incurred by AugUCB. Although UCBEV in literature provides a better guarantee, it is important to emphasize that UCBEV has access to problem complexity (whose computation requires arms' mean and variances), and hence is not realistic in practice; this is in contrast to AugUCB whose implementation does not require any such complexity inputs. We conduct extensive simulation experiments to validate the performance of AugUCB. Through our simulation work, we establish that AugUCB, owing to its utilization of variance estimates, performs significantly better than the state-of-the-art APT, CSAR and other non variance-based algorithms.


%%%%%%%%%%%%%%%%%%%%%%%%%%%%%%%%%%%%%%%%%%%%%%%%%%%%%%%%%%%%


%%%%%%%%%%%%%%%%%%%%%%%%%%%%%%%%%%%%%%%%%%%%%%%%%%%%%%%%%%%%
\chapter{Efficient UCB Variance}
\label{chap:EUCBV}
\section{Introduction}
\label{sec:intro}
In this paper, we deal with the stochastic multi-armed bandit (MAB) setting. In its classical form, stochastic MABs represent a sequential learning problem where a learner is exposed to a finite set of actions (or arms) and needs to choose one of the actions at each timestep. After choosing (or pulling) an arm the learner  receives a reward, which is conceptualized as an independent random draw from stationary distribution associated with the selected arm. 
%Each of these rewards is random and drawn independently from the distribution associated with each arm. 
The mean of the reward distribution associated with an arm $i$ is denoted by $r_i$ whereas the mean of the reward distribution of the optimal arm $*$ is denoted by $r^*$ such that $r_i < r^*, \forall i\in \A$, where $\A$ is the set of arms such that $|\A|=K$. With this formulation the learner faces the task of balancing exploitation and exploration. In other words, should the learner pull the arm which currently has the best known estimates or explore arms more thoroughly to ensure that a correct decision is being made. The objective in the stochastic bandit problem is to minimize the cumulative regret, which is defined as follows:
\begin{align*}
R_{T}=r^{*}T - \sum_{i\in \A} r_{i}z_{i}(T),
\end{align*}
where $T$ is the number of timesteps, and  $z_{i}(T)$ is the number of times the algorithm has chosen arm $i$ up to timestep $T$.
The expected regret of an algorithm after $T$ timesteps can be written as,
\begin{align*}
\E[R_{T}]= \sum_{i=1}^{K} \E[z_i (T)] \Delta_i,
\end{align*}
where $\Delta_{i}=r^{*}-r_{i}$ is the gap between the means of the optimal arm and the $i$-th arm.

% One of the fundamental assumptions in stochastic MAB is that the distribution associated with each arm does not change over the entire time horizon $T$.

	In recent years the MAB setting has garnered extensive popularity because of its simple learning  model and its practical applications in a wide-range of industries, including, but not limited to, mobile channel allocations, online advertising and computer simulation games. 
	
	%industry defined problems

\subsection{Related Work}
\label{sec:related}
%There has been a significant amount of research in the area of stochastic MABs. One of the earliest work can be traced to \cite{thompson1933likelihood}, which deals with  the problem of choosing between two treatments to administer on patients who come in sequentially. Other seminal works include that of  \cite{robbins1952some} and then that of \cite{lai1985asymptotically} which established an asymptotic lower bound for the cumulative regret. It showed that for any consistent allocation strategy, we can have
%$\liminf_{T \to \infty}\frac{\E[R_{T}]}{\log T}\geq\sum_{\{i:r_{i}<r^{*}\}}\frac{(r^{*}-r_{i})}{D(Q_{i}||Q^{*})},$
%where $D(Q_{i}||Q^{*})$ is the Kullback-Leibler divergence between the reward densities $Q_{i}$ and $Q^{*}$, corresponding to arms with mean $r_{i}$ and $r^{*}$, respectively.

	Bandit problems have been extensively studied in several earlier works such as \citet{thompson1933likelihood}, \citet{robbins1952some} and \citet{lai1985asymptotically}. Lai and Robbins in  \citet{lai1985asymptotically} established an asymptotic lower bound for the cumulative regret. Over the years stochastic MABs have seen several algorithms with strong regret guarantees. For further reference an interested reader can look into \citet{bubeck2012regret}. The upper confidence bound algorithms balance the exploration-exploitation dilemma by linking the uncertainty in estimate of an arm with the number of times an arm is pulled, and therefore ensuring sufficient exploration. One of the earliest among these algorithms is UCB1 \citep{auer2002finite}, which has a gap-dependent regret upper bound of  $O\left(\frac{K\log T}{\Delta}\right)$, where $\Delta = \min_{i:\Delta_i>0} \Delta_i$. This result is asymptotically order-optimal for the class of distributions considered. But, the worst case gap-independent regret bound of UCB1 is found to be  $O \left(\sqrt{KT\log T}\right)$. In the later work of \citet{audibert2009minimax}, the authors propose the MOSS algorithm and showed that the worst case gap-independent regret bound of MOSS is $O\left( \sqrt{KT} \right)$ which improves upon UCB1 by a factor of order $\sqrt{\log T}$. However, the gap-dependent regret of MOSS is $O\left( \frac{K^{2}\log\left(T\Delta^{2}/K\right)}{\Delta}\right)$ and in certain regimes, this can be worse than even UCB1 (see \citet{audibert2009minimax,lattimore2015optimally}).
	
	 The UCB-Improved algorithm, proposed in \citet{auer2010ucb}, is a round-based\footnote{An algorithm is \textit{round-based} if it pulls all the arms equal number of times in each round and then eliminates one or more arms that it deems  to be sub-optimal.} variant of UCB1, that 
incurs a gap-dependent regret bound of $O\left(\frac{K\log (T\Delta^{2})}{\Delta}\right)$, which is better than that of UCB1. On the other hand, the worst case gap-independent regret bound of UCB-Improved is $O\left(\sqrt{KT\log K}\right)$. Recently in \citet{lattimore2015optimally}, the authors showed that  the algorithm OCUCB achieves order-optimal gap-dependent regret bound of $O\left(\sum_{i=2}^{K}\frac{\log\left(T/H_i\right)}{\Delta_i}\right)$ where $H_i=\sum_{j=1}^{K}\min\left\lbrace \frac{1}{\Delta_i^2},\frac{1}{\Delta_j^2}\right\rbrace$, and a gap-independent regret bound of $O\left( \sqrt{KT}\right)$. This is the best known gap-dependent and gap-independent regret bounds in the stochastic MAB framework. However, unlike our proposed EUCBV algorithm, OCUCB does not take into account the variance of the arms; as a result, empirically  we find  that our algorithm outperforms OCUCB in all the environments considered. 

	In contrast to the above work, the UCBV \citep{audibert2009exploration} algorithm utilizes variance estimates to compute the confidence intervals for each arm. UCBV has a gap-dependent regret bound of $O\left(\frac{K\sigma_{\max}^{2}\log T}{\Delta}\right)$, where $\sigma_{\max}^{2}$ denotes the maximum variance among all the arms $i\in \A$. Its gap-independent regret bound can be inferred to be same as that of UCB1 i.e $O \left(\sqrt{KT\log T}\right)$. Empirically, \citet{audibert2009exploration} showed that UCBV outperforms UCB1 in several scenarios. 
	
	Another notable design principle which has recently gained a lot of popularity is the Thompson Sampling (TS) algorithm (\citep{thompson1933likelihood}, \citep{agrawal2011analysis})  and  Bayes-UCB (BU) algorithm \citep{kaufmann2012bayesian}. % which employs the Bayesian approach in solving the MAB problem.
The TS algorithm maintains a posterior reward distribution for each arm; at each round, the algorithm samples values from these distribution and the arm corresponding to the highest sample value is chosen. Although TS is found to perform extremely well when the reward distributions are Bernoulli, it is established that with Gaussian priors the worst case regret can be as bad as $\Omega \left( \sqrt{KT\log T}\right)$ \citep{lattimore2015optimally}. The BU algorithm is an extension of the TS algorithm that takes quartile deviations into consideration while choosing arms.
	
	The final design principle we state is the information theoretic approach of DMED  \citep{honda2010asymptotically} and KLUCB \citep{garivier2011kl} algorithms. The algorithm KLUCB uses Kullbeck-Leibler divergence to compute the upper confidence bound for the arms. KLUCB is stable for a short horizon and is known to reach the \citet{lai1985asymptotically} lower bound in the special case of Bernoulli distribution. However, \citet{garivier2011kl} showed that KLUCB, MOSS and UCB1 algorithms are  empirically outperformed by UCBV in the exponential distribution as they do not take the variance of the arms into consideration. 


\subsection{Our Contributions}
\label{sec:contri}
In this paper we propose the Efficient-UCB-Variance (henceforth referred to as EUCBV) algorithm for the stochastic MAB setting. EUCBV combines the approach of UCB-Improved, CCB \citep{liu2016modification} and UCBV algorithms. EUCBV, by virtue of taking into account the empirical variance of the arms, exploration parameters  and non-uniform arm selection (as opposed to UCB-Improved), performs significantly better than the existing algorithms in the stochastic MAB setting. EUCBV outperforms UCBV \citep{audibert2009exploration} which also takes into account empirical variance but is less powerful than EUCBV because of the usage of exploration regulatory factor by EUCBV. Also, we carefully design the confidence interval term with the variance estimates along with the pulls allocated to each arm to balance the risk of eliminating the optimal arm against excessive optimism. Theoretically we refine the analysis of \citet{auer2010ucb} and prove that for $T\geq K^{2.4}$ our algorithm is order optimal and achieves a worst case gap-independent regret bound of $O\left( \sqrt{KT} \right)$ which is same as that of MOSS and OCUCB but better than that of UCBV, UCB1 and UCB-Improved. Also, the gap-dependent regret bound of EUCBV is better than UCB1, UCB-Improved and MOSS but is poorer than OCUCB. However, EUCBV's gap-dependent bound matches OCUCB in the worst case scenario when all the gaps are equal. Through our theoretical analysis we establish the exact values of the exploration parameters for the best performance of EUCBV. Our proof technique is highly generic and can be easily extended to other MAB settings. An illustrative table containing the bounds is provided in Table \ref{tab:comp-bds}. 


\begin{table}[t]
\caption{Regret upper bound of different algorithms}
\label{tab:comp-bds}
\begin{center}
\begin{tabular}{|p{5em}|p{12em}|p{7em}|}
\hline
Algorithm  &   \hspace*{1mm}Gap-Dependent & Gap-Independent \\
\hline
\hline
EUCBV		& $O\left( \dfrac{K\sigma_{\max}^{2}\log (\frac{T\Delta^2}{K})}{\Delta}\right)$ & $O\left(\sqrt{KT}\right)$\\
\hline
\hline
UCB1        & $O\left( \dfrac{K\log T}{\Delta} \right)$ & $O\left(\sqrt{KT\log T}\right)$ \\%\midrule
\hline
\hline
UCBV        & $O\left( \dfrac{K\sigma_{\max}^{2}\log T}{\Delta} \right)$ & $O\left(\sqrt{KT\log T}\right)$ \\
\hline
\hline
UCB-Imp 		& $O\left( \dfrac{K\log (T\Delta^2)}{\Delta} \right)$ & $O\left(\sqrt{KT\log K}\right)$ \\%\midrule
\hline
\hline
MOSS	     	& $O\left( \dfrac{K^2\log (T\Delta^2 /K)}{\Delta}\right)$ & $O\left(\sqrt{KT}\right)$\\%\midrule
\hline
\hline
OCUCB     	& $O\left( \dfrac{K\log (T/ H_{i})}{\Delta}\right)$ & $O\left(\sqrt{KT}\right)$\\\midrule
\end{tabular}
\end{center}
%\vspace*{-2em}
\end{table}


Empirically, we show that EUCBV, owing to its estimating the variance of the arms, exploration parameters and non-uniform arm pull, performs significantly better than MOSS, OCUCB, UCB-Improved, UCB1, UCBV, TS, BU, DMED, KLUCB and Median Elimination algorithms. Note that except UCBV, TS, KLUCB and BU (the last three with Gaussian priors) all the aforementioned algorithms do not take into account the empirical variance estimates of the arms. Also, for the optimal performance of TS, KLUCB and BU one has to have the prior knowledge of the type of distribution, but EUCBV requires no such prior knowledge. EUCBV is the first arm-elimination algorithm that takes into account the variance estimates of the arm for minimizing cumulative regret and thereby answers an open question raised by \citet{auer2010ucb}, where the authors conjectured that an UCB-Improved like arm-elimination algorithm can greatly benefit by taking into consideration the variance of the arms. Also, it is the first algorithm that follows the same proof technique of UCB-Improved and achieves a gap-independent regret bound of $O\left( \sqrt{KT} \right)$ thereby, closing the gap of UCB-Improved which achieved a gap-independent regret bound of $O\left( \sqrt{KT\log K} \right)$. 
	
	The rest of the paper is organized as follows. In section~\ref{sec:eucbv} we present the  EUCBV algorithm. Our main theoretical results are stated in section~\ref{sec:results}, while the proofs are established in   section \ref{sec:proofTheorem}. Section~\ref{sec:expt} contains results and discussions from our numerical experiments. We draw our conclusions in section \ref{sec:conc} and section \ref{sec:app} is Appendix (supplementary material).
	
	%discuss about future works. 
	
	%The section \ref{sec:app} containing further proofs is given as supplementary.
	
	
	

\section{Algorithm: Efficient UCB Variance}
\label{sec:eucbv}
%%%%%%%%%%%%%%%% alg-custom-block %%%%%%%%%%%%
%\algblock{ArmElim}{EndArmElim}
%\algnewcommand\algorithmicArmElim{\textbf{\em Arm Elimination}}
% \algnewcommand\algorithmicendArmElim{}
%\algrenewtext{ArmElim}[1]{\algorithmicArmElim\ #1}
%\algrenewtext{EndArmElim}{\algorithmicendArmElim}
%
%\algblock{ResParam}{EndResParam}
%\algnewcommand\algorithmicResParam{\textbf{\em Reset Parameters}}
% \algnewcommand\algorithmicendResParam{}
%\algrenewtext{ResParam}[1]{\algorithmicResParam\ #1}
%\algrenewtext{EndResParam}{\algorithmicendResParam}

\begin{algorithm}[!h]
\caption{EUCBV}
\label{alg:eucbv}
\begin{algorithmic}
\State {\bf Input:} Time horizon $T$, exploration parameters $\rho$ and $\psi$.
\State {\bf Initialization:} Set $m:=0$, $B_{0}:=\mathcal{A}$, $\epsilon_{0}:=1$, $M=\big \lfloor \frac{1}{2}\log_{2} \frac{T}{e}\big\rfloor$, $n_{0}=\big\lceil\frac{\log{(\psi T\epsilon_{0}^{2})}}{2\epsilon_{0}}\big\rceil$ and  $N_{0}=Kn_{0}$.
\State Pull each arm once
\For{$t=K+1,..,T$}	
\State Pull arm $i\in \argmax_{j\in B_{m}}\bigg\lbrace \hat{r}_{j} + \sqrt{\frac{\rho(\hat{v}_{j}+2)\log{(\psi T\epsilon_{m})}}{4 z_{j}}} \bigg\rbrace$, where $z_j$ is the number of times arm $j$ has been pulled.
%\State $t:=t+1$
\ArmElim
\State For each arm $i \in B_{m}$, remove arm $i$ from $B_{m}$ if,
\begin{align*}
%%%%%%%%%%%%%%%%%%%%%%%
%& \hat{r}_{i} + \sqrt{\frac{\rho\hat{v}_{i}\log{(\psi T\epsilon_{m})}}{4 z_{i}} + \frac{\rho\log{(\psi T\epsilon_{m})}}{4 z_{i}}} < \max_{{j}\in B_{m}}\bigg\lbrace\hat{r}_{j} -\sqrt{\frac{\rho\hat{v}_{j}\log{(\psi T\epsilon_{m})}}{4 z_{j}} + \frac{\rho\log{(\psi T\epsilon_{m})}}{4 z_{j}}} \bigg\rbrace
%%%%%%%%%%%%%%%%%%%%%%%
 \hat{r}_{i} + & \sqrt{\frac{\rho(\hat{v}_{i}+2)\log{(\psi T\epsilon_{m})}}{4 z_{i}}}  
  < \max_{{j}\in B_{m}}\bigg\lbrace\hat{r}_{j} -\sqrt{\frac{\rho(\hat{v}_{j}+2)\log{(\psi T\epsilon_{m})}}{4 z_{j}}} \bigg\rbrace
\end{align*}
\EndArmElim

\If{$t\geq N_{m}$ and $m\leq M$}
\ResParam
\State $\epsilon_{m+1}:=\frac{\epsilon_{m}}{2}$\vspace{0.5ex}
\State $B_{m+1}:=B_{m}$
\State $n_{m+1}:=\bigg\lceil\frac{\log{(\psi T\epsilon_{m+1}^{2})}}{2\epsilon_{m+1}}\bigg\rceil$
\State $N_{m+1}:=t+|B_{m+1}| n_{m+1}$
\State $m:=m+1$
\EndResParam
\EndIf
\State Stop if $|B_{m}|=1$ and pull ${i}\in B_{m}$ till $T$ is reached.
\EndFor
\end{algorithmic}
%\vspace*{-0.42em}
\end{algorithm}
%\vspace*{-0.42em}
\textbf{2.1 Notations:} We denote the set of arms by $\A$, with the individual arms labeled $i$, where  $i=1,\ldots,K$. We denote an arbitrary round of EUCBV by $m$. For simplicity, we assume that the optimal arm is unique and denote it by ${*}$. We denote the sample mean of the rewards for an arm $i$ at time instant $t$ by $\hat{r}_{i}(t)=\frac{1}{z_{i}(t)}\sum_{\ell=1}^{z_i(t)} X_{i,\ell}$, where $X_{i,\ell}$ is the reward sample received when arm $i$ is pulled for the $\ell$-th time, and $z_i(t)$ is the number of times arm $i$ has been pulled until timestep $t$. We denote the true variance of an arm by $\sigma_i^{2}$ while $\hat{v}_{i}(t)$ is the estimated variance, i.e., $\hat{v}_{i}(t)=\frac{1}{z_i(t)}\sum_{\ell=1}^{z_{i}(t)}(X_{i,\ell}-\hat{r}_{i})^{2}$. Whenever there is no ambiguity about the underlaying  time index $t$, for simplicity we neglect $t$ from the notations and simply use  $\hat{r}_i, \hat{v}_i,$ and $z_i$ to denote the respective quantities. We assume the rewards of all arms are bounded in $[0,1]$.

\textbf{2.2 The algorithm:} Earlier round-based arm elimination algorithms like Median Elimination \citep{even2006action} and UCB-Improved mainly suffered from two basic problems: \\
\begin{inparaenum}[\bfseries(i)]
\item \textit{Initial exploration:} Both of these algorithms pull each arm equal number of times in each round, and hence waste a significant number of pulls in initial explorations. \\
\item \textit{Conservative arm-elimination:} In UCB-Improved, arms are eliminated conservatively, i.e, only after $\epsilon_{m}<\frac{\Delta_{i}}{2}$, 
% the sub-optimal arm $i$ is discarded with high probability. 
where the quantity $\epsilon_{m}$ is initialized to $1$ and halved after every round. In the worst case scenario when $K$ is large, and the gaps are uniform  ($r_{1}=r_{2}=\cdots=r_{K-1}<r^{*}$) and small this results in very high regret.\\
\end{inparaenum}
%For any round $m$ UCB-Improved pulls all arms $n_{m}=\left\lceil \frac{ 2\log(T\epsilon_{m})}{\epsilon_{m}} \right\rceil$ number of times. The quantity $\epsilon_{m}$ is initialized to $1$ and halved after every round.
\\
	The EUCBV algorithm, which is mainly based on the arm elimination technique of the UCB-Improved algorithm,  remedies these by employing exploration regulatory factor $\psi$ and arm elimination parameter $\rho$ for aggressive elimination of sub-optimal arms. Along with these, similar to CCB \citep{liu2016modification} algorithm, EUCBV uses optimistic greedy sampling whereby at every timestep it only pulls the arm with the highest upper confidence bound rather than pulling all the arms equal number of times in each round. Also, unlike the UCB-Improved, UCB1, MOSS and OCUCB algorithms (which are based on mean estimation) EUCBV employs mean and variance estimates (as in \citet{audibert2009exploration}) for arm elimination. Further, we allow for arm-elimination at every time-step, which is in contrast to the earlier work (e.g., \citet{auer2010ucb}; \citet{even2006action}) where the arm elimination takes place only at the end of the respective exploration rounds. 






\section{Main Results} 
\label{sec:results}
The main result of the paper is presented in the following theorem, where we establish a regret upper bound for the proposed EUCBV  algorithm. 
% \subsection*{Main Theorem}
\begin{theorem}[\textbf{\textit{Gap-Dependent Bound}}]
\label{Result:Theorem:1}
For $T\geq K^{2.4}$, $\rho=\frac{1}{2}$ and $\psi=\frac{T}{K^2}$, the regret $R_T$ for EUCBV satisfies
\begin{align*}
\E [R_{T}] \leq &\sum\limits_{i\in \A :\Delta_{i} > b}\bigg\lbrace \dfrac{C_0 K^{4}}{T^{\frac{1}{4}}} + \bigg(\Delta_{i}+\dfrac{320\sigma_i^2\log{(\frac{T\Delta_{i}^{2}}{K})}}{\Delta_{i}}\bigg)\bigg \rbrace\\ 
  & +\sum\limits_{i\in \A :0 < \Delta_{i}\leq b} \dfrac{C_2 K^{4}}{T^{\frac{1}{4}}} + \max_{i\in \A :0 < \Delta_{i}\leq b}\Delta_{i}T.
\end{align*}

for all $b\geq\sqrt{\frac{e}{T}}$ and $C_0, C_2$ are integer constants. 
\end{theorem}

\begin{proof}[Outline]
The proof is along the lines of the technique in \citet{auer2010ucb}. It comprises of three modules. In the first module we prove the necessary conditions for arm elimination within a specified number of rounds. However, here we require some additional technical results (see Lemma~\ref{proofTheorem:Lemma:1} and Lemma~\ref{proofTheorem:Lemma:2}) to bound the length of the confidence intervals. Further, note that our algorithm combines the variance-estimate based approach of \citet{audibert2009exploration} with the arm-elimination technique of \citet{auer2010ucb} (see Lemma~\ref{proofTheorem:Lemma:3}). Also, while \citet{auer2010ucb} uses Chernoff-Hoeffding bound to derive their regret bound whereas in our work we use  Bernstein inequality (as in \citet{audibert2009exploration}) to obtain the bound. To bound the probability of the non-uniform arm selection before it gets eliminated we use Lemma~\ref{proofTheorem:Lemma:4} and Lemma~\ref{proofTheorem:Lemma:5}. In the second module we bound the number of pulls required if an arm is eliminated on or before a particular number of rounds. Note that the number of pulls allocated in a round $m$ for each arm is $n_{m}:=\bigg\lceil\frac{\log{(\psi T\epsilon_{m}^{2})}}{2\epsilon_{m}}\bigg\rceil$ which is much lower than the number of pulls of each arm required by UCB-Improved or Median-Elimination. We introduce the variance term in the most significant term in the bound by Lemma~\ref{proofTheorem:Lemma:6}. Finally, the third module deals with case of bounding the regret, given that a sub-optimal arm eliminates the optimal arm.
% (see Lemma~\ref{proofTheorem:Lemma:9}). The detailed proof is available in Section \ref{sec:proofTheorem:Theorem1}.
\hfill $\blacksquare$
\end{proof}

\emph{Discussion:} From the above result we see that the most significant term in the gap-dependent bound is of the order $O\left(\frac{K\sigma^2_{\max}\log{(T\Delta^{2}/K)}}{\Delta}\right)$ which is better than the existing results for UCB1, UCBV, MOSS and UCB-Improved (see Table~\ref{tab:comp-bds}). Also as like UCBV, this term scales with the variance. \citet{audibert2010best} have defined the term $H_1=\sum_{i=1}^{K}\frac{1}{\Delta_i^2}$, which is referred to as the hardness of a problem; \citet{bubeck2012regret} have conjectured that the gap-dependent regret upper bound can match $O\left(\frac{K\log{(T/H_1)}}{\Delta}\right)$. However, in  \citet{lattimore2015optimally} it is proved that the gap-dependent regret bound cannot be lower than $O\left(\sum_{i=2}^{K}\frac{\log\left(T/H_i\right)}{\Delta_i}\right)$, where $H_i=\sum_{j=1}^{K}\min\left\lbrace \frac{1}{\Delta_i^2},\frac{1}{\Delta_j^2}\right\rbrace$ (OCUCB proposed in \citet{lattimore2015optimally} achieves this bound). Further, in \citet{lattimore2015optimally} it is shown that only in the worst case scenario when all the gaps are equal (so that $H_1=H_{i}=\sum_{i=1}^{K}\frac{1}{\Delta^2}$) the above two bounds match. In the latter scenario, considering $\sigma^2_{\max}\leq \frac{1}{4}$ as all rewards are bounded in $[0,1]$, we see that the gap-dependent bound of EUCBV simplifies to $O\left(\frac{K\log{(T/H_1)}}{\Delta}\right)$, thus matching the gap-dependent bound of OCUCB which is order optimal.

Next, we specialize the result of Theorem \ref{Result:Theorem:1} in Corollary \ref{Result:Corollary:1} to  obtain the gap-independent worst case regret bound. %and Corollary \ref{Result:Corollary:2}.


%\subsection*{Corollary 1}

\begin{corollary}[\textbf{\textit{Gap-Independent Bound}}]
\label{Result:Corollary:1}
When the gaps of all the sub-optimal arms are identical, i.e., $\Delta_i =\Delta = \sqrt{\frac{K\log K}{T}}>\sqrt{\frac{e}{T}}, \forall i\in \A$ and $C_3$ being an integer constant, the
regret of EUCBV is upper bounded by the following gap-independent expression:
\begin{align*}
	\E[R_{T}]\leq  \dfrac{C_3 K^5}{T^{\frac{1}{4}}} + 320\sqrt{KT}.
\end{align*}	
\end{corollary}
	
%\begin{proof}
The proof is given in Appendix \ref{App:Corollary:1}.
%\end{proof}

\emph{Discussion:} In the non-stochastic scenario, \citet{auer2002nonstochastic} showed that the bound on the cumulative regret for EXP-4 is $O\left(\sqrt{KT\log K}\right)$. However, in the stochastic case, UCB1 proposed in \citet{auer2002finite} incurred a regret of order of  $O\left(\sqrt{KT\log T}\right)$ which is clearly improvable. From the above result we see that in the gap-independent bound of EUCBV the most significant term is $O\left(\sqrt{KT}\right)$ which  matches the upper bound of MOSS and OCUCB, and is better than UCB-Improved, UCB1 and UCBV (see Table~\ref{tab:comp-bds}).


%%%%%%%%%%%%%%%%%%%%%%%%%%%%%%%%%%
% Shifted to Appendix
%%%%%%%%%%%%%%%%%%%%%%%%%%%%%%%%%%

%\begin{proof}
%\label{Proof:Corollary:1}
%From \cite{bubeck2011pure}  we know that the function $x\in [0,1]\mapsto x\exp(-Cx^2)$ is  decreasing on $\left[\frac{1}{\sqrt{2C}},1\right ]$ for any $C>0$. Thus, we take $C=\left\lfloor \frac{T}{e}\right\rfloor$ and choose  $\Delta_{i}=\Delta=\sqrt{\frac{K\log K}{T}}>\sqrt{\frac{e}{T}}$ for all $i$.
%
%First, let us recall the result in Theorem \ref{Result:Theorem:1} below:
%\begin{align*}
%\E [R_{T}] \leq &\sum\limits_{i\in \A :\Delta_{i} > b}\bigg\lbrace 64 K + \bigg(\Delta_{i}+\dfrac{64\log{(\frac{T\Delta_{i}^{2}}{K})}}{\Delta_{i}}\bigg)\bigg \rbrace\\ 
%  & +\sum\limits_{i\in \A :0 < \Delta_{i}\leq b} 32 K + \max_{i\in \A :0 < \Delta_{i}\leq b}\Delta_{i}T  
%\end{align*}
%
%Now,  with  $\Delta_i =\Delta = \sqrt{\frac{K\log K}{T}}>\sqrt{\frac{e}{T}}$ we obtain,
%	\begin{align*}
%	&\sum_{i\in \A :\Delta_{i} > b}\dfrac{64\log{(\frac{T\Delta_{i}^{2}}{K})}}{\Delta_{i}} \leq  \dfrac{64K\sqrt{T}\log{(T\dfrac{K(\log K)}{T K})}}{\sqrt{K\log K}}\\ 
%	&\leq  \dfrac{64\sqrt{KT}\log{(\log K)}}{\sqrt{\log K}}
%	\overset{(a)}{\leq} 64\sqrt{KT} 
%	\end{align*}		
%	where $(a)$ follows from the identity $\dfrac{\log{(\log K)}}{\sqrt{\log K}}\leq 1$ for $K\geq 2$. Thus, the total worst case gap-independent bound is given by
%	\begin{align*}
%	\E[R_{T}]\leq 96 K^2 + 64\sqrt{KT}.
%	\end{align*}	
%\hfill $\blacksquare$	
%\end{proof}


\section{Proofs}
\label{sec:proofTheorem}
%\subsection{Lemma 1}
%\label{sec:proofTheorem:Lemma1}
We first present a few technical lemmas that is required  to prove the result in Theorem \ref{Result:Theorem:1}.

\begin{lemma}
\label{proofTheorem:Lemma:1}
If $T\geq K^{2.4}$, $\psi=\frac{T}{ K^2}$, $\rho=\frac{1}{2}$ and $m\leq \frac{1}{2} \log_2\left(\frac{T}{e}\right) $, then,
\begin{align*}
\dfrac{\rho m \log(2)}{\log(\psi T) - 2m\log( 2)} \leq \frac{3}{2}.
\end{align*}
\end{lemma}



\begin{lemma}
\label{proofTheorem:Lemma:2}
If $T\geq K^{2.4}$, $\psi=\frac{T}{ K^2}$, $\rho =\frac{1}{2}$, $m_i = min\lbrace m|\sqrt{4\epsilon_{m} } < \frac{\Delta_i}{4} \rbrace $ and $c_{i} =\sqrt{\frac{\rho (\hat{v}_i + 2)\log (\psi T\epsilon_{m_{i}})}{4 z_i}}$, then,
%\begin{align*}
\center $c_{i} < \frac{\Delta_i}{4}$.
%\end{align*}
\end{lemma}



\begin{lemma}
\label{proofTheorem:Lemma:3}
If $m_i = min\lbrace m|\sqrt{4\epsilon_{m} } < \frac{\Delta_i}{4} \rbrace $,  $c_{i} = \sqrt{\frac{\rho (\hat{v}_i + 2) \log (\psi T\epsilon_{m_{i}})}{4 z_{i}}}$ and $n_{m_i} = \frac{\log{(\psi T\epsilon_{m_{i}})}}{2\epsilon_{m_{i}}}$ then we can show that,
\begin{align*}
\mathbb{P}(\hat{r}_{i}> r_{i} + c_{i})\le \dfrac{2}{(\psi  T\epsilon_{m_{i}})^{\frac{3\rho}{2}}}.
\end{align*}
\end{lemma}



%\begin{lemma}
%\label{proofTheorem:Lemma:3}
%If $m_i = min\lbrace m|\sqrt{4\epsilon_{m} } < \frac{\Delta_i}{4} \rbrace $,  $\bar{c}_i=\sqrt{\frac{\rho (\sigma_{i}^{2}+\sqrt{\epsilon_{m_{i}}} + 2)\log(\psi T\epsilon_{m_{i}})}{4z_i}}$ and $n_{m_i} = \frac{\log{(\psi T\epsilon_{m_{i}})}}{2\epsilon_{m_{i}}}$ then we can show that,
%\begin{align*}
%\mathbb{P}\left( \hat{r}_{i} > r_{i}+ \bar{c}_i\right) 
%+ \mathbb{P}\left( \hat{v}_{i}\geq \sigma_{i}^{2}+\sqrt{\epsilon_{m_{i}}}\right) \leq \dfrac{2}{(\psi  T\epsilon_{m_{i}})^{\frac{3\rho}{2}}}.
%\end{align*}
%\end{lemma}



\begin{lemma}
\label{proofTheorem:Lemma:4}
If $m_i = min\lbrace m|\sqrt{4\epsilon_{m} } < \frac{\Delta_i}{4} \rbrace $, $\psi=\frac{T}{ K^2}$, $\rho=\frac{1}{2}$, $c_{i} =\sqrt{\frac{\rho(\hat{v}_i + 2)\log (\psi T\epsilon_{m_{i}})}{4 z_{i}}}$ and $n_{m_i}=\frac{\log{(\psi T\epsilon_{m_{i}}^{2})}}{2\epsilon_{m_{i}}}$ then in the $m_i$-th round, 
\begin{align*}
\Pb\lbrace c^{*} > c_i \rbrace  \leq \dfrac{182 K^4}{T^{\frac{5}{4}}\sqrt{\epsilon_{m_i}}}.
\end{align*}
\end{lemma}



\begin{lemma}
\label{proofTheorem:Lemma:5}
If $m_i = min\lbrace m|\sqrt{4\epsilon_{m} } < \frac{\Delta_i}{4} \rbrace $,$\psi=\frac{T}{ K^2}$, $\rho=\frac{1}{2}$, $c_{i} =\sqrt{\frac{\rho (\hat{v}_i + 2)\log (\psi T\epsilon_{m_{i}})}{4 z_i}}$ and $n_{m_i}=\frac{\log{(\psi T\epsilon_{m_{i}}^{2})}}{2\epsilon_{m_{i}}}$ then in the $m_i$-th round, 
\begin{align*}
\Pb\lbrace z_i < n_{m_i} \rbrace  \leq \dfrac{182 K^4}{T^{\frac{5}{4}}\sqrt{\epsilon_{m_i}}}.
\end{align*}
\end{lemma}



%\begin{lemma}
%\label{proofTheorem:Lemma:6}
%For $T\geq K^{2.4}$, $\epsilon_{m_i}\geq \sqrt{\frac{e}{T}}$, $\psi=\frac{T}{K^2}$ and $\rho=\frac{1}{2}$,  
%\begin{align*}
%\dfrac{6K}{(\psi T \epsilon_{m_i})^{\frac{3\rho}{2}}} > \dfrac{K\log T}{(\psi T)^{3\rho}}\sum_{m=0}^{m_i}\dfrac{1}{\epsilon_{m_i}^{3\rho + 1}}
%\end{align*}
%\end{lemma}



%\begin{lemma}
%\label{proofTheorem:Lemma:6}
%For all bounded rewards in $[0,1]$, $\frac{\Delta_i}{4} \geq \frac{\Delta_i}{4\sigma_i^2 + 4} $.
%\end{lemma}



\begin{lemma}
\label{proofTheorem:Lemma:6}
For two integer constants $c_1$ and $c_2$, if $20 c_1 \leq c_2$ then,
\begin{align*}
c_1 \frac{4\sigma_i^2 + 4}{\Delta_i}\log\bigg( \frac{T\Delta_i^2}{K}\bigg) \leq c_2 \frac{\sigma_i^2}{\Delta_i}\log\bigg( \frac{T\Delta_i^2}{K}\bigg).
\end{align*}
\end{lemma}


%\begin{lemma}
%\label{proofTheorem:Lemma:8}
%If $m_*$ be the first round that the optimal arm $*$ gets eliminated, then we can show that the regret is upper bounded by,
%
%\begin{align*}
%\sum_{m_{*}=0}^{max_{j\in \A^{'}}m_{j}}\sum_{i\in \A^{''}:m_{i}>m_{*}}\bigg(\dfrac{388 K}{(\psi  T\epsilon_{m_{*}})^{\frac{3\rho}{2}}} \bigg).T\max_{j\in \A^{''}:m_{j}\geq m_{*}}{\Delta}_{j} \\
%%%%%%%%%%%%%%%%%%%%%%%%%
% \leq\sum_{i\in \A^{'}}\dfrac{C_2^{'} K^{\frac{5}{2}}}{\sqrt{T\Delta_i}} +\sum_{i\in \A^{''}\setminus \A^{'}}\dfrac{C_2^{'} K^{\frac{5}{2}}}{\sqrt{T b}}
%\end{align*}
%
%\end{lemma}


The proofs of lemmas \ref{proofTheorem:Lemma:1} - \ref{proofTheorem:Lemma:6} can be found in Appendix ~\ref{App:Lemma:1}, ~\ref{App:Lemma:2}, ~\ref{App:Lemma:3}, ~\ref{App:Lemma:4}, ~\ref{App:Lemma:5} and
 ~\ref{App:Lemma:6} respectively.

%The proofs of all the Lemmas can be found in Appendix ~\ref{App:Lemma:1} - Appendix ~\ref{App:Lemma:9} respectively.

\subsection*{Proof of Theorem 1}
\label{sec:proofTheorem:Theorem1}
\begin{customproof}{1}
For each sub-optimal arm ${i}\in\mathcal{A}$, let $m_{i}=\min{\left\lbrace m|\sqrt{4\epsilon_{m_i}} < \frac{\Delta_{i}}{4}\right\rbrace}$. Also, let $\A^{'}=\lbrace i\in \A: \Delta_{i} > b \rbrace$ and $\A^{''}=\lbrace i\in \A: \Delta_{i} > 0 \rbrace$. Note that as all rewards are bounded in $[0,1]$, it implies that $0\leq \sigma_i^2 \leq \frac{1}{4},\forall i\in \A$. Now, as in \citet{auer2010ucb}, we bound the regret under the following two cases: 
\begin{itemize}
\item {Case $(a)$}: some sub-optimal arm ${i}$ is not eliminated in round $m_{i}$ or before and the optimal arm ${*}\in B_{m_{i}}$
\item {Case $(b)$}: an arm ${i}\in B_{m_i}$ is eliminated in round $m_{i}$ (or before), or there is no optimal arm $*\in B_{m_i}$
\end{itemize} 
The details of each case are contained in the following sub-sections.

%Note that in in round $m_i$ as $\sqrt{4\epsilon_{m_i}} < \dfrac{\Delta_{i}}{4}$ implies that $\sqrt{4\epsilon_{m_i}} < \dfrac{\Delta_{i}}{4\sigma_i^2}$, since $\sigma_i^2\in (0,1]$

\textbf{Case $(a)$:}
For simplicity, let $c_{i} := \sqrt{\frac{\rho (\hat{v}_i + 2) \log (\psi T\epsilon_{m_{i}})}{4 z_{i}}}$ denote the length of the confidence interval corresponding to arm $i$ in round $m_i$. Thus, in round $m_i$ (or before) whenever $z_i \geq n_{m_{i}}\ge\frac{\log{(\psi T\epsilon_{m_{i}}^{2})}}{2\epsilon_{m_{i}}}$, by applying Lemma \ref{proofTheorem:Lemma:2} we obtain $c_{i} < \frac{\Delta_{i}}{4}$.
%\begin{align*}
%	c_{i} < \dfrac{\Delta_{i}}{4} 
%\end{align*}
Now, the sufficient conditions for arm $i$ to get eliminated by an optimal arm in round $m_i$ is given by
	\begin{eqnarray}
	\hat{r}_{i} \leq r_{i} + c_{i} \text{, } 
 	\hat{r}^{*} \geq r^{*} - c^{*} \text{, } c_{i} \geq c^* \text{ and } z_i \geq n_{m_i} \label{eq:armelim-casea}.
	\end{eqnarray}

Indeed, in round $m_i$ suppose (\ref{eq:armelim-casea}) holds, then we have
%	 
  \begin{align*}
\hat{r}_{i} + c_{i}&\leq r_{i} + 2c_{i} 
= r_{i} + 4c_{i} - 2c_{i} \\
 &< r_{i} + \Delta_{i} - 2c_{i}
 \leq r^{*} -2c^{*} 
 \leq \hat{r}^{*} - c^{*}
  \end{align*}
  so that a sub-optimal arm ${i} \in \A^{'}$ gets eliminated.	
Thus, the probability of the complementary event of these four conditions in (\ref{eq:armelim-casea}) yields a bound on the probability that arm $i$ is not eliminated in round $m_i$. Following the proof of Lemma 1 of \citet{audibert2009exploration} we can show that a bound on the complementary of the first condition is given by,

\begin{align}
\mathbb{P}(\hat{r}_{i}> r_{i} + c_{i})
&\leq \mathbb{P}\left( \hat{r}_{i} > r_{i}+ \bar{c}_i\right) 
+ \mathbb{P}\left( \hat{v}_{i}\geq \sigma_{i}^{2}+\sqrt{\epsilon_{m_{i}}}\right)\label{eq:prob_eq2}
\end{align}
where 
\begin{align*}
\bar{c}_i=\sqrt{\dfrac{\rho (\sigma_{i}^{2}+\sqrt{\epsilon_{m_{i}}} + 2)\log(\psi T\epsilon_{m_{i}})}{4n_{m_i}}}.
\end{align*}

%%%%%%%%%%%%%%%%%%%%%%%%%%%%%%%%%%
% Shifted as Lemma
%%%%%%%%%%%%%%%%%%%%%%%%%%%%%%%%%%
%Note that, substituting $ n_{m_i} \geq \frac{\log{(\psi T\epsilon_{m_{i}})}}{2\epsilon_{m_{i}}}$, $\bar{c}_i$ can be simplified to obtain,
%\begin{align}
%\bar{c}_i
%\leq \sqrt{\dfrac{\rho\epsilon_{m_{i}}(\sigma_{i}^{2}+\sqrt{\epsilon_{m_{i}}} + 2)}{2}}\leq \sqrt{ \epsilon_{m_{i}}}.
%\label{si_bar_equn}
%\end{align}
%%
%The first term in the LHS of (\ref{eq:prob_eq2}) can be bounded using the Bernstein inequality as below:
%\begin{align}
%&\mathbb{P}\left( \hat{r}_{i} > r_{i}+ \bar{c}_i\right)\nonumber 
%\le \exp\left(- \dfrac{(\bar{c}_i)^2 z_{i}}{2\sigma_i^2 + \frac{2}{3}\bar{c}_i} \right)\nonumber 
%%%%%%%%%%%%%%%%
%\\
%& \overset{(a)}{\le} \exp\left(- \rho \left(\dfrac{3\sigma_{i}^{2}+3\sqrt{\epsilon_{m_{i}}} + 6}{6\sigma_i^2 + 2\sqrt{\epsilon_{m_i}}} \right)\log(\psi  T\epsilon_{m_{i}}\right)\nonumber \\
%%%%%%%%%%%%%%%%
%% &\le \exp\left(- \rho (\sigma_{i}^{2}+\sqrt{\epsilon_{m_{i}}} + 2)\log(\psi  T\epsilon_{m_{i}})\right)\nonumber \\
%%%%%%%%%%%%%%%%
%& \overset{(b)}{\leq} \exp\left(- \rho \log(\psi  T\epsilon_{m_{i}})\right) 
%%%%%%%%%%%%%%%%
%\le \dfrac{1}{(\psi  T\epsilon_{m_{i}})^{\rho}}
%\label{lhs1_equn}
%\end{align}
%where, $(a)$ is obtained by substituting equation \ref{si_bar_equn} and $(b)$ occurs because for all $\sigma_{i}^2 \in [0,1]$, $\left(\dfrac{3\sigma_{i}^{2}+3\sqrt{\epsilon_{m_{i}}} + 6}{6\sigma_i^2 + 2\sqrt{\epsilon_{m_i}}}\right) \geq 1$ .
%
% 
%The second term in the LHS of (\ref{eq:prob_eq2}) can be simplified as follows:
%\begin{align}
%&\mathbb{P}\bigg\lbrace \hat{v}_{i}\geq \sigma_{i}^{2}+\sqrt{\epsilon_{m_{i}}}\bigg\rbrace\nonumber\\
%%%%%%%%%%%%%%%%%%%
%&\leq \mathbb{P}\bigg\lbrace \dfrac{1}{n_{i}}\sum_{t=1}^{n_{i}}(X_{i,t}-r_{i})^{2}-(\hat{r}_{i}-r_{i})^{2}\geq \sigma_{i}^{2}+\sqrt{\epsilon_{m_{i}}}\bigg\rbrace\nonumber\\
%%%%%%%%%%%%%%%%%%%
%&\leq \mathbb{P}\bigg\lbrace \dfrac{\sum_{t=1}^{n_{i}}(X_{i,t}-r_{i})^{2}}{n_{i}}\geq \sigma_{i}^{2}+\sqrt{\epsilon_{m_{i}}} \bigg\rbrace\nonumber\\
%%%%%%%%%%%%%%%%%%%
%&\overset{(a)}{\leq} \mathbb{P}\bigg\lbrace \dfrac{\sum_{t=1}^{n_{i}}(X_{i,t}-r_{i})^{2}}{n_{i}}\geq \sigma_{i}^{2} + \bar{c}_i\bigg\rbrace \nonumber\\
%%%%%%%%%%%%%%%%%%%
%&\overset{(b)}{\leq} \exp\left(- \rho \left(\dfrac{3\sigma_{i}^{2}+3\sqrt{\epsilon_{m_{i}}} + 6}{6\sigma_i^2 + 2\sqrt{\epsilon_{m_i}}} \right)\log(\psi  T\epsilon_{m_{i}})\right)
%%%%%%%%%%%%%%%%%%
%\le \dfrac{1}{(\psi  T\epsilon_{m_{i}})^{\rho}}
%\label{lhs2_equn}
%\end{align}
%where inequality $(a)$ is obtained using (\ref{si_bar_equn}), while $(b)$ follows from the Bernstein inequality.
  
%Thus, using (\ref{lhs1_equn}) and (\ref{lhs2_equn}) in (\ref{eq:prob_eq2}) we obtain $\mathbb{P}(\hat{r}_{i}> r_{i} + c_{i})\le \dfrac{2}{(\psi  T\epsilon_{m_{i}})^{\rho}}$. 

From Lemma \ref{proofTheorem:Lemma:3} we can show that $\mathbb{P}(\hat{r}_{i}> r_{i} + c_{i})\leq\mathbb{P}\left( \hat{r}_{i} > r_{i}+ \bar{c}_i\right) + \mathbb{P}\left( \hat{v}_{i}\geq \sigma_{i}^{2}+\sqrt{\epsilon_{m_{i}}}\right) \leq \frac{2}{(\psi  T\epsilon_{m_{i}})^{\frac{3\rho}{2}}}$. Similarly, $\mathbb{P}\lbrace\hat{r}^{*} < r^{*} - c^{*}\rbrace \leq \frac{2}{(\psi  T\epsilon_{m_{i}})^{\frac{3\rho}{2}}}$. Summing the above two contributions, the probability that a sub-optimal arm ${i}$ is not eliminated on or before $m_{i}$-th round by the first two conditions in  (\ref{eq:armelim-casea}) is,  
\begin{eqnarray}
\bigg(\dfrac{4}{(\psi T\epsilon_{m_{i}})^{\frac{3\rho}{2}}} \bigg). \label{eq:arm:elim:c1}
\end{eqnarray}
 

Again, from Lemma \ref{proofTheorem:Lemma:4} and Lemma \ref{proofTheorem:Lemma:5} we can bound the probability of the  complementary of the event $c_{i} \geq c^* $ and $ z_i \geq n_{m_i}$ by,

\begin{eqnarray}
\dfrac{182 K^4}{T^{\frac{5}{4}}\sqrt{\epsilon_{m_i}}} + \dfrac{182 K^4}{T^{\frac{5}{4}}\sqrt{\epsilon_{m_i}}}\leq \dfrac{364 K^4}{T^{\frac{5}{4}}\sqrt{\epsilon_{m_i}}}. \label{eq:arm:elim:c2}
\end{eqnarray}

Also, for eq. $(\ref{eq:arm:elim:c1})$ we can show that for any $\epsilon_{m_i}\in[\sqrt{\frac{e}{T}},1]$
\begin{eqnarray}
\bigg(\dfrac{4}{(\psi T\epsilon_{m_{i}})^{\frac{3\rho}{2}}} \bigg) &\overset{(a)}{\leq} \bigg(\dfrac{4}{(\frac{T^2}{K^2}\epsilon_{m_{i}})^{\frac{3}{4}}} \bigg)\leq \bigg(\dfrac{4 K^{\frac{3}{2}}}{(T^\frac{3}{2} \epsilon_{m_i}^{\frac{1}{4}}\sqrt{\epsilon_{m_{i}}})}\bigg) \nonumber \\
%%%%%%%%%%%%%%%%%%%%%%%
&\overset{(b)}{\leq} \bigg(\dfrac{4 K^{\frac{3}{2}}}{(T^{\frac{3}{2}-\frac{1}{8}}\sqrt{\epsilon_{m_{i}}})}  \bigg)
\leq \dfrac{4 K^4}{T^{\frac{5}{4}}\sqrt{\epsilon_{m_i}}}. \label{eq:arm:elim:c3}
\end{eqnarray}

Here, in $(a)$ we substitute the values of $\psi$ and $\rho$ and $(b)$ follows from the identity $\epsilon_{m_i}^{\frac{1}{4}}\geq (\frac{e}{T})^{\frac{1}{8}} $ as $\epsilon_{m_i}\geq \sqrt{\frac{e}{T}}$.

Summing up over all arms in $\A^{'}$ and bounding the regret for all the \textit{four} arm elimination conditions in (\ref{eq:armelim-casea}) by $(\ref{eq:arm:elim:c2}) + (\ref{eq:arm:elim:c3})$ for each arm $i\in \A^{'}$ trivially by $T\Delta_{i}$, we obtain
	\begin{align*}
&\sum_{i\in \A^{'}}\bigg(\dfrac{4 K^4 T\Delta_i}{T^{\frac{5}{4}}\sqrt{\epsilon_{m_i}}}\bigg) + \sum_{i\in \A^{'}}\bigg(\dfrac{364 K^4 T\Delta_i}{T^{\frac{5}{4}}\sqrt{\epsilon_{m_i}}}\bigg)\\
%%%%%%%%%%%%%%%%%%%%%%%%%%%%%
&\overset{(a)}{\leq}\sum_{i\in \A^{'}}\bigg(\dfrac{368 K^4 T\Delta_{i}}{T^{\frac{5}{4}}\left(\frac{\Delta_{i}^{2}}{4.16}\right)^{\frac{1}{2}}}\bigg)
%%%%%%%%%%%%%%%%%%%%%%%%%%%%%%%
\overset{(b)}{\leq} \sum_{i\in \A^{'}}\bigg(\dfrac{C_1 K^4}{(T)^{\frac{1}{4}}}\bigg).\\  
%%%%%%%%%%%%%%%%%%%%%%%%%%%%%%%
	\end{align*}

%   \begin{align*}
%&\sum_{i\in \A^{'}}\bigg(\dfrac{388 K T\Delta_{i}}{(\psi T\epsilon_{m_{i}})^{\frac{3\rho}{2}}}\bigg)
%\leq\sum_{i\in \A^{'}}\bigg(\dfrac{388 K T\Delta_{i}}{(\psi T\dfrac{\Delta_{i}^{2}}{4.16})^{\frac{3\rho}{2}}}\bigg)\\
%%%%%%%%%%%%%%%%%%%%%%%%%%%%%%%
%&\leq \sum_{i\in \A^{'}}\bigg(\dfrac{388.2^{2+2\frac{3\rho}{2}}.16^{\frac{3\rho}{2}} K T^{1-\frac{3\rho}{2}}}{\psi^{\frac{3\rho}{2}}\Delta_{i}^{2\frac{3\rho}{2} -1}}\bigg)\\  
%%%%%%%%%%%%%%%%%%%%%%%%%%%%%%%
%& \overset{(a)}{\leq} \sum_{i\in \A^{'}}\bigg(\dfrac{388.2^{2+\frac{3}{2}}.16^{\frac{3}{4}} K T^{1-\frac{3}{4}}}{(\frac{T}{K^2})^{\frac{3}{4}}\Delta_{i}^{2.\frac{3}{4} -1}}\bigg)\leq \sum_{i\in \A^{'}}\dfrac{C_1 K^{\frac{5}{2}}}{\sqrt{T\Delta_i}}  
%   \end{align*}
%Here in $(a)$ we substitute the values of $\rho$ and $\psi$ and $C_1$ denotes a constant integer value.\\
Here, $(a)$ happens because $\sqrt{4\epsilon_{m_i}} < \frac{\Delta_i}{4}$, and in $(b)$, $C_1$ denotes a constant integer value.\\


%%%%%%%%%%%%%%%%%%%%%%%%%%%%%%%%%%%%%
% Case (b)
%%%%%%%%%%%%%%%%%%%%%%%%%%%%%%%%%%%%%
\textbf{Case $(b)$:} Here, there are two sub-cases to be considered.
% \subsection*{Case $b$: \textit{An arm ${i}\in B_{m_i}$ is eliminated in round $m_{i}$ or before or there is no $*\in B_{m_i}$}}

\noindent
\textbf{Case $(b1)$ (\textit{${*}\in B_{m_{i}}$ and each ${i}\in \A^{'}$ is  eliminated on or before $m_{i}$ }): } Since we are eliminating a sub-optimal arm ${i}$ on or before round $m_{i}$, it is pulled no longer than, 
 \begin{align*}
 z_{i} < \bigg\lceil\dfrac{\log{(\psi T\epsilon_{m_{i}}^{2})}}{2\epsilon_{m_{i}}}\bigg\rceil
 \end{align*}
%\hspace*{4em}
%%$, since $\sqrt{\rho_{a}\epsilon_{m_{i}}}\leq\dfrac{\Delta_{i}}{2}
So, the total contribution of ${i}$  till round $m_{i}$ is given by, 
\begin{align*}
&\Delta_{i}\bigg\lceil\dfrac{\log{(\psi T\epsilon_{m_{i}}^{2})}}{2\epsilon_{m_{i}}}\bigg\rceil
\overset{(a)}{\leq}    \Delta_{i}\bigg\lceil\dfrac{\log{(\psi T(\dfrac{\Delta_{i}}{16 \times 256})^{4})}}{2(\dfrac{\Delta_{i}}{4\sqrt{4}})^{2}}\bigg\rceil \\
%%%%%%%%%%%%%%%%%%%%%%%%%%%%%%
&\leq   \Delta_{i}\bigg(1+\dfrac{32\log{(\psi T(\dfrac{\Delta_{i}^{4}}{16384})}}{\Delta_{i}^{2}}\bigg)
\leq \Delta_{i}\bigg(1+\dfrac{32\log{(\psi T\Delta_{i}^{4})}}{\Delta_{i}^{2}}\bigg) .
\end{align*} 

Here, $(a)$ happens because $\sqrt{4\epsilon_{m_{i}}} < \frac{\Delta_{i}}{4}$. Summing over all arms in $\A^{'}$ the total regret is given by, 
\begin{align*}
&\sum_{i\in \A^{'}}\Delta_{i}\bigg(1+\dfrac{32\log{(\psi T\Delta_{i}^{4}})}{\Delta_{i}^{2}}\bigg) = \sum_{i\in \A^{'}}\bigg(\Delta_{i} +\dfrac{32\log{(\psi T\Delta_{i}^{4}})}{\Delta_{i}}\bigg) \\
%%%%%%%%%%%%%%%%%%%%%%%%%%%
&\overset{(a)}{\leq} \sum_{i\in \A^{'}} \left(\Delta_{i}+\dfrac{64\log{( \frac{T\Delta_{i}^{2}}{K})}}{\Delta_{i}}\right)\\
%%%%%%%%%%%%%%%%%%%%%%%%%%%
&\overset{(b)}{\leq} \sum_{i\in \A^{'}} \left(\Delta_{i} +\dfrac{16(4\sigma_i^2 + 4)\log{( \frac{T\Delta_{i}^{2}}{K})}}{\Delta_{i}}\right)\\
&%%%%%%%%%%%%%%%%%%%%%%%%%%%
\overset{(c)}{\leq} \sum_{i\in \A^{'}} \left(\Delta_{i} +\dfrac{320\sigma_i^2\log{( \frac{T\Delta_{i}^{2}}{K})}}{\Delta_{i}}\right).\\
\end{align*}

We obtain $(a)$ by substituting the value of $\psi$, $(b)$ from $0\leq\sigma_i^2 \leq\frac{1}{4},\forall i\in \A$ and $(c)$ from Lemma \ref{proofTheorem:Lemma:6}.\\

\noindent
\textbf{Case $(b2)$ (\textit{Optimal arm ${*}$ is eliminated by a sub-optimal arm):  }} Firstly, if conditions of Case $a$ holds then the optimal arm ${*}$ will not be eliminated in round $m=m_{*}$ or it will lead to the contradiction that $r_{i}>r^{*}$. In any round $m_{*}$, if the optimal arm ${*}$ gets eliminated then for any round from $1$ to $m_{j}$ all arms ${j}$ such that $m_{j}< m_{*}$ were eliminated according to assumption in Case $a$. Let the arms surviving till $m_{*}$ round be denoted by $\A^{'}$. This leaves any arm $a_{b}$ such that $m_{b}\geq m_{*}$ to still survive and eliminate arm ${*}$ in round $m_{*}$. Let such arms that survive ${*}$ belong to $\A^{''}$. Also maximal regret per step after eliminating ${*}$ is the maximal $\Delta_{j}$ among the remaining arms ${j}$ with $m_{j}\geq m_{*}$.  Let $m_{b}=\min\left\lbrace m|\sqrt{4\epsilon_{m}}<\frac{\Delta_{b}}{4}\right\rbrace$. Hence, the maximal regret after eliminating the arm ${*}$ is upper bounded by, 

\begin{align*}
&\sum_{m_{*}=0}^{max_{j\in \A^{'}}m_{j}}\sum_{i\in \A^{''}:m_{i}>m_{*}}\bigg(\dfrac{368 K^4}{(T^{\frac{5}{4}}\sqrt{\epsilon_{m_{*}}})} \bigg).T\max_{j\in \A^{''}:m_{j}\geq m_{*}}{\Delta}_{j}\\
%%%%%%%%%%%%%%%%%%%%%%%%%%%%
&\leq\sum_{m_{*}=0}^{max_{j\in \A^{'}}m_{j}}\sum_{i\in \A^{''}:m_{i}>m_{*}}\bigg(\dfrac{368 K^4 \sqrt{4}}{(T^{\frac{5}{4}}\sqrt{\epsilon_{m_{*}}})} \bigg).T.4\sqrt{\epsilon_{m_{*}}}\\
%%%%%%%%%%%%%%%%%%%%%%%%%%%%
&\overset{(a)}{\leq}\sum_{m_{*}=0}^{max_{j\in \A^{'}}m_{j}}\sum_{i\in \A^{''}:m_{i}>m_{*}}\bigg(\dfrac{C_2 K^4}{T^{\frac{1}{4}}\epsilon_{m_{*}}^{\frac{1}{2}-\frac{1}{2}}} \bigg)\\
%%%%%%%%%%%%%%%%%%%%%%%%%%%%
&\leq\sum_{i\in \A^{''}:m_{i}>m_{*}}\sum_{m_{*}=0}^{\min{\lbrace m_{i},m_{b}\rbrace}}\bigg(\dfrac{C_2 K^4}{T^{\frac{1}{4}}} \bigg)\\
%%%%%%%%%%%%%%%%%%%%%%%%%%%%
&\leq\sum_{i\in \A^{'}}\bigg(\dfrac{C_2 K^4}{T^{\frac{1}{4}}} \bigg)+\sum_{i\in \A^{''}\setminus \A^{'}}\bigg(\dfrac{C_2 K^4}{T^{\frac{1}{4}}} \bigg).\\
\end{align*}
Here at $(a)$, $C_2$ denotes an integer constant.



%\begin{align*}
%\sum_{m_{*}=0}^{max_{j\in \A^{'}}m_{j}}\sum_{i\in \A^{''}:m_{i}>m_{*}}\bigg(\dfrac{388 K}{(\psi  T\epsilon_{m_{*}})^{\frac{3\rho}{2}}} \bigg).T\max_{j\in \A^{''}:m_{j}\geq m_{*}}{\Delta}_{j}
%\end{align*}
%
%Again applying Lemma \ref{proofTheorem:Lemma:8} we can show that the above expression is upper bounded by 
%\begin{align*}
%\sum_{i\in \A^{'}}\dfrac{C_2^{'} K^{\frac{5}{2}}}{\sqrt{T\Delta_i}} +\sum_{i\in \A^{''}\setminus \A^{'}}\dfrac{C_2^{'} K^{\frac{5}{2}}}{\sqrt{T b}}
%\end{align*}

%%%%%%%%%%%%%%%%%%%%%%%%%%%%%%%%
%Moved to Appendix as Lemma 9
%%%%%%%%%%%%%%%%%%%%%%%%%%%%%%%%

%\begin{align*}
%&\sum_{m_{*}=0}^{max_{j\in \A^{'}}m_{j}}\sum_{i\in \A^{''}:m_{i}>m_{*}}\bigg(\dfrac{388 K}{(\psi  T\epsilon_{m_{*}})^{\frac{3\rho}{2}}} \bigg).T\max_{j\in \A^{''}:m_{j}\geq m_{*}}{\Delta}_{j}\\
%%%%%%%%%%%%%%%%%%%%%%%%%%%%%
%&\leq\sum_{m_{*}=0}^{max_{j\in \A^{'}}m_{j}}\sum_{i\in \A^{''}:m_{i}>m_{*}}\bigg(\dfrac{388 K\sqrt{4}}{(\psi  T\epsilon_{m_{*}})^{\frac{3\rho}{2}}} \bigg).T.4\sqrt{\epsilon_{m_{*}}}\\
%%%%%%%%%%%%%%%%%%%%%%%%%%%%%
%&\leq\sum_{m_{*}=0}^{max_{j\in \A^{'}}m_{j}}\sum_{i\in \A^{''}:m_{i}>m_{*}}C_2 K\bigg(\dfrac{T^{1-\frac{3\rho}{2}}}{\psi^{\frac{3\rho}{2}}\epsilon_{m_{*}}^{\frac{3\rho}{2}-\frac{1}{2}}} \bigg)\\
%%%%%%%%%%%%%%%%%%%%%%%%%%%%%
%&\leq\sum_{i\in \A^{''}:m_{i}>m_{*}}\sum_{m_{*}=0}^{\min{\lbrace m_{i},m_{b}\rbrace}}\bigg(\dfrac{C_2 K T^{1-\frac{3\rho}{2}}}{\psi^{\frac{3\rho}{2}}2^{-(\frac{3\rho}{2} -\frac{1}{2})m_{*}}} \bigg)\\
%%%%%%%%%%%%%%%%%%%%%%%%%%%%%
%&\leq\sum_{i\in \A^{'}}\bigg(\dfrac{C_2 K T^{1-\frac{3\rho}{2}}}{\psi^{\frac{3\rho}{2}}2^{-(\frac{3\rho}{2} -\frac{1}{2})m_{*}}} \bigg)+\sum_{i\in \A^{''}\setminus \A^{'}}\bigg(\dfrac{C_2 K T^{1-\frac{3\rho}{2} }}{\psi^{\frac{3\rho}{2}}2^{-(\frac{3\rho}{2} -\frac{1}{2})m_{b}}} \bigg)\\
%%%%%%%%%%%%%%%%%%%%%%%%%%%%%
%&\leq\sum_{i\in \A^{'}}\bigg(\dfrac{C_2 K T^{1-\frac{3\rho}{2}}.2^{\frac{\frac{3\rho}{2}}{2}-\frac{1}{4}}}{\psi^{\frac{3\rho}{2}}\Delta_{i}^{\frac{3\rho}{2} -\frac{1}{2}}} \bigg)+\sum_{i\in \A^{''}\setminus \A^{'}}\bigg(\dfrac{C_2 K T^{1-\frac{3\rho}{2}}.2^{\frac{\frac{3\rho}{2}}{2}-\frac{1}{4}}}{\psi^{\frac{3\rho}{2}}b^{\frac{3\rho}{2} -\frac{1}{2}}} \bigg)\\
%%%%%%%%%%%%%%%%%%%%%%%%%%%%%
%&\leq\sum_{i\in \A^{'}}\bigg(\dfrac{ C_2 K 2^{\frac{\frac{3\rho}{2}}{2}+\frac{19}{4}}.T^{1-\frac{3\rho}{2} } }{\psi^{\rho}\Delta_{i}^{2\frac{3\rho}{2} -1}} \bigg)+\sum_{i\in \A^{''}\setminus \A^{'}}\bigg(\dfrac{C_2 K 2^{\frac{\frac{3\rho}{2}}{2}+\frac{19}{4}}.T^{1-\frac{3\rho}{2}} }{\psi^{\frac{3\rho}{2} }b^{2\frac{3\rho}{2}-1}} \bigg)\\
%%%%%%%%%%%%%%%%%%%%%%%%%%%%%
%&\overset{(a)}{\leq}\sum_{i\in \A^{'}}\bigg(\dfrac{C_2^{'} K .T^{1-\frac{3}{4}}}{(\frac{T}{K^2})^{\frac{3}{4}}\Delta_{i}^{2.\frac{3}{4} -1}} \bigg)+\sum_{i\in \A^{''}\setminus \A^{'}}\bigg(\dfrac{C_2^{'} K T^{1-\frac{3}{4}}}{(\frac{T}{K^2})^{\frac{3}{4}}b^{2.\frac{3}{4}-1}} \bigg)\\
%%%%%%%%%%%%%%%%%%%%%%%%%%%%%
%&\leq\sum_{i\in \A^{'}}\dfrac{C_2^{'} K^{\frac{5}{2}}}{\sqrt{T\Delta_i}} +\sum_{i\in \A^{''}\setminus \A^{'}}\dfrac{C_2^{'} K^{\frac{5}{2}}}{\sqrt{T b}}
%%%%%%%%%%%%%%%%%%%%%%%%%%%%%
%\end{align*}
%In the above simplification, $(a)$ is obtained by substituting the values of $\psi$ and $\rho$.

Finally, summing up the regrets in \textbf{Case a} and \textbf{Case b}, the total regret is given by
\begin{align*}
\E [R_{T}] \leq &\sum\limits_{i\in \A :\Delta_{i} > b}\bigg\lbrace \dfrac{C_0 K^{4}}{T^{\frac{1}{4}}} + \bigg(\Delta_{i}+\dfrac{320\sigma_i^2\log{(\frac{T\Delta_{i}^{2}}{K})}}{\Delta_{i}}\bigg)\bigg \rbrace\\ 
  & +\sum\limits_{i\in \A :0 < \Delta_{i}\leq b} \dfrac{C_2 K^{4}}{T^{\frac{1}{4}}} + \max_{i\in \A :0 < \Delta_{i}\leq b}\Delta_{i}T
\end{align*}

where $C_0, C_1, C_2$ are integer constants s.t. $C_0 = C_1 + C_2$.
\end{customproof}




\section{Experiments}
\label{sec:expt}
In this section, we conduct extensive empirical evaluations of EUCBV against several other popular MAB  algorithms. We use expected cumulative regret as the metric of comparison. The comparison is conducted against the following algorithms: KLUCB+ \citep{garivier2011kl}, DMED \citep{honda2010asymptotically}, MOSS \citep{audibert2009minimax}, UCB1 \citep{auer2002finite}, UCB-Improved \citep{auer2010ucb}, Median Elimination \citep{even2006action}, Thompson Sampling (TS) \citep{agrawal2011analysis}, OCUCB \citep{lattimore2015optimally}, Bayes-UCB (BU) \citep{kaufmann2012bayesian} and UCB-V \citep{audibert2009exploration}\footnote{The implementation for KLUCB, Bayes-UCB and DMED were taken from \citet{CapGarKau12}}. The parameters of EUCBV algorithm for all the experiments are set as follows: $\psi=\frac{T}{K^2}$ and $\rho =0.5$ (as in Corollary \ref{Result:Corollary:1}). Note that KLUCB+ empirically outperforms KLUCB (as shown in \citet{garivier2011kl}).

\begin{figure}[!h]
    \centering
    \begin{tabular}{cc}
    \setlength{\tabcolsep}{0.1pt}
    \subfigure[0.25\textwidth][Expt-$1$: $20$ Bernoulli-distributed arms ]
    %with $r_{i_{{i}\neq {*}}}=0.07$ and $r^{*}=0.1$
    {
    		\pgfplotsset{
		tick label style={font=\Large},
		label style={font=\Large},
		legend style={font=\Large},
		ylabel style={yshift=5pt},
		%legend style={legendshift=32pt},
		}
        \begin{tikzpicture}[scale=0.6]
      	\begin{axis}[
		xlabel={timestep},
		ylabel={Cumulative Regret},
		grid=major,
        %clip mode=individual,grid,grid style={gray!30},
        clip=true,
        %clip mode=individual,grid,grid style={gray!30},
  		legend style={at={(0.5,1.5)},anchor=north, legend columns=3} ]
      	% UCB
		\addplot table{EUCBV/results/NewExpt/Expt1/UCBV01_comp_subsampled.txt};
		\addplot table{EUCBV/results/NewExpt/Expt1/EUCBV01_comp_subsampled.txt};
		\addplot table{EUCBV/results/NewExpt/Expt1/KLUCB01_comp_subsampled.txt};
		\addplot table{EUCBV/results/NewExpt/Expt1/MOSS01_comp_subsampled.txt};
		\addplot table{EUCBV/results/NewExpt/Expt1/DMED01_comp_subsampled.txt};
		\addplot table{EUCBV/results/NewExpt/Expt1/UCB01_comp_subsampled.txt};
		\addplot table{EUCBV/results/NewExpt/Expt1/TS01_comp_subsampled.txt};
		\addplot table{EUCBV/results/NewExpt/Expt1/OCUCB01_comp_subsampled.txt};
		\addplot table{EUCBV/results/NewExpt/Expt1/BU01_comp_subsampled.txt};
      	\legend{UCB-V,EUCBV,KLUCB+,MOSS,DMED,UCB1,TS,OCUCB,BU}      	
      	\end{axis}
      	\end{tikzpicture}
  		\label{fig:1}
    }
    &
    \subfigure[0.25\textwidth][Expt-$2$: $3$ Group Mean Setting ]
    %with $r_{i_{{i}\neq {*}:1-33}}=0.1$, $r_{i_{{i}\neq {*}:34-99}}=0.6$, $r^{*}_{i=100}=0.9$ and $\sigma_{i=1:100}^{2} = 0.3$
    {
    		\pgfplotsset{
		tick label style={font=\Large},
		label style={font=\Large},
		legend style={font=\Large},
		ylabel style={yshift=5pt},
		}
        \begin{tikzpicture}[scale=0.6]
        \begin{axis}[
		xlabel={timestep},
		ylabel={Cumulative Regret},
        %clip mode=individual,grid,grid style={gray!30},
       	grid=major,
       	clip=true,
  		legend style={at={(0.5,1.5)},anchor=north, legend columns=3} ]
      	% UCB
		\addplot table{EUCBV/results/NewExpt/Expt2/UCBV01_comp_subsampled.txt};
		\addplot table{EUCBV/results/NewExpt/Expt2/EUCBV01_comp_subsampled.txt};
		\addplot table{EUCBV/results/NewExpt/Expt2/KLUCB01_comp_subsampled.txt};
		\addplot table{EUCBV/results/NewExpt/Expt2/MOSS01_comp_subsampled.txt};
		\addplot table{EUCBV/results/NewExpt/Expt2/UCBR01_comp_subsampled.txt};
		\addplot table{EUCBV/results/NewExpt/Expt2/UCB01_comp_subsampled.txt};
		\addplot table{EUCBV/results/NewExpt/Expt2/TS01_comp_subsampled.txt};
		\addplot table{EUCBV/results/NewExpt/Expt2/OCUCB01_comp_subsampled.txt};
		\addplot table{EUCBV/results/NewExpt/Expt2/BU01_comp_subsampled.txt};      	
      	\legend{UCB-V,EUCBV,KLUCB-G+,MOSS,UCB-Imp,UCB1,TS-G,OCUCB,BU-G}
      	\end{axis}
      	\end{tikzpicture}
   		\label{fig:2}
    }
    \end{tabular}
    \caption{A comparison of the cumulative regret incurred by the various bandit algorithms. }
    \label{fig:karmed}
    \vspace*{-1em}
\end{figure}
% For the purpose of performance comparison


\textbf{Experiment-1 (Bernoulli with uniform gaps):} This experiment is conducted to observe the performance of EUCBV over a short horizon. The horizon $T$ is set to $60000$. The testbed comprises of $20$ Bernoulli distributed arms with expected rewards of the arms as $r_{1:19}=0.07$ and $r^{*}_{20}=0.1$ and these type of cases are frequently encountered in web-advertising domain (see \cite{garivier2011kl}). The regret is averaged over $100$ independent runs and is shown in Figure \ref{fig:1}. EUCBV, MOSS, OCUCB, UCB1, UCB-V, KLUCB+, TS, BU and DMED are run in this experimental setup. Not only do we observe that EUCBV performs better than all the non-variance based algorithms such as MOSS, OCUCB, UCB-Improved and UCB1, but it also outperforms UCBV because of the choice of the exploration parameters. Because of the small gaps and short horizon $T$, we do not compare with UCB-Improved and Median Elimination for this test-case. 

\textbf{Experiment-2 (Gaussian $3$ Group Mean Setting):} This experiment is conducted to observe the performance of EUCBV over a large horizon in Gaussian distribution testbed. This setting comprises of a large horizon of $T = 3\times 10^{5}$ timesteps and a large set of arms. This testbed comprises of $100$ arms involving Gaussian reward distributions with expected rewards of the arms in $3$ groups, $r_{1:66}=0.07$, $r_{67:99}=0.01$ and $r^{*}_{100}=0.09$ with variance set as $\sigma_{1:66}^{2} = 0.01,\sigma_{67:99}^{2} = 0.25$ and $\sigma^{2}_{100}=0.25$. The regret is averaged over $100$ independent runs and is shown in Figure \ref{fig:2}. From the results in Figure \ref{fig:2}, we observe that since the gaps are small and the variances of the optimal arm and the arms farthest from the optimal arm are the highest, EUCBV, which allocates pulls proportional to the variances of the arms, outperforms all the non-variance based algorithms MOSS, OCUCB, UCB1, UCB-Improved and Median-Elimination ($\epsilon=0.1,\delta=0.1$). The performance of Median-Elimination is extremely weak in comparison with the other algorithms and its plot is not shown in Figure \ref{fig:2}. We omit its plot in order to more clearly show the difference between EUCBV, MOSS and OCUCB. Also note that the order of magnitude in the y-axis (cumulative regret) of Figure \ref{fig:2} is $10^4$. KLUCB-Gauss+ (denoted by KLUCB-G+), TS-G and BU-G are initialized with Gaussian priors. Both KLUCB-G+ and UCBV which is a variance-aware algorithm perform much worse than TS-G and EUCBV. The performance of DMED is similar to KLUCB-G+ in this setup and its plot is omitted. 


\begin{figure}[!h]
    \centering
    \begin{tabular}{cc}
    \subfigure[0.25\textwidth][Expt-$3$: Failure of TS]
    {
    		\pgfplotsset{
		tick label style={font=\Large},
		label style={font=\Large},
		legend style={font=\Large},
		ylabel style={yshift=5pt},
		}
        \begin{tikzpicture}[scale=0.6]
      	\begin{axis}[
		ylabel={Cumulative Regret},
		xlabel={timestep},
		grid=major,
        %clip mode=individual,grid,grid style={gray!30},
        clip=true,
        %clip mode=individual,grid,grid style={gray!30},
  		legend style={at={(0.5,1.3)},anchor=north, legend columns=3} ]
      	% UCB
		\addplot table{EUCBV/results/NewExpt/Expt3/UCBV01_comp_subsampled.txt};
		\addplot table{EUCBV/results/NewExpt/Expt3/EUCBV01_comp_subsampled.txt};
		\addplot table{EUCBV/results/NewExpt/Expt3/MOSS01_comp_subsampled.txt};
		\addplot table{EUCBV/results/NewExpt/Expt3/TS01_comp_subsampled.txt};
		\addplot table{EUCBV/results/NewExpt/Expt3/OCUCB01_comp_subsampled.txt};
		\addplot table{EUCBV/results/NewExpt/Expt3/BU01_comp_subsampled.txt};
      	\legend{UCBV,EUCBV,MOSS,TS-G,OCUCB,BU-G} 
      	\end{axis}
      	\end{tikzpicture}
  		\label{fig:3}
    }
    &
    \subfigure[0.25\textwidth][Expt-$4$: $3$ Group Variance Setting]
    %with $r_{i_{{i}\neq {*}}}=0.05$ and $r^{*}=0.1$
    {
    	\pgfplotsset{
		tick label style={font=\Large},
		label style={font=\Large},
		legend style={font=\Large},
		ylabel style={yshift=5pt},
		}
        \begin{tikzpicture}[scale=0.6]
        \begin{axis}[
		xlabel={timestep},
		ylabel={Cumulative Regret},
        %clip mode=individual,grid,grid style={gray!30},
		grid=major,
		clip=true,
  		legend style={at={(0.5,1.3)},anchor=north, legend columns=3} ]
        % UCB
		\addplot table{EUCBV/results/NewExpt/Expt41/UCBV01_comp_subsampled.txt};
		\addplot table{EUCBV/results/NewExpt/Expt41/EUCBV01_comp_subsampled.txt};
		\addplot table{EUCBV/results/NewExpt/Expt41/MOSS01_comp_subsampled.txt};
		\addplot table{EUCBV/results/NewExpt/Expt41/TS01_comp_subsampled.txt};
		\addplot table{EUCBV/results/NewExpt/Expt41/OCUCB01_comp_subsampled.txt};
		\addplot table{EUCBV/results/NewExpt/Expt41/BU01_comp_subsampled.txt};
      	\legend{UCBV,EUCBV,MOSS,TS-G,OCUCB,BU-G} 
      	\end{axis}
        \end{tikzpicture}
        \label{fig:4}
    }
	\end{tabular}
	\label{fig:furtherExpt1}
    \caption{Further Experiments with EUCBV}
    \vspace*{-1em}
\end{figure}
%\vspace*{-0.5em}


%Because of the poor performance of Bayes-UCB and KL-UCB in the last two experiments we do not implement them in this setup.
%Please note that algorithms like Thompson Sampling or Bayes-UCB are too slow to be run for such large arms (see \citet{lattimore2015optimally}) and  over such large horizon.
%This also corroborates the finding of \citet{lattimore2015optimally} which states that MOSS breaks down only when the number of arms are exceptionally large or the horizon is unreasonably high and gaps are very small. We consistently see that in uniform gap testcases EUCBV outperforms OCUCB.

\textbf{Experiment-3 (Failure of TS):} This experiment is conducted to demonstrate that in certain environments when the horizon is large, gaps are small and the variance of the optimal arm is high, the Bayesian algorithms (like TS) do not perform well but EUCBV performs exceptionally well. This experiment is conducted on $100$ Gaussian distributed arms such that expected rewards of the arms $r_{1:10}=0.045$, $r_{11:99}=0.04$, $r^{*}_{100}=0.05$ and the variance is set as $\sigma_{1:10}^{2}=0.01$,   $\sigma_{100}^{2}=0.25$ and $T=4\times 10^5$. The variance of the arms $i=11:99$ are chosen uniform randomly between $[0.2,0.24]$. TS and BU with Gaussian priors fail because here the chosen variance values are such that only variance-aware algorithms with appropriate exploration factors will perform  well or otherwise it will get bogged down in costly exploration. The algorithms that are not variance-aware will spend a significant amount of pulls trying to find the optimal arm. The result is shown in Figure \ref{fig:3}. Predictably EUCBV, which allocates pulls proportional to the variance of the arms, outperforms its closest competitors TS-G, BU-G, UCBV, MOSS and OCUCB. The plots for KLUCB-G+, DMED, UCB1, UCB-Improved and Median Elimination are omitted from the figure as their performance is extremely weak in comparison with other algorithms. We omit their plots to clearly show how EUCBV outperforms its nearest competitors. Note that EUCBV by virtue of its aggressive exploration parameters outperforms UCBV in all the experiments even though UCBV is a variance-based algorithm. The performance of TS-G is also weak and this is in line with the observation in \citet{lattimore2015optimally} that the worst case regret of TS when Gaussian prior is used is $\Omega\left( \sqrt{KT\log T}\right)$.



\textbf{Experiment-4 (Gaussian $3$ Group Variance setting):} This experiment is conducted to show that when the gaps are uniform and variance of the arms are the only discriminative factor then the EUCBV performs extremely well over a very large horizon and over a large number of arms. This testbed comprises of $100$ arms with Gaussian reward distributions, where the expected rewards of the arms are $r_{1:99}=0.09$ and $r^{*}_{100}=0.1$. The variances of the arms are divided into $3$ groups. The group $1$ consist of arms $i=1:49$ where the variances are chosen uniform randomly between $[0.0,0.05]$, group $2$ consist of arms $i=50:99$ where the variances are chosen uniform randomly between $[0.19,0.24]$ and for the optimal arm $i=100$ (group $3$) the variance is set as $\sigma_{*}^{2}=0.25$. We report the cumulative regret averaged over $100$ independent runs. The horizon is set at $T=4\times 10^{5}$ timesteps. We report the performance of MOSS,BU-G, UCBV, TS-G and OCUCB who are the closest competitors of EUCBV over this uniform gap setup. From the results in Figure \ref{fig:4}, it is evident that the growth of regret for EUCBV  is much lower than that of TS-G, MOSS, BU-G, OCUCB and UCBV. Because of the poor performance of KLUCB-G+ in the last two experiments we do not implement it in this setup. Also, note that for optimal performance BU-G, TS-G and KLUCB-G+ require the knowledge of the type of distribution to set their priors . Also, in all the experiments with Gaussian distributions EUCBV significantly outperforms all the Bayesian algorithms initialized with Gaussian priors.




\section{Conclusion and Future Works}
\label{sec:conc}
In this paper, we studied the EUCBV algorithm which takes into account the empirical variance of the arms and  employs aggressive exploration parameters in conjunction with non-uniform arm selection (as opposed to UCB-Improved) to eliminate sub-optimal arms. Our theoretical analysis conclusively established that EUCBV exhibits an order-optimal gap-independent regret bound of $O\left(\sqrt{KT}\right)$. Empirically, we show that EUCBV performs superbly across diverse experimental settings and outperforms most of the bandit algorithms in a  stochastic MAB setup. Our experiments show that EUCBV is extremely stable for larger horizons and performs consistently well across different types of distributions. One avenue for future work is to remove the constraint of $T\geq K^{2.4}$ required for EUCBV to reach the order optimal regret bound. Another future direction is to come up with an anytime version of EUCBV. An anytime algorithm does not need the horizon $T$ as an input parameter.


%\section*{Acknowledgement} This work was supported by a funding from IIT Madras under project CSE/14-15/831/ RFTP/BRAV.

\section{Summary}
\label{tbandit:Summary}
In this chapter we looked at a novel variant of the UCB algorithm (referred to as Efficient-UCB-Variance (EUCBV)) for minimizing cumulative regret in the stochastic multi-armed bandit (MAB) setting. EUCBV incorporates the arm elimination strategy proposed in UCB-Improved \citep{auer2010ucb}, while taking into account the variance estimates to compute the arms' confidence bounds, similar to UCBV \citep{audibert2009exploration}. Through a theoretical analysis we establish that EUCBV incurs a \emph{gap-dependent} regret bound of {\scriptsize $O\left( \dfrac{K\sigma^2_{\max} \log (T\Delta^2 /K)}{\Delta}\right)$} after $T$ trials, where $\Delta$ is the minimal gap between optimal and sub-optimal arms; the above bound is an improvement over that of existing state-of-the-art UCB algorithms (such as UCB1, UCB-Improved, UCBV,  MOSS). Further, EUCBV incurs a \emph{gap-independent} regret bound of {\scriptsize $O\left(\sqrt{KT}\right)$}  which is an improvement over that of UCB1, UCBV and UCB-Improved, while being comparable with that of MOSS and OCUCB. Through an extensive numerical study we show that EUCBV significantly outperforms the popular UCB variants (like MOSS, OCUCB, etc.) as well as Thompson sampling and Bayes-UCB algorithms. 


%%%%%%%%%%%%%%%%%%%%%%%%%%%%%%%%%%%%%%%%%%%%%%%%%%%%%%%%%%%%

%%%%%%%%%%%%%%%%%%%%%%%%%%%%%%%%%%%%%%%%%%%%%%%%%%%%%%%%%%%%
% Appendices.

\appendix

\chapter{APPENDIX}
\section{Appendix for EUCBV}
\label{sec:app}

\subsection{Proof of Lemma \ref{proofTheorem:Lemma:1}} 

\label{App:Lemma:1}

\begin{customlem}{1}
%\label{proofTheorem:Lemma:1}
If $T\geq K^{2.4}$, $\psi=\dfrac{T}{ K^2}$, $\rho=\dfrac{1}{2}$ and $m\leq \dfrac{1}{2} \log_2\left(\dfrac{T}{e}\right) $, then,
\begin{align*}
\dfrac{\rho m \log(2)}{\log(\psi T) - 2m\log( 2)} \leq \frac{3}{2}.
\end{align*}
\end{customlem}

\begin{proof}
The proof is based on contradiction. Suppose
\begin{eqnarray*}
\dfrac{\rho m \log(2)}{\log(\psi T) - 2m\log( 2)} > \frac{3}{2}.
\end{eqnarray*}
Then, with $\psi=\dfrac{T}{ K^2}$ and $\rho=\dfrac{1}{2}$, we obtain
\begin{eqnarray*}
6\log(K) 
&>& 6\log(T) - 7m\log(2) \\
&\overset{(a)}{\ge}& 6\log(T) - \frac{7}{2} \log_2\left(\frac{T}{e}\right) \log(2) \\
&=& 2.5\log(T) + 3.5 \log_2(e)\log(2)  \\
&\overset{(b)}{=}& 2.5\log(T) +3.5
\end{eqnarray*}
where $(a)$ is obtained using $m\leq \dfrac{1}{2} \log_2\left(\dfrac{T}{e}\right)$, while $(b)$ follows from the identity $\log_2(e)\log(2) =1$. Finally, for $T\ge K^{2.4}$ we obtain, $6\log(K)>6\log(K)+3.5$, which is a contradiction.
\hfill $\blacksquare$	
\end{proof}

\subsection{Proof of Lemma \ref{proofTheorem:Lemma:2}}
\label{App:Lemma:2}
\begin{customlem}{2}
%\label{proofTheorem:Lemma:2}
If $T\geq K^{2.4}$, $\psi=\dfrac{T}{ K^2}$, $\rho =\dfrac{1}{2}$, $m_i = min\lbrace m|\sqrt{4\epsilon_{m} } < \dfrac{\Delta_i}{4} \rbrace $ and $c_{i} =\sqrt{\frac{\rho (\hat{v}_i + 2)\log (\psi T\epsilon_{m_{i}})}{4 z_i}}$, then, 
\begin{align*}
c_{i} < \dfrac{\Delta_i}{4}
\end{align*}

\end{customlem}

\begin{proof}

	In the $m_i$-th round since $z_i\geq n_{m_i}$, by substituting $z_i$ with $n_{m_i}$ we can show that, 

\begin{align*}
	c_{i} &\leq \sqrt{\dfrac{\rho (\hat{v}_i + 2)\epsilon_{m_{i}}\log (\psi T\epsilon_{m_{i}})}{2\log(\psi T\epsilon_{m_{i}}^{2})}} \overset{(a)}{\leq} \sqrt{\dfrac{2\rho\epsilon_{m_{i}}\log (\frac{\psi T\epsilon_{m_{i}}^{2}}{\epsilon_{m_{i}}})}{\log(\psi T\epsilon_{m_{i}}^{2})}} \\
	%%%%%%%%%%%%%%%%%%%%%%%%%%
	& = \sqrt{\dfrac{2\rho\epsilon_{m_{i}}\log (\psi T\epsilon_{m_{i}}^{2}) - 2\rho\epsilon_{m_{i}}\log (\epsilon_{m_{i}})}{\log(\psi T\epsilon_{m_{i}}^{2})}} \\
	%%%%%%%%%%%%%%%%%%%%%%%%%%
	& \leq  \sqrt{2\rho\epsilon_{m_{i}} - \dfrac{2\rho\epsilon_{m_i}\log(\frac{1}{2^{m_i}})}{\log(\psi T \frac{1}{2^{2m_i}})}} \\
	%%%%%%%%%%%%%%%%%%%%%%%%%%
	&\leq \sqrt{2\rho\epsilon_{m_{i}} + \dfrac{2\rho\epsilon_{m_i}\log(2^{m_i})}{\log(\psi T) - \log( 2^{2m_i})}}\\
	%%%%%%%%%%%%%%%%%%%%%%%%%%
	& \leq \sqrt{2\rho\epsilon_{m_{i}} + \dfrac{2\rho\epsilon_{m_i}m_i \log(2)}{\log(\psi T) - 2m_i\log( 2)}} \\ 
	%%%%%%%%%%%%%%%%%%%%%%%%%%
	 & \overset{(b)}{\leq} \sqrt{2\rho\epsilon_{m_{i}} + 2.\frac{3}{2}\epsilon_{m_i}} 
	  < \sqrt{4\epsilon_{m_i}} < \dfrac{\Delta_{i}}{4}.
	\end{align*}
In the above simplification, $(a)$ is due to $\hat{v}_i \in [0,1]$, while $(b)$ is obtained using Lemma~\ref{proofTheorem:Lemma:1}.
% < \dfrac{\Delta_{i}}{4 \sigma_i^2} \overset{(c)}{<}
% and $(c)$ happens because $\sigma_i \in (0,1]$. 
%Similarly, it can be shown that $c^* < \frac{\Delta_i}{4}$ in round $m_i$.
\hfill $\blacksquare$	
\end{proof}


\subsection{Proof of Lemma \ref{proofTheorem:Lemma:3}}
\label{App:Lemma:3}

\begin{customlem}{3}
If $m_i = min\lbrace m|\sqrt{4\epsilon_{m} } < \frac{\Delta_i}{4} \rbrace $,  $c_{i} = \sqrt{\frac{\rho (\hat{v}_i + 2) \log (\psi T\epsilon_{m_{i}})}{4 z_{i}}}$ and $n_{m_i} = \frac{\log{(\psi T\epsilon_{m_{i}})}}{2\epsilon_{m_{i}}}$ then we can show that,
\begin{align*}
\mathbb{P}(\hat{r}_{i}> r_{i} + c_{i})\le \dfrac{2}{(\psi  T\epsilon_{m_{i}})^{\frac{3\rho}{2}}}.
\end{align*}
\end{customlem}

%\begin{customlem}{3}
%If $m_i = min\lbrace m|\sqrt{4\epsilon_{m} } < \dfrac{\Delta_i}{4} \rbrace $,  $\bar{c}_i=\sqrt{\dfrac{\rho (\sigma_{i}^{2}+\sqrt{\epsilon_{m_{i}}} + 2)\log(\psi T\epsilon_{m_{i}})}{4z_i}}$ and $n_{m_i} = \frac{\log{(\psi T\epsilon_{m_{i}})}}{2\epsilon_{m_{i}}}$ then we can show that,
%\begin{align*}
%\mathbb{P}\left( \hat{r}_{i} > r_{i}+ \bar{c}_i\right) 
%+ \mathbb{P}\left( \hat{v}_{i}\geq \sigma_{i}^{2}+\sqrt{\epsilon_{m_{i}}}\right) \leq \dfrac{2}{(\psi  T\epsilon_{m_{i}})^{\frac{3\rho}{2}}}.
%\end{align*}
%\end{customlem}

\begin{proof}

We start by recalling from equation (\ref{eq:prob_eq2}) that,

\begin{align}
\mathbb{P}(\hat{r}_{i}> r_{i} + c_{i})
&\leq \mathbb{P}\left( \hat{r}_{i} > r_{i}+ \bar{c}_i\right) 
+ \mathbb{P}\left( \hat{v}_{i}\geq \sigma_{i}^{2}+\sqrt{\epsilon_{m_{i}}}\right)\label{eq:prob_eq3}
\end{align}
where 
\begin{align*}
&c_i =\sqrt{\frac{\rho (\hat{v}_i + 2)\log (\psi T\epsilon_{m_{i}})}{4 z_i}} \text{ and } \\
&\bar{c}_i=\sqrt{\dfrac{\rho (\sigma_{i}^{2}+\sqrt{\epsilon_{m_{i}}} + 2)\log(\psi T\epsilon_{m_{i}})}{4 z_i}}.
\end{align*}

Note that, substituting $ z_i \geq n_{m_i} \geq \frac{\log{(\psi T\epsilon_{m_{i}})}}{2\epsilon_{m_{i}}}$, $\bar{c}_i$ can be simplified to obtain,
\begin{align}
\bar{c}_i
\leq \sqrt{\dfrac{\rho\epsilon_{m_{i}}(\sigma_{i}^{2}+\sqrt{\epsilon_{m_{i}}} + 2)}{2}}\leq \sqrt{ \epsilon_{m_{i}}}.
\label{si_bar_equn}
\end{align}
%
The first term in the LHS of (\ref{eq:prob_eq3}) can be bounded using the Bernstein inequality as below:
\begin{align}
&\mathbb{P}\left( \hat{r}_{i} > r_{i}+ \bar{c}_i\right)\nonumber 
\le \exp\left(- \dfrac{(\bar{c}_i)^2 z_{i}}{2\sigma_i^2 + \frac{2}{3}\bar{c}_i} \right)\nonumber 
%%%%%%%%%%%%%%%
\\
& \overset{(a)}{\le} \exp\left(- \rho \left(\dfrac{3\sigma_{i}^{2}+3\sqrt{\epsilon_{m_{i}}} + 6}{6\sigma_i^2 + 2\sqrt{\epsilon_{m_i}}} \right)\log(\psi  T\epsilon_{m_{i}}\right)\nonumber \\
%%%%%%%%%%%%%%%
% &\le \exp\left(- \rho (\sigma_{i}^{2}+\sqrt{\epsilon_{m_{i}}} + 2)\log(\psi  T\epsilon_{m_{i}})\right)\nonumber \\
%%%%%%%%%%%%%%%
& \overset{(b)}{\leq} \exp\left(- \rho \log(\psi  T\epsilon_{m_{i}})\right) 
%%%%%%%%%%%%%%%
\le \dfrac{1}{(\psi  T\epsilon_{m_{i}})^{\frac{3\rho}{2}}}
\label{lhs1_equn}
\end{align}
where, $(a)$ is obtained by substituting equation \ref{si_bar_equn} and $(b)$ occurs because for all $\sigma_{i}^2 \in [0,\frac{1}{4}]$, $\left(\frac{3\sigma_{i}^{2}+3\sqrt{\epsilon_{m_{i}}} + 6}{6\sigma_i^2 + 2\sqrt{\epsilon_{m_i}}}\right) \geq \frac{3}{2}$ .

 
The second term in the LHS of (\ref{eq:prob_eq3}) can be simplified as follows:
\begin{align}
&\mathbb{P}\bigg\lbrace \hat{v}_{i}\geq \sigma_{i}^{2}+\sqrt{\epsilon_{m_{i}}}\bigg\rbrace\nonumber\\
%%%%%%%%%%%%%%%%%%
&\leq \mathbb{P}\bigg\lbrace \dfrac{1}{n_{i}}\sum_{t=1}^{n_{i}}(X_{i,t}-r_{i})^{2}-(\hat{r}_{i}-r_{i})^{2}\geq \sigma_{i}^{2}+\sqrt{\epsilon_{m_{i}}}\bigg\rbrace\nonumber\\
%%%%%%%%%%%%%%%%%%
&\leq \mathbb{P}\bigg\lbrace \dfrac{\sum_{t=1}^{n_{i}}(X_{i,t}-r_{i})^{2}}{n_{i}}\geq \sigma_{i}^{2}+\sqrt{\epsilon_{m_{i}}} \bigg\rbrace\nonumber\\
%%%%%%%%%%%%%%%%%%
&\overset{(a)}{\leq} \mathbb{P}\bigg\lbrace \dfrac{\sum_{t=1}^{n_{i}}(X_{i,t}-r_{i})^{2}}{n_{i}}\geq \sigma_{i}^{2} + \bar{c}_i\bigg\rbrace \nonumber\\
%%%%%%%%%%%%%%%%%%
&\overset{(b)}{\leq} \exp\left(- \rho \left(\dfrac{3\sigma_{i}^{2}+3\sqrt{\epsilon_{m_{i}}} + 6}{6\sigma_i^2 + 2\sqrt{\epsilon_{m_i}}} \right)\log(\psi  T\epsilon_{m_{i}})\right)
%%%%%%%%%%%%%%%%%
\le \dfrac{1}{(\psi  T\epsilon_{m_{i}})^{\frac{3\rho}{2}}}
\label{lhs2_equn}
\end{align}
where inequality $(a)$ is obtained using (\ref{si_bar_equn}), while $(b)$ follows from the Bernstein inequality. 

Thus, using (\ref{lhs1_equn}) and (\ref{lhs2_equn}) in (\ref{eq:prob_eq3}) we obtain $\mathbb{P}(\hat{r}_{i}> r_{i} + c_{i})\le \dfrac{2}{(\psi  T\epsilon_{m_{i}})^{\frac{3\rho}{2}}}$.
\hfill $\blacksquare$	
\end{proof}


\subsection{Proof of Lemma \ref{proofTheorem:Lemma:4}}
\label{App:Lemma:4}
\begin{customlem}{4}
%\label{proofTheorem:Lemma:4}
If $m_i = min\lbrace m|\sqrt{4\epsilon_{m} } < \dfrac{\Delta_i}{4} \rbrace $, $\psi=\frac{T}{ K^2}$, $\rho=\frac{1}{2}$,  $c_{i} =\sqrt{\dfrac{\rho(\hat{v}_i + 2)\log (\psi T\epsilon_{m_{i}})}{4 z_{i}}}$ and $n_{m_i}=\dfrac{\log{(\psi T\epsilon_{m_{i}}^{2})}}{2\epsilon_{m_{i}}}$ then in the $m_i$-th round, 
%\begin{align*}
%\Pb\lbrace c^{*} > c_i \rbrace \leq \dfrac{32K\log T}{(\psi T)^{3\rho}}\sum_{m=0}^{m_i}\dfrac{1}{\epsilon_{m_i}^{3\rho + 1}}. 
%\end{align*}
\begin{align*}
\Pb\lbrace c^{*} > c_i \rbrace  \leq \dfrac{182 K^4}{T^{\frac{5}{4}}\sqrt{\epsilon_{m_i}}}.
\end{align*}
\end{customlem}

\begin{proof}
From the definition of $c_i$ we know that $c_i\propto \frac{1}{z_i}$ as $\psi$ and $T$ are constants. Therefore in the $m_i$-th round,
\begin{align*}
&\Pb\lbrace c^{*} > c_i \rbrace
%%%%%%%%%%%%%%%%%%%%%%%%%%%%%%%%%%%%%
\leq  \Pb\lbrace  z^* < z_i  \rbrace \\
%%%%%%%%%%%%%%%%%%%%%%%%%%%%%%%%%%%%%
&\leq \sum_{m=0}^{m_i}\sum_{z^* =1}^{n_{m}}\sum_{z_i =1}^{n_{m}}\bigg(\Pb\lbrace \hat{r}^* < r^* - c^{*}\rbrace + \Pb\lbrace \hat{r}_i > r_i + c_i\rbrace\bigg)
\end{align*}

%Again, we can show that
%\begin{align*}
%&\sum_{m=0}^{m_i}\sum_{z^* =1}^{n_{m_i}}\sum_{z_i =1}^{n_{m_i}}\mathbb{I}_{i}\cup\mathbb{I}_{*} \leq \sum_{m=0}^{m_i}|B_{m_i}|n_{m_i} \\
%%%%%%%%%%%%%%%%%%%%%%%%%%%%%%%%%%%%%
%&\leq \sum_{m=0}^{m_i}\dfrac{4K}{(\psi T \epsilon_{m_i})^{\frac{3\rho}{2}}}\dfrac{\log{(\psi T\epsilon_{m_{i}}^{2})}}{2\epsilon_{m_{i}}} 
%%\leq 
%%%%%%%%%%%%%%%%%%%%%%%%%%%%%%%%%%%%%%
%\overset{(a)}{\leq} \sum_{m=0}^{m_i}\dfrac{4K}{(\psi T \epsilon_{m_i})^{\frac{3\rho}{2}}}\dfrac{2\log(\frac{T}{K})}{2\epsilon_{m_i}}
%\end{align*}
%
%Here, in $(a)$ happens by substituting the value of $\psi$ and considering $\epsilon_{m_i}\in [\sqrt{\frac{e}{T}},1]$. 

Now, applying Bernstein inequality and following the same way as in Lemma \ref{proofTheorem:Lemma:3} we can show that,
\begin{align*}
&\Pb\lbrace \hat{r}^* < r^* - c^{*}\rbrace \leq \exp(- \frac{(c^{*})^2}{2\sigma_*^2 + \frac{2 c^{*}}{3}} z^*)\leq \frac{4}{(\psi T\epsilon_{m_i})^{\frac{3\rho}{2}}} \\ 
%%%%%%%%%%%%%%%%%%%%%%%%%%%%%%%%%%%
&\Pb\lbrace \hat{r}_i > r_i + c_i\rbrace \leq \exp(- \frac{(c_{i})^2}{2\sigma_i^2 + \frac{2 c_{i}}{3}} z_i)\leq \frac{4}{(\psi T\epsilon_{m_i})^{\frac{3\rho}{2}}}
\end{align*}

Hence, summing everything up, 
\begin{align*}
&\Pb\lbrace c^{*} > c_i \rbrace \\
%%%%%%%%%%%%%%%%%%%%%%%%%%%%%%%%
&\leq \sum_{m=0}^{m_i}\sum_{z^* =1}^{n_{m}}\sum_{z_i =1}^{n_{m}}\bigg(\Pb\lbrace \hat{r}^* < r^* - c^{*}\rbrace + \Pb\lbrace \hat{r}_i > r_i + c_i\rbrace\bigg)\\
%%%%%%%%%%%%%%%%%%%%%%%%%%%%%%%%
&\overset{(a)}{\leq} \sum_{m=0}^{m_i}|B_{m}|n_{m}\bigg(\Pb\lbrace \hat{r}^* < r^* - c^{*}\rbrace + \Pb\lbrace \hat{r}_i > r_i + c_i\rbrace\bigg)\\
%%%%%%%%%%%%%%%%%%%%%%%%%%%%%%%%
&\overset{(b)}{\leq} \sum_{m=0}^{m_i}\dfrac{4K}{(\psi T \epsilon_{m_i})^{\frac{3\rho}{2}}}\dfrac{\log{(\psi T\epsilon_{m}^{2})}}{2\epsilon_{m}}\times 
\\
&\bigg(\Pb\lbrace \hat{r}^* < r^* - c^{*}\rbrace + \Pb\lbrace \hat{r}_i > r_i + c_i\rbrace\bigg)\\
%%%%%%%%%%%%%%%%%%%%%%%%%%%%%%%%
&\overset{(c)}{\leq} \sum_{m=0}^{m_i}\dfrac{4K}{(\psi T \epsilon_{m})^{\frac{3\rho}{2}}}\dfrac{\log(T)}{\epsilon_{m}}\bigg[\frac{4}{(\psi T\epsilon_{m})^{\frac{3\rho}{2}}} + \frac{4}{(\psi T\epsilon_{m})^{\frac{3\rho}{2}}}  \bigg]\\
%%%%%%%%%%%%%%%%%%%%%%%%%%%%%%%%
&\leq \sum_{m=0}^{m_i}\dfrac{32K\log T}{(\psi T\epsilon_{m})^{3\rho}\epsilon_{m}} \leq 
\dfrac{32K\log T}{(\psi T)^{3\rho}}\sum_{m=0}^{m_i}\dfrac{1}{\epsilon_{m}^{3\rho + 1}} \\
%%%%%%%%%%%%%%%%%%%%%%%%%%%%%%%%
&\overset{(d)}{\leq} \sum_{m=0}^{m_i}\dfrac{32K\log T}{(\psi T)^{3\rho}}\left(\sum_{m=0}^{m_i}\dfrac{1}{\epsilon_{m}}\right)^{3\rho + 1}\\
%%%%%%%%%%%%%%%%%%%%%%%%%%%%%%%%
&\overset{(e)}{\leq} 
\dfrac{32 K\log T}{(\frac{T^2}{K^2})^{\frac{3}{2}}}\bigg[\left( 1 + \dfrac{2(2^{ \frac{1}{2}\log_{2} \frac{T}{e}}-1)}{2-1} \right)^{\frac{5}{2}}\bigg] \\
%%%%%%%%%%%%%%%%%%%%%%%%%%%%%%%%
&\leq \dfrac{182 K^4 T^{\frac{5}{4}}\log T}{T^3} \overset{(f)}{\leq} \dfrac{182 K^4}{T^{\frac{5}{4}}} \overset{(g)}{\leq} \dfrac{182 K^4}{T^{\frac{5}{4}}\sqrt{\epsilon_{m_i}}}
\end{align*}

where, $(a)$ comes from the total pulls allocated for all $i\in B_m$ till the $m$-th round, in $(b)$ the arm count $|B_m|$ can be bounded by using equation $(\ref{eq:arm:elim:c1})$ and then we substitute the value of $n_{m}$, $(c)$ happens by substituting the value of $\psi$ and considering $\epsilon_{m}\in [\sqrt{\frac{e}{T}},1]$, $(d)$ follows as $\frac{1}{\epsilon_{m}}\geq 1,\forall m $, in $(e)$ we use the standard geometric progression formula and then we substitute the values of $\rho$ and $\psi$, $(f)$ follows from the inequality $\log T \leq \sqrt{T}$ and $(g)$ is valid for any $\epsilon_{m_i}\in[\sqrt{\frac{e}{T}},1]$. 

\hfill $\blacksquare$	
\end{proof}


\subsection{Proof of Lemma \ref{proofTheorem:Lemma:5}}
\label{App:Lemma:5}
\begin{customlem}{5}
%\label{proofTheorem:Lemma:5}
If $m_i = min\lbrace m|\sqrt{4\epsilon_{m} } < \dfrac{\Delta_i}{4} \rbrace $, $\psi=\frac{T}{ K^2}$, $\rho=\frac{1}{2}$, $c_{i} =\sqrt{\frac{\rho (\hat{v}_i + 2)\log (\psi T\epsilon_{m_{i}})}{4 z_i}}$ and $n_{m_i}=\dfrac{\log{(\psi T\epsilon_{m_{i}}^{2})}}{2\epsilon_{m_{i}}}$ then in the $m_i$-th round, 
%\begin{align*}
%\Pb\lbrace z_i < n_{m_i} \rbrace \leq \dfrac{32K\log T}{(\psi T)^{3\rho}}\sum_{m=0}^{m_i}\dfrac{1}{\epsilon_{m_i}^{3\rho + 1}}.
%\end{align*}
\begin{align*}
\Pb\lbrace z_i < n_{m_i} \rbrace \leq \dfrac{182 K^4}{T^{\frac{5}{4}}\sqrt{\epsilon_{m_i}}}.
\end{align*}
\end{customlem}

\begin{proof}
Following a similar argument as in Lemma \ref{proofTheorem:Lemma:4}, we can show that in the $m_i$-th round,
\begin{align*}
&\Pb\lbrace z_i < n_{m_i} \rbrace \\
%%%%%%%%%%%%%%%%%%%%%%%%%%%%%%%%%%%%
&\leq \sum_{m=0}^{m_i}\sum_{z_i =1}^{n_{m}}\sum_{z^* =1}^{n_{m}}\bigg(\Pb\lbrace \hat{r}^* > r^* - c^{*}\rbrace + \Pb\lbrace \hat{r}_i < r_i + c_i\rbrace\bigg) \\
%%%%%%%%%%%%%%%%%%%%%%%%%%%%%%%%%%%%
&\leq \dfrac{32K\log T}{(\psi T)^{3\rho}}\sum_{m=0}^{m_i}\dfrac{1}{\epsilon_{m}^{3\rho + 1}}\leq \dfrac{182 K^4}{T^{\frac{5}{4}}\sqrt{\epsilon_{m_i}}}.
\end{align*}
\hfill $\blacksquare$	
\end{proof}

%\subsection{Proof of Lemma 6}
%\label{App:Lemma:6}
%\begin{lemma}
%%\label{proofTheorem:Lemma:5}
%For $T\geq K^{2.4}$, $\epsilon_{m_i}\geq \sqrt{\frac{e}{T}}$, $\psi=\frac{T}{K^2}$ and $\rho=\frac{1}{2}$,  
%\begin{align*}
%\dfrac{8K}{(\psi T \epsilon_{m_i})^{\frac{3\rho}{2}}} \geq \dfrac{K\log T}{(\psi T)^{3\rho}}\sum_{m=0}^{m_i}\dfrac{1}{\epsilon_{m_i}^{3\rho + 1}}
%\end{align*}
%\end{lemma}
%
%\begin{proof}
%%We prove this lemma by contradiction. Suppose,
%%
%%\begin{align}
%%\dfrac{6K}{(\psi T \epsilon_{m_i})^{\frac{3\rho}{2}}} < \dfrac{K\log T}{(\psi T)^{3\rho}}\sum_{m=0}^{m_i}\dfrac{1}{\epsilon_{m_i}^{3\rho + 1}} \label{eq:contra:1}
%%\end{align}
%%
%%Again, we know that 
%%\begin{align*}
%%\dfrac{6K}{(\psi T \epsilon_{m_i})^{\frac{3\rho}{2}}} \overset{(a)}{\leq} \dfrac{6K}{( \frac{T^2}{K^2} \epsilon_{m_i})^{\frac{3}{4}}} \overset{(b)}{\leq} \dfrac{6K}{( \frac{T^2}{K^2} \sqrt{\frac{e}{T}})^{\frac{3}{4}}}
%%< \dfrac{6K^{\frac{5}{2}}}{T^{\frac{9}{8}}}
%%\end{align*}
%%
%%where, in $(a)$ we substitute the values of $\rho$ and $\psi$ and $(b)$ happens because $\epsilon_{m_i} \geq \sqrt{\frac{e}{T}}$. 
%From the conditions stated we can show that,
%
%\begin{align*}
%\dfrac{K\log T}{(\psi T)^{3\rho}}\sum_{m=0}^{m_i}\dfrac{1}{\epsilon_{m_i}^{3\rho + 1}} &\overset{(a)}{\leq} 
%\dfrac{K\log T}{(\frac{T^2}{K^2})^{\frac{3}{2}}}\bigg[\left( 1 + \dfrac{2(2^{ \frac{1}{2}\log_{2} \frac{T}{e}}-1)}{2-1} \right)^{\frac{5}{2}}\bigg] \\
%%%%%%%%%%%%%%%%%%%%%%%%%%%%%%%%%
%&\leq \dfrac{12 K^4 T^{\frac{5}{4}}\log T}{T^3} \overset{(b)}{\leq} \dfrac{12 K^4}{T^{\frac{5}{4}}}
%\end{align*}
%
%where, in $(a)$ we substitute the values of $\rho$ and $\psi$ and $(b)$ follows from the inequality $\log T \leq \sqrt{T}$. 
%%Substituting the values in Equation \ref{eq:contra:1} we get,
%%\begin{align}
%%& \dfrac{6K^{\frac{5}{2}}}{T^{\frac{9}{8}}} < \dfrac{6K^4\log T}{T^{\frac{5}{4}}}\nonumber\\
%%& \dfrac{T^{\frac{11}{8}}}{\log T} < K^{\frac{3}{2}}\nonumber\\
%%& \dfrac{T^{\frac{11}{8}}}{\sqrt{T}} \overset{(a)}{<} K^{1.5}\nonumber\\
%%& K^{2.1} \overset{(b)}{<} K^{1.5} \label{eq:ineq1}
%%\end{align}
%%
%%where, $(a)$ occurs because of the inequality $\log T < \sqrt{T}$ and $(b)$ happens because of the condition that $T\geq K^{2.4}$. But the inequality \ref{eq:ineq1} is not possible for any $K\geq 2$. 
%So clearly we can conclude that,
%\begin{align*}
% \dfrac{32K\log T}{(\psi T)^{3\rho}}\sum_{m=0}^{m_i}\dfrac{1}{\epsilon_{m_i}^{3\rho + 1}} \leq \dfrac{12 K^4}{T^{\frac{5}{4}}}
%\end{align*}
%%\begin{align*}
%%\dfrac{6K}{(\psi T \epsilon_{m_i})^{\frac{3\rho}{2}}} > \dfrac{32K\log T}{(\psi T)^{3\rho}}\sum_{m=0}^{m_i}\dfrac{1}{\epsilon_{m_i}^{3\rho + 1}}
%%\end{align*}
%
%\hfill $\blacksquare$	
%\end{proof}

%\subsection{Proof of Lemma 6}
%\label{App:Lemma:6}
%\begin{lemma}
%For all bounded rewards in $[0,1]$, $\dfrac{\Delta_i}{4} \geq \dfrac{\Delta_i}{4\sigma_i^2 + 4} $.
%\end{lemma}
%
%\begin{proof}
%Since all rewards are bounded in $[0,1]$, we know that $0\leq\sigma_i^2 \leq \frac{1}{4}$. Hence we can show that,
%\begin{align*}
%\dfrac{\Delta_i}{4\sigma_i^2 + 4} &\leq \dfrac{\Delta_i}{4.\frac{1}{4} + 4}\\
%& \leq \dfrac{\Delta_i}{5}
%\end{align*}
% 
%But for all $\Delta_i \in [0,1]$ we know that $\dfrac{\Delta_{i}}{5} \leq \dfrac{\Delta_{i}}{4} $. Hence, 
%$\dfrac{\Delta_i}{4} \geq \dfrac{\Delta_i}{4\sigma_i^2 + 4} $.
%
%\hfill $\blacksquare$	
%\end{proof}



\subsection{Proof of Lemma \ref{proofTheorem:Lemma:6}}
\label{App:Lemma:6}
\begin{customlem}{6}
For two integer constants $c_1$ and $c_2$, if $20 c_1 \leq c_2$ then,
\begin{align*}
c_1 \dfrac{4\sigma_i^2 + 4}{\Delta_i}\log\bigg( \dfrac{T\Delta_i^2}{K}\bigg) \leq c_2 \dfrac{\sigma_i^2}{\Delta_i}\log\bigg( \dfrac{T\Delta_i^2}{K}\bigg).
\end{align*}
 
\end{customlem}

\begin{proof}
We again prove this by contradiction. Suppose, 
\begin{align*}
c_1 \dfrac{4\sigma_i^2 + 4}{\Delta_i}\log\bigg( \dfrac{T\Delta_i^2}{K}\bigg) > c_2 \dfrac{\sigma_i^2}{\Delta_i}\log\bigg( \dfrac{T\Delta_i^2}{K}\bigg).
\end{align*}

Further reducing the above two terms we can show that, 

\begin{align*}
& 4c_1\sigma_i^2 + 4c_1 > c_2\sigma_i^2\\
& \Rightarrow 4c_1.\dfrac{1}{4} + 4c_1 \overset{(a)}{>} \dfrac{c_2}{4}\\
& \Rightarrow 20 c_1 > c_2.
\end{align*}

Here, $(a)$ occurs because $0\leq\sigma_i^2 \leq \frac{1}{4},\forall i\in \A$. But, we already know that $20 c_1 \leq c_2$. Hence, 
\begin{align*}
c_1 \dfrac{4\sigma_i^2 + 4}{\Delta_i}\log\bigg( \dfrac{T\Delta_i^2}{K}\bigg) \leq c_2 \dfrac{\sigma_i^2}{\Delta_i}\log\bigg( \dfrac{T\Delta_i^2}{K}\bigg).
\end{align*}

\hfill $\blacksquare$	
\end{proof}

%\subsection{Proof of Lemma 8}
%\label{App:Lemma:8}
%\begin{lemma}
%If $m_*$ be the round that the optimal arm $*$ gets eliminated, then we can show that the regret is upper bounded by,
%
%\begin{align*}
%\sum_{m_{*}=0}^{max_{j\in \A^{'}}m_{j}}\sum_{i\in \A^{''}:m_{i}>m_{*}}\bigg(\dfrac{388 K}{(\psi  T\epsilon_{m_{*}})^{\frac{3\rho}{2}}} \bigg).T\max_{j\in \A^{''}:m_{j}\geq m_{*}}{\Delta}_{j} \\
%%%%%%%%%%%%%%%%%%%%%%%%%
% \leq\sum_{i\in \A^{'}}\dfrac{C_2^{'} K^{\frac{5}{2}}}{\sqrt{T\Delta_i}} +\sum_{i\in \A^{''}\setminus \A^{'}}\dfrac{C_2^{'} K^{\frac{5}{2}}}{\sqrt{T b}}
%\end{align*}
%
%\end{lemma}
%
%\begin{proof}
%\begin{align*}
%&\sum_{m_{*}=0}^{max_{j\in \A^{'}}m_{j}}\sum_{i\in \A^{''}:m_{i}>m_{*}}\bigg(\dfrac{364 K^4}{(T^{\frac{5}{4}}\sqrt{\epsilon_{m_{*}}})} \bigg).T\max_{j\in \A^{''}:m_{j}\geq m_{*}}{\Delta}_{j}\\
%%%%%%%%%%%%%%%%%%%%%%%%%%%%%
%&\leq\sum_{m_{*}=0}^{max_{j\in \A^{'}}m_{j}}\sum_{i\in \A^{''}:m_{i}>m_{*}}\bigg(\dfrac{364 K^4 \sqrt{4}}{(T^{\frac{5}{4}}\sqrt{\epsilon_{m_{*}}})} \bigg).T.4\sqrt{\epsilon_{m_{*}}}\\
%%%%%%%%%%%%%%%%%%%%%%%%%%%%%
%&\leq\sum_{m_{*}=0}^{max_{j\in \A^{'}}m_{j}}\sum_{i\in \A^{''}:m_{i}>m_{*}}\bigg(\dfrac{C_2 K^4}{T^{\frac{1}{4}}\epsilon_{m_{*}}^{\frac{1}{2}-\frac{1}{2}}} \bigg)\\
%%%%%%%%%%%%%%%%%%%%%%%%%%%%%
%&\leq\sum_{i\in \A^{''}:m_{i}>m_{*}}\sum_{m_{*}=0}^{\min{\lbrace m_{i},m_{b}\rbrace}}\bigg(\dfrac{C_2 K^4}{T^{\frac{1}{4}}} \bigg)\\
%%%%%%%%%%%%%%%%%%%%%%%%%%%%%
%&\leq\sum_{i\in \A^{'}}\bigg(\dfrac{C_2 K^4}{T^{\frac{1}{4}}} \bigg)+\sum_{i\in \A^{''}\setminus \A^{'}}\bigg(\dfrac{C_2 K^4}{T^{\frac{1}{4}}} \bigg)\\
%\end{align*}
%In the above simplification, $C_2$ is an integer constant.
%
%\hfill $\blacksquare$	
%\end{proof}

%\subsection{Proof of Lemma 8}
%\label{App:Lemma:8}
%\begin{lemma}
%If $m_*$ be the round that the optimal arm $*$ gets eliminated, then we can show that the regret is upper bounded by,
%
%\begin{align*}
%\sum_{m_{*}=0}^{max_{j\in \A^{'}}m_{j}}\sum_{i\in \A^{''}:m_{i}>m_{*}}\bigg(\dfrac{388 K}{(\psi  T\epsilon_{m_{*}})^{\frac{3\rho}{2}}} \bigg).T\max_{j\in \A^{''}:m_{j}\geq m_{*}}{\Delta}_{j} \\
%%%%%%%%%%%%%%%%%%%%%%%%%
% \leq\sum_{i\in \A^{'}}\dfrac{C_2^{'} K^{\frac{5}{2}}}{\sqrt{T\Delta_i}} +\sum_{i\in \A^{''}\setminus \A^{'}}\dfrac{C_2^{'} K^{\frac{5}{2}}}{\sqrt{T b}}
%\end{align*}
%
%\end{lemma}
%
%\begin{proof}
%\begin{align*}
%&\sum_{m_{*}=0}^{max_{j\in \A^{'}}m_{j}}\sum_{i\in \A^{''}:m_{i}>m_{*}}\bigg(\dfrac{388 K}{(\psi  T\epsilon_{m_{*}})^{\frac{3\rho}{2}}} \bigg).T\max_{j\in \A^{''}:m_{j}\geq m_{*}}{\Delta}_{j}\\
%%%%%%%%%%%%%%%%%%%%%%%%%%%%%
%&\leq\sum_{m_{*}=0}^{max_{j\in \A^{'}}m_{j}}\sum_{i\in \A^{''}:m_{i}>m_{*}}\bigg(\dfrac{388 K\sqrt{4}}{(\psi  T\epsilon_{m_{*}})^{\frac{3\rho}{2}}} \bigg).T.4\sqrt{\epsilon_{m_{*}}}\\
%%%%%%%%%%%%%%%%%%%%%%%%%%%%%
%&\leq\sum_{m_{*}=0}^{max_{j\in \A^{'}}m_{j}}\sum_{i\in \A^{''}:m_{i}>m_{*}}C_2 K\bigg(\dfrac{T^{1-\frac{3\rho}{2}}}{\psi^{\frac{3\rho}{2}}\epsilon_{m_{*}}^{\frac{3\rho}{2}-\frac{1}{2}}} \bigg)\\
%%%%%%%%%%%%%%%%%%%%%%%%%%%%%
%&\leq\sum_{i\in \A^{''}:m_{i}>m_{*}}\sum_{m_{*}=0}^{\min{\lbrace m_{i},m_{b}\rbrace}}\bigg(\dfrac{C_2 K T^{1-\frac{3\rho}{2}}}{\psi^{\frac{3\rho}{2}}2^{-(\frac{3\rho}{2} -\frac{1}{2})m_{*}}} \bigg)\\
%%%%%%%%%%%%%%%%%%%%%%%%%%%%%
%&\leq\sum_{i\in \A^{'}}\bigg(\dfrac{C_2 K T^{1-\frac{3\rho}{2}}}{\psi^{\frac{3\rho}{2}}2^{-(\frac{3\rho}{2} -\frac{1}{2})m_{*}}} \bigg)+\sum_{i\in \A^{''}\setminus \A^{'}}\bigg(\dfrac{C_2 K T^{1-\frac{3\rho}{2} }}{\psi^{\frac{3\rho}{2}}2^{-(\frac{3\rho}{2} -\frac{1}{2})m_{b}}} \bigg)\\
%%%%%%%%%%%%%%%%%%%%%%%%%%%%%
%&\leq\sum_{i\in \A^{'}}\bigg(\dfrac{C_2 K T^{1-\frac{3\rho}{2}}.2^{\frac{\frac{3\rho}{2}}{2}-\frac{1}{4}}}{\psi^{\frac{3\rho}{2}}\Delta_{i}^{\frac{3\rho}{2} -\frac{1}{2}}} \bigg)+\sum_{i\in \A^{''}\setminus \A^{'}}\bigg(\dfrac{C_2 K T^{1-\frac{3\rho}{2}}.2^{\frac{\frac{3\rho}{2}}{2}-\frac{1}{4}}}{\psi^{\frac{3\rho}{2}}b^{\frac{3\rho}{2} -\frac{1}{2}}} \bigg)\\
%%%%%%%%%%%%%%%%%%%%%%%%%%%%%
%&\leq\sum_{i\in \A^{'}}\bigg(\dfrac{ C_2 K 2^{\frac{\frac{3\rho}{2}}{2}+\frac{19}{4}}.T^{1-\frac{3\rho}{2} } }{\psi^{\rho}\Delta_{i}^{2\frac{3\rho}{2} -1}} \bigg)+\sum_{i\in \A^{''}\setminus \A^{'}}\bigg(\dfrac{C_2 K 2^{\frac{\frac{3\rho}{2}}{2}+\frac{19}{4}}.T^{1-\frac{3\rho}{2}} }{\psi^{\frac{3\rho}{2} }b^{2\frac{3\rho}{2}-1}} \bigg)\\
%%%%%%%%%%%%%%%%%%%%%%%%%%%%%
%&\overset{(a)}{\leq}\sum_{i\in \A^{'}}\bigg(\dfrac{C_2^{'} K .T^{1-\frac{3}{4}}}{(\frac{T}{K^2})^{\frac{3}{4}}\Delta_{i}^{2.\frac{3}{4} -1}} \bigg)+\sum_{i\in \A^{''}\setminus \A^{'}}\bigg(\dfrac{C_2^{'} K T^{1-\frac{3}{4}}}{(\frac{T}{K^2})^{\frac{3}{4}}b^{2.\frac{3}{4}-1}} \bigg)\\
%%%%%%%%%%%%%%%%%%%%%%%%%%%%%
%&\leq\sum_{i\in \A^{'}}\dfrac{C_2^{'} K^{\frac{5}{2}}}{\sqrt{T\Delta_i}} +\sum_{i\in \A^{''}\setminus \A^{'}}\dfrac{C_2^{'} K^{\frac{5}{2}}}{\sqrt{T b}}
%%%%%%%%%%%%%%%%%%%%%%%%%%%%%
%%& = \sum_{i\in \A^{'}}\bigg(\dfrac{ C_{2}(\rho) T^{1-\rho}}{\Delta_{i}^{2\rho-1}} \bigg)+\sum_{i\in \A^{''}\setminus \A^{'}}\bigg(\dfrac{C_{2}(\rho)T^{1-\rho}}{b^{2\rho -1}} \bigg) \text{, where } C_2(x) = \frac{2^{\frac{x}{2}+\frac{19}{4}}}{\psi^{x}}
%\end{align*}
%In the above simplification, $(a)$ is obtained by substituting the values of $\psi$ and $\rho$.
%
%\hfill $\blacksquare$	
%\end{proof}

\subsection{Proof of Corollary \ref{Result:Corollary:1}}
\label{App:Corollary:1}

\begin{customCorollary}{1}(\textbf{\textit{Gap-Independent Bound}})
%\label{Result:Corollary:1}
When the gaps of all the sub-optimal arms are identical, i.e., $\Delta_i =\Delta = \sqrt{\frac{K\log K}{T}}>\sqrt{\frac{e}{T}}, \forall i\in \A$ and $C_3$ being an integer constant, the
regret of EUCBV is upper bounded by the following gap-independent expression:
\begin{align*}
	\E[R_{T}]\leq  \dfrac{C_3 K^5}{T^{\frac{1}{4}}} + 320\sqrt{KT}.
\end{align*}	
\end{customCorollary}


\begin{proof}
\label{Proof:Corollary:1}
From \cite{bubeck2011pure}  we know that the function $x\in [0,1]\mapsto x\exp(-Cx^2)$ is  decreasing on $\left[\frac{1}{\sqrt{2C}},1\right ]$ for any $C>0$. Thus, we take $C=\left\lfloor \frac{T}{e}\right\rfloor$ and choose  $\Delta_{i}=\Delta=\sqrt{\frac{K\log K}{T}}>\sqrt{\frac{e}{T}}$ for all $i$.

First, let us recall the result in Theorem \ref{Result:Theorem:1} below:
\begin{align*}
\E [R_{T}] \leq &\sum\limits_{i\in \A :\Delta_{i} > b}\bigg\lbrace \dfrac{C_0 K^{4}}{T^{\frac{1}{4}}} + \bigg(\Delta_{i}+\dfrac{320\sigma_i^2\log{(\frac{T\Delta_{i}^{2}}{K})}}{\Delta_{i}}\bigg)\bigg \rbrace\\ 
  & +\sum\limits_{i\in \A :0 < \Delta_{i}\leq b} \dfrac{C_2 K^{4}}{T^{\frac{1}{4}}} + \max_{i\in \A :0 < \Delta_{i}\leq b}\Delta_{i}T.
\end{align*}

Now,  with  $\Delta_i =\Delta = \sqrt{\frac{K\log K}{T}}>\sqrt{\frac{e}{T}}$ we obtain,
	\begin{align*}
	&\sum_{i\in \A :\Delta_{i} > b}\dfrac{320\sigma_i^2\log{(\frac{T\Delta_{i}^{2}}{K})}}{\Delta_{i}} \leq  \dfrac{320\sigma_{\max}^2 K\sqrt{T}\log{(T\dfrac{K(\log K)}{T K})}}{\sqrt{K\log K}}\\ 
	&\leq  \dfrac{320\sigma_{\max}^2\sqrt{KT}\log{(\log K)}}{\sqrt{\log K}}
	\overset{(a)}{\leq} 320\sigma_{\max}^2\sqrt{KT} 
	\end{align*}		
	where $(a)$ follows from the identity $\dfrac{\log{(\log K)}}{\sqrt{\log K}}\leq 1$ for $K\geq 2$. 
	
%For the term $\sum\limits_{i\in \A :\Delta_{i} > b}\dfrac{C_0 K^{\frac{5}{2}}}{\sqrt{T\Delta_i}}$ by substituting the value of $\Delta_i=\Delta=\sqrt{\dfrac{K\log K}{T}}$ we get,
%\begin{align*}
%\dfrac{C_0 K^{\frac{5}{2}+1}}{\sqrt{T\Delta_i}} &\leq \dfrac{C_0 K^{\frac{5}{2}}}{\sqrt{T\sqrt{\dfrac{K\log K}{T}}}} \\
%%%%%%%%%%%%%%%%%%%%%%%%%%%
%\leq \dfrac{C_0 K^{\frac{5}{2}+1}}{(KT\log K)^{\frac{1}{4}}} \leq \dfrac{C_0 K^3}{T^{\frac{1}{4}}}
%\end{align*}	
%	
%Similarly for the term $\sum\limits_{i\in \A :0 < \Delta_{i}\leq b} \dfrac{C_2^{'} K^{\frac{5}{2}}}{\sqrt{T\Delta_i}}$ we can show that,
%\begin{align*}
%\sum\limits_{i\in \A :0 < \Delta_{i}\leq b} \dfrac{C_2^{'} K^{\frac{5}{2}}}{\sqrt{T\Delta_i}} &\leq \dfrac{C_2^{'} K^{\frac{5}{2}+1}}{\sqrt{T\sqrt{\dfrac{e}{T}}}} \leq \dfrac{C_2^{'} K^3}{T^{\frac{1}{4}}} 
%\end{align*}

Thus, the total worst case gap-independent bound is given by
	\begin{align*}
	\E[R_{T}] &\overset{(a)}{\leq}  \dfrac{C_3 K^5}{T^{\frac{1}{4}}} + 320\sigma_{\max}^2\sqrt{KT}\\
	&\overset{(b)}{\leq} \dfrac{C_3 K^5}{T^{\frac{1}{4}}} + 320\sqrt{KT}
	\end{align*}	
	
where, in$(a)$, $C_3$ is an integer constant such that $C_3 = C_0 + C_2 $ and $(b)$ occurs because $\sigma_i^2 \in [0,\frac{1}{4}], \forall i\in \A$.

\hfill $\blacksquare$	
\end{proof}

%%%%%%%%%%%%%%%%%%%%%%%%%%%%%%%%%%%%%%%%%%%%%%%%%%%%%%%%%%%%
% Bibliography.

\begin{singlespace}
\bibliographystyle{iitm}
\bibliography{refs}
\end{singlespace}


%%%%%%%%%%%%%%%%%%%%%%%%%%%%%%%%%%%%%%%%%%%%%%%%%%%%%%%%%%%%
% List of papers

\listofpapers

\begin{enumerate}  
\item Authors....  \newblock
 Title...
  \newblock {\em Journal}, Volume,
  Page, (year).
\end{enumerate}  

\end{document}
