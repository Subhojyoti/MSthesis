In this chapter, we propose an actively adaptive upper confidence bound (UCB) algorithm, referred to as Improved Changepoint Detector (ImpCPD), that tries to locate the chnagepoints in the distribution and adapt accordingly to minimize the cumulative regret. Our solution to the piecewise stochastic problem also depends on restarting where we employ ImpCPD which tries to minimize the cumulative regret by balancing exploration-exploitation and a changepoint detector which tries to predict a changepoint with high probability. When the changepoint is detected ImpCPD is again restarted with all the past history of actions and rewards erased. For the changepoint detection we propose a novel algorithm CPDI which works as a subroutine in conjunction with the main ImpCPD algorithm.

%We give a general framework such that any of algorithms that has been discussed in chapter \ref{chap:SMAB} and chapter \ref{chap:EUCBV} can be employed to minimize the cumulative regret whereas 

%, CPD(\ref{alg:CPD2}) and CPD(\ref{alg:CPD3}) based on the three dominating ideas of construction on confidence interval, the union bound involving Chernoff-Hoeffding bound (see Appendix \ref{sec:app:Conc}) as in UCB1 \citep{auer2002finite}, the peeling argument as used in MOSS\citep{audibert2009minimax} and DUCB \citep{garivier2011upper} and finally the Laplace method which has been previously explored in UCB-$\delta$ \citep{abbasi2011improved}.

	Also, in contrast to other existing works we take minimal assumption on the environment considered. The three assumptions taken, that is Assumption\ref{assm:global}, Assumption\ref{assm:space-gap} and Assumption\ref{assm:chg-gap} are the standard assumptions taken by all algorithms which is required for detecting significant changes. Furthermore, among the actively adaptive algorithms, CUSUM-UCB requires only Bernoulli rewards whereas our algorithm is applicable to all sub-Gaussian distribution, EXP3.R is more pessimistic compared to our approach as it employs exponential weighting algorithm that does not fully leverage the piecewise stochastic structure of the environment. 
	
	Empirically, we show that ImpCPD significantly outperforms several algorithms in piecewise stochastic setting. In all the environment considered ImpCPD performs better than all the passively adaptive algorithms and most of the actively adaptive algorithms. 
	
	%Also, our proposed algorithms does not require any additional exploration parameters to be tuned. Further, both of the proposed algorithms are anytime algorithm which does not require horizon $T$ as input. 