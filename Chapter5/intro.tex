In this chapter we look at the Augmented-UCB (AugUCB) algorithm for a fixed-budget version of the thresholding bandit problem (TBP), where the objective is to identify a set of arms whose expected mean is above a threshold. A key feature of AugUCB is that it uses both mean and variance estimates to eliminate arms that have been sufficiently explored; to the best of our knowledge this is the first algorithm to employ such an approach for the considered TBP.  Theoretically, we obtain an upper bound on the loss (probability of mis-classification) incurred by AugUCB. Although UCBEV in literature provides a better guarantee, it is important to emphasize that UCBEV has access to problem complexity (whose computation requires arms' mean and variances), and hence is not realistic in practice; this is in contrast to AugUCB whose implementation does not require any such complexity inputs. We conduct extensive simulation experiments to validate the performance of AugUCB. Through our simulation work, we establish that AugUCB, owing to its utilization of variance estimates, performs significantly better than the state-of-the-art APT, CSAR and other non variance-based algorithms.

The rest of the chapter is organized as follows. We elaborate our contributions in Section~\ref{tbandit:contribution} and  we present the AugUCB algorithm in Section~\ref{tbandit:algorithm}. Our main theoretical result on expected loss is stated in Section~\ref{tbandit:results}. Section~\ref{tbandit:expt} contains numerical simulations on various testbeds to show the performance of AugUCB against state-of-the-art algorithms and finally, we summarize in Section~\ref{tbandit:conclusion}.