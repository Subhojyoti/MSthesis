We proposed the AugUCB algorithm for a fixed-budget, pure-exploration TBP. Our algorithm employs both mean and variance estimates for arm elimination. This, to our knowledge is the first variance-based algorithm for the specific TBP that we have considered. We first prove an upper bound on the expected loss incurred by AugUCB. We then conduct simulation experiments to validate the performance of AugUCB. In comparison with APT, CSAR and other non variance-based algorithms, we find that the performance of AugUCB is significantly better. Further, the performance of AugUCB is comparable with UCBEV (which is also variance-based), although the latter exhibits a slightly better performance.  However, UCBEV is not implementable in practice as it requires computing problem complexity, $H_{\sigma,1}$, while AugUCB (requiring no such inputs) can be easily deployed in real-life scenarios. 

%It would be an interesting future work to design an anytime version of the AugUCB algorithm. 

%Although UCBEV provides a better guarantee, it is important to emphasize that UCBEV has access to the problem complexity, and is hence not realistic in practice. This is in contrast to AugUCB whose implementation does not require any such complexity inputs. 
%Finally, through extensive simulation experiments we have validated the performance of AugUCB.

%From a theoretical viewpoint we conclude the expected loss AugUCB is more than UCBEV (which has access to problem complexity). From the numerical experiments on settings with large number of arms with different mean and variance, we observed that AugUCB outperforms all the non-variance aware algorithms.
%This is also the first paper to apply elimination by variance estimation in the TBP problem by modifying UCB-Improved and CCB algorithms. 
% It would be interesting future research to come up with an anytime version of AugUCB algorithm. This is also the first paper to apply elimination by variance estimation in the TBP problem by modifying UCB-Improved and CCB algorithms. 