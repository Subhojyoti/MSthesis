I am grateful to my primary thesis advisor Dr. Balaraman Ravindran from Computer Science and Engineering Department at IIT Madras. Without his constant guidance, motivation and perseverance with me, none of these works would have been completed. His courses of Introduction to Machine Learning and Introduction to Reinforcement Learning were my first step into the research world. Not only these courses carried a wealth of information on the current state of literature in the respective fields but they used to be hugely competitive which used to constantly motivate me to strive for excellence. Apart from our formal meetings every week, I used to catch him in his office or in the corridors or anywhere in the department whenever I had any doubts and we used to converse on them. One of his greatest influences on me is to inculcate the constant urge to collaborate with as many interested people as possible, both inside and outside of my immediate field of research. This became a huge contributing factor in the research that I conducted for this thesis.

I am also grateful to my co-advisor Dr. Nandan Sudarsanam from Department of Management Studies at IIT Madras. It is from him that I first learned the most important factor of publication, how to properly write research papers in a concise and crisp manner. I fondly remember that how intensely he used to scrutinize my drafts for errors and incoherent ideas. His dedication and trust in my work are evident from this one incident during the ICML 2017 submission, when at 2 am in the night he called me up to correct portions of my draft. Also, his courses in Department of Management Studies are very enjoyable to attend and are very informative. Nobody can make you understand a complex idea in a simple way and yet retain all its subtleties as Nandan sir can and his courses used to reflect this idea.

One of the most important collaborators of mine is Dr. K.P Naveen from Electrical Engineering Department at IIT Tirupati who advised me on a significant portion of this thesis. I first attended one of his courses on Introduction to Probability Theory here at IIT Madras (when he was INSPIRE Faculty member here) and instantly liked the way he taught the complex idea of Probability theory in simple terms. Even though the area of Multi-armed bandits is not his core area, yet we collaborated on this topic which led to multiple publications. I fondly remember his rigor and patience with me while correcting the theoretical portions of my draft, the most important section of any bandit research paper.

I am also grateful to Dr. L.A. Prashanth from Computer Science and Engineering Department at IIT Madras in my initial days of research for advising me on a significant portion of stochastic multi-armed bandits. He is up-to-date with all the research work in this area and my interactions with him always led to new ideas and a broader way to look at a problem. All collaborations do not always lead to publications and I hope that our future interactions will lead to a more fruitful outcome.

While writing this thesis I must also acknowledge the contributions of Dr. Odalric-Ambrym Maillard from INRIA, SequeL Lab at Lille, France where I went for winter internship from September to December 2017. He is one of the most inspiring researchers that I met in my foray into the research world. He was always available at his office, always working on interesting ideas and always interested to listen to any idea that I came up with. The best thing about him that I can fondly relate to is that he never rejected any idea even though they were initially outrageous, but used to guide me slowly towards the reason on why they would fail in the long run. 

I am thankful to the other members of my committee for their constant support and guidance. I am especially thankful to Dr.  Sutanu Chakraborti for his trust on me. I fondly remember that whenever we used to meet we used to converse in Bengali and he used to constantly motivate me to work more. His course on Natural Language Processing is a very enjoyable course and the project on humor generation that I did under his supervision is one of most interesting projects I did in my IIT Madras coursework. I am also thankful to my Dr. Nirav P Bhatt and my committee chair Dr. P Sreenivasa Kumar for being always available for me.


I am thankful towards a host of my colleagues in the RISE Lab at Computer Science and Engineering Department where I spent a significant time while conducting research at IIT Madras. Thanks Patanjali for creating such an environment in the lab, conducive to research. I also thank my first roommate Priyesh for making my first semester so livable on the campus. In the course of time, I made more friends who contributed in making my life at IIT Madras memorable; Shashank, Siddharth, Deepak, Tarun, Madhuri, Ditty, Ahana, without you people life at IIT Madras would have been boring.

Last but not the least I must thank Dr. A.P. Vijay Rengarajan from Electrical Engineering Department at IIT Madras. He was also conducting research for his doctoral studies in the IPCV lab and it his from him that I learned the true meaning of dedication. Every day from 9 am to 6 pm for 365 days a year he was available in the lab, sitting in front of his desktop and working tirelessly. This dedication of him, of course, showed up in all the publications he got but most of all he was always earful for all my concerns on a host of issues that I faced in my research and personal life. Vijay without you life at IIT Madras would not have been same.

Finally, I must acknowledge the influence of Dr. Sven Koenig whom I met on the sidelines of IJCAI 2017 in the "Lunch with  A Fellow" event. That four-hour interaction with him is a memory I cherish because we talked about every possible aspect of a researcher's life and how to conduct good research. In the end, he said to me the three mantras of conducting good research; shed ego, collaborate and learn something new every day.

