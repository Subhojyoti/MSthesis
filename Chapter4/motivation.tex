The above TBP formulation has several applications, for instance, from areas ranging from anomaly detection and classification (see  \citet{locatelli2016optimal}) to industrial application. Particularly in industrial applications, a learners objective is to choose (i.e., keep in operation) all machines whose productivity is above a threshold. The TBP also finds applications in mobile communications (see \citet{audibert2010best})  where the users are to be allocated only those channels whose quality is above an acceptable threshold. Again, some of these problems have been already discussed in Chapter \ref{chap:intro}, Section \ref{motivation} and an interested reader can refer to it. In some cases the TBP problem is more relevant than the variants of $p$-best problem (identifying the best $p$ arms from $K$ given arms). As explained in \citet{locatelli2016optimal}, the $p$-best problem is a "contest" whereas the TBP is an  "exam" and in many domains, one requires the idea of "efficiency" or "correctness" threshold above which one wants to utilize an option rather than simply selecting the $p$-best options.

%where the learner has to keep all those workers active whose productivity is above a particular threshold $\tau$, or allocating channels whose quality is above a threshold for Mobile Communications 
% or in crowd-sourcing while hiring workers the TBP problem 

%
%	1. \emph{Product Selection:} A company wants to introduce a new product in market and there is a clear separation of the test phase from the commercialization phase. In this case the company tries to minimize the loss it might incur in the commercialization phase by testing as much as possible in the test phase. So from the several variants of the product that are in the test phase the learning agent must suggest the product variant(s) that are above a particular threshold $\tau$ at the end of the test phase that have the highest probability of minimizing loss in the commercialization phase. A similar problem has been discussed for single best product variant identification without threshold in \cite{bubeck2011pure}. 
%
%	2. \emph{Mobile Phone Channel Allocation:} Another similar problem as above concerns channel allocation for mobile phone communications (\cite{audibert2009exploration}). Here there is a clear separation between the allocation phase and communication phase whereby in the allocation phase a learning algorithm has to explore as many channels as possible to suggest the best possible set of channel(s) that are above a particular threshold $\tau$. The threshold depends on the subscription level of the customer. With higher subscription the customer is allowed better channel(s) with the $\tau$ set high. Each evaluation of a channel is noisy and the learning algorithm must come up with the best possible suggestion within a very small  number of attempts.
%
%	3. \emph{Anomaly Detection and Classification:} Thresholding bandit can also be used for anomaly detection and classification where we define a cutoff level $\tau$ and for any samples above this cutoff gets classified as an anomaly. For further reading we point the reader to section 3 of \cite{locatelli2016optimal}.
%
%