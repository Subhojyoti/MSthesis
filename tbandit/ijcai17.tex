%%%% ijcai17.tex

\typeout{IJCAI-17 Instructions for Authors}

% These are the instructions for authors for IJCAI-17.
% They are the same as the ones for IJCAI-11 with superficical wording
%   changes only.

\documentclass{article}
% The file ijcai17.sty is the style file for IJCAI-17 (same as ijcai07.sty).
\usepackage{ijcai17}

% Use the postscript times font!
\usepackage{times}

\usepackage{macros}

\usepackage{latexsym} 


\title{Thresholding Bandits with Augmented UCB}
%\author{Author names withheld}

\author{Subhojyoti Mukherjee${}^1$, K. P. Naveen${}^2$, Nandan
Sudarsanam${}^3$, Balaraman Ravindran${}^1$\\
${}^1$Department of Computer Science \& Engineering,\\ ${}^2$Department of Electrical Engineering,
${}^3$Department of Management Studies,\\ Indian Institute of
Technology Madras}


\begin{document}
\maketitle


\vspace*{2mm}
\begin{abstract}
In this paper we propose the Augmented-UCB (AugUCB) algorithm for a fixed-budget version of the thresholding bandit problem (TBP), where the objective is to identify a set of arms whose quality is above a threshold. A key feature of AugUCB is that it uses both mean and variance estimates to eliminate arms that have been sufficiently explored; to the best of our knowledge this is the first algorithm to employ such an approach for the considered TBP.  Theoretically, we obtain an upper bound on the loss (probability of mis-classification) incurred by AugUCB. Although UCBEV in literature provides a better guarantee, it is important to emphasize that UCBEV has access to problem complexity (whose computation requires arms' mean and variances), and hence is not realistic in practice; this is in contrast to AugUCB whose implementation does not require any such complexity inputs. We conduct extensive simulation experiments to validate the performance of AugUCB. Through our simulation work, we establish that AugUCB, owing to its utilization of variance estimates, performs significantly better than the state-of-the-art APT, CSAR and other non variance-based algorithms.
\end{abstract}

%\begin{keywords}
%Multi-Armed Bandit, Regret, Exploration-exploitation, UCB
%\end{keywords}

\section{Introduction}
\label{intro}
In this chapter, we consider the piece-wise stochastic multi-armed bandit problem, an interesting variation of the stochastic multi-armed bandit (SMAB) problem in sequential decision making which was discussed in detail in chapter \ref{chap:SMAB}. In this setting,  a learning algorithm is provided with a set of decisions (or arms) with reward distributions unknown to the learner. The learning proceeds in an iterative fashion, where in each round, the algorithm chooses an arm and receives a stochastic reward that is drawn from a distribution specific to the arm selected. There exist a finite number of changepoints such that the reward distribution of arms changes at those changepoints. Given the goal of maximizing the cumulative reward, the learner faces the \textit{exploration-exploitation-changepoint} dilemma, as opposed to the simple \textit{exploration-exploitation} dilemma, i.e., in each round should the algorithm select the arm which has the highest observed reward so far (\textit{exploitation}), or should the algorithm choose a new arm to gain more knowledge of the expected reward of the arms and thereby avert a sub-optimal greedy decision (\textit{exploration}) and finally keep track of the \textit{changepoints} and adapt accordingly.

    The rest of the chapter is organized as follows. We first state the notations, definitions, and assumptions required for this setting in section~\ref{psbandit:notations}. Then we define our problem statement in section~\ref{psbandit:probDef} and in section~\ref{psbandit:related} we discuss the related works in this setting. We elaborate our contributions in section~\ref{psbandit:contribution} and in section~\ref{psbandit:algorithm} we present the changepoint detection algorithms. Section~\ref{psbandit:results} contains our main result, Section~\ref{psbandit:expt} contains numerical simulations where we test our proposed algorithms and finally we summarize in section \ref{psbandit:conclusion}. 


%\subsection{Motivation}
%\label{motivation}

The MAB model fits very well in various real-world scenarios that can be modeled as sequential decision-making problems. Some of which are mentioned as follows:-
\begin{enumerate}
\item \emph{Online Shop Domain:} In the online shop domain \citep{ghavamzadeh2015bayesian}, a retailer aims to maximize profit by sequentially suggesting products to online shopping customers. In this scenario, at every timestep, the retailer displays an item to a customer from a pool of items which has the highest probability of being selected by the customer. The episode ends when the customer selects or does not select a product (which will be considered as a loss to the retailer). This feedback is incorporated by the learner as a feedback from the environment and it modifies its policy for the next suggestion. This process is repeated till a pre-specified number of times with the retailer gathering valuable information regarding the customer from this behaviour and modifying its policy to display other items to different customers.
\item \emph{Medical Treatment Design:} Another interesting domain that MAB model was first studied was for the medical treatment design \citep{thompson1933likelihood},\citep{thompson1935theory}. Here at every timestep, the learner chooses to administer one out of several treatments sequentially on a stream of patients who are suffering from the same ailment (say). Let's also assume that there is a single treatment which will be able to alleviate the patients from their disease. Here, the episode ends when the patient responds well or does not respond well to the treatment whereby the learner modifies its policy for the suggestion to the next patient. The goal of the learner is to quickly converge on the best treatment so that whenever a new patient comes with the same ailment, the learner can suggest the best treatment which can relive the patient of its ailment with a high probability.
\item \emph{Financial Portfolio Management:} In financial portfolio management MAB models can also be used. Here, the learner is faced with the choice of selecting the most profitable stock option out of several stock options. The simplest strategy where we can employ a bandit model is this; at the start of every trading session the learner suggests a stock to purchase worth Re $1$, while at the closing of the trading session it sells off the stock to witness its value after a day's trading. The  profit recorded is treated as the reward revealed by the environment and the learner modifies its policy for the next day. Let's assume that no new stock options are being introduced over the considered time horizon and there is a single best stock option which if selected for perpetuity will always give the best returns. Then, the goal of the learner is reduced to identifying the best stock option as quickly as possible. 
\item \emph{Product Selection:} A company wants to introduce a new product in market and there is a clear separation of the test phase from the commercialization phase. In this case the company tries to minimize the loss it might incur in the commercialization phase by testing as much as possible in the test phase. So from the several variants of the product that are in the test phase the learning learner must suggest the product variant(s) whose qualities are above a particular threshold $\tau$ at the end of the test phase that have the highest probability of minimizing loss in the commercialization phase. A similar problem has been discussed for single best product variant identification without threshold in \citet{bubeck2011pure}. 
\item \emph{Mobile Phone Channel Allocation:} Another similar problem as above concerns channel allocation for mobile phone communications \citep{audibert2009exploration}. Here there is a clear separation between the allocation phase and communication phase whereby in the allocation phase a learner has to explore as many channels as possible to suggest the best possible set of channel(s) whose qualities are above a particular threshold $\tau$. The threshold may depend on the subscription level of the customer such that with higher subscription the customer is allowed better channel(s) with the $\tau$ set high. Each evaluation of a channel is noisy and the learning algorithm must come up with the best possible set of suggestions within a very small  number of attempts.
\item \emph{Anomaly Detection and Classification:} MABs can also be used for anomaly detection where the goal is to seek out extreme values in the data. Anomalies may not always be naturally concentrated which was shown in  \citet{steinwart2005classification}. To implement a MAB model the best possible way is to define a cut-off level $\tau$ and classify the samples above this level $\tau$ as anomalous along with a tolerance factor which gives it a degree of flexibility. Such an approach has already been mentioned in \citet{streeter2006selecting} and further studied in \citet{locatelli2016optimal}.
\end{enumerate}


%	In all the above examples the MAB model performs well mainly because all of them suffer from \textit{exploration-exploitation dilemma}. This is characterized by action-selection choice faced by the learner where it must decide whether to stay with the action yielding highest reward till now or to explore newer actions which might be more profitable in the long run. MAB's are suited for such scenarios because 
%\begin{enumerate}
%\item They are easy to implement.
%\item The switch between exploration and exploitation is more well defined theoretically.
%\item They perform well empirically.
%\end{enumerate}

\subsection{Related Work}
\label{prevRes}
A significant amount of literature is available on the stochastic MAB setting with respect to minimizing the cumulative regret. Chapter \ref{chap:SMAB} and \ref{chap:EUCBV} deals with that. In this work, we are particularly interested in \emph{pure-exploration MABs},  where the focus is primarily on simple regret rather than the cumulative regret. The relationship between cumulative regret and simple regret is proved in \citet{bubeck2011pure} where the authors prove that minimizing the simple regret necessarily results in maximizing the cumulative regret.
The pure exploration problem has been explored  mainly in the following two settings:
    
\subsection{Fixed Budget setting} 

Here the learning algorithm has to suggest the best arm(s) within a fixed budget or time-horizon $T$, that is given as an input. The objective is to maximize the probability of returning the best arm(s).  This is the scenario we consider in this chapter. Some of the important algorithms used in pure exploration setting are discussed in the next part.

\subsubsection{UCB-Exploration Algorithm}

One of the first algorithms proposed for the fixed budget setting is the UCB-Exploration (UCBE) algorithm in \citet{audibert2010best} used for identifying a single best arm. This is shown in algorithm \ref{alg:ucbe}.


\begin{algorithm}[!ht]
\caption{UCBE}
\label{alg:ucbe}
\begin{algorithmic}[1]
\State \textbf{Input: } The budget $T$, exploration parameter $a$
\State Pull each arm once
\For{$t=K+1,..., T$}
\State Pull the arm such that $\argmax_{i\in \A}\bigg\lbrace\hat{r}_{i} + \sqrt{\dfrac{a}{z_i}}\bigg\rbrace$, where $a = \dfrac{25(T-K)}{36 H_1}$ and $H_1 = \sum_{i=1}^{K}\dfrac{1}{\Delta_i^2}$.
\State $t:=t+1 $
\EndFor
\end{algorithmic}
\end{algorithm}

This algorithm is quite similar to the UCB1 algorithm discussed in \citet{auer2002finite} (see algorithm \ref{alg:ucb1}). The major difference between the two algorithms is the confidence interval such that for UCB1 it is designed for minimizing the cumulative regret but for UCBE it is designed for minimizing simple regret. An illustrative table comparing the two is provided in table \ref{table:comp-exp}.

\begin{table}
\caption{Confidence interval and exploration parameters of different algorithms}
\label{table:comp-exp}
\begin{center}
\begin{tabular}{|p{5em}|p{6em}|p{8em}|p{12em}|}
\hline
Algorithm  &  Confidence interval & Exploration Parameter$(a)$ & Remarks \\
\hline
\hline
UCB1       & $\sqrt{\dfrac{a}{z_i}}$ & \begin{align*}a = 2\log (t)\end{align*} & $a$ is logarithmic in $t$ to minimize cumulative regret. This achieves a balance between exploration and exploitation. Hence, the cumulative regret grows logarithmically with $t$. \\%\midrule
\hline
\hline
UCBE       & $\sqrt{\dfrac{a}{z_i}}$ & \begin{align*}a = \dfrac{25(T-K)}{36 H_1},\\ \text{ where }H_1 = \sum_{i=1}^{K}\dfrac{1}{\Delta_i^2}\end{align*} & $a$ is linear in $T$ to minimize simple regret. Here, the main concern is to minimize the probability of error at the end of budget $T$ and conduct as much exploration as possible. Hence, a large $a$ helps to reach exponentially low probability of error.\\\midrule
\end{tabular}
\end{center}
\end{table}


\subsubsection{Successive Reject Algorithm}

The Successive Reject (SR) algorithm has also been proposed in \citet{audibert2010best} and is used for identifying a single best arm. This algorithm is quite different than upper confidence bound based algorithms because it does not rely on any explicit confidence interval to select arm at every timestep. It is shown in algorithm \ref{alg:sr}.


\begin{algorithm}[!th]
\caption{Successive Reject(SR)}
\label{alg:sr}
\begin{algorithmic}[1]
\State \textbf{Input: } The budget $T$
\State \textbf{Initialization: } $n_0 = 0$
\State \textbf{Definition: } $\overline{\log K} = \dfrac{1}{2} + \sum_{i=2}^{K}\dfrac{1}{i}$, $n_m = \dfrac{1}{\bar{\log K}}\dfrac{T-K}{K + 1 - m}$
\For{For each phase $m=1,..., K-1$}
\State For each $i \in B_{m}$, select arm $i$ for $n_m - n_{m-1}$ timesteps.
\State Let $B_{m+1} = B_m\setminus \argmin_{i\in B_m} \hat{r}_i$
(remove one element from $B_m$ , if there
is a tie, select randomly the arm to dismiss among the worst arms).
\State $m:=m+1 $
\EndFor
\State Output the single remaining $i\in B_{m}$.
\end{algorithmic}
\end{algorithm}


From algorithm \ref{alg:sr} we see that SR is a round based algorithm quite similar to UCB-Improved (see algorithm \ref{alg:ucbi}). Similar to UCB-Improved, SR pulls all arms equal number of times in each round and then discards some arm that it deems to be sub-optimal until it is left with a single best arm. However, SR does not have any explicit confidence interval as UCBE, rather the idea of the confidence interval is hidden in the number of pulls allocated to each arm in every round. The number of times each arm is pulled in every round, that is $n_m - n_{m-1}$ timesteps makes sure that the optimal arm is not eliminated in the $m$-th round with high probability. 

%In the combinatorial fixed budget setup \citet{gabillon2011multi} propose the GapE and GapE-V algorithms that suggest, with high probability, the best-$p$ arms at the end of the time budget. 

\subsection{Successive Accept Reject Algorithm for best-p arms setting}

The goal in the best p-arms setting is to identify the top $p$ arms out of $K$ given arms where $p$ is supplied as an input. Several algorithms have been proposed for this setup starting with \citet{gabillon2011multi} where the authors proposed the GapE and GapE-V algorithms that suggest, with high probability, the best-$p$ arms at the end of the time budget. These algorithms are similar to the UCBE type algorithm discussed in the previous section. In this section, we will discuss the Successive Accept Reject strategy shown in algorithm \ref{alg:sar}.


\begin{algorithm}[!th]
\caption{Successive Accept Reject(SAR)}
\label{alg:sar}
\begin{algorithmic}[1]
\State \textbf{Input: } The budget $T$, $p$ 
\State \textbf{Initialization: } $n_0 = 0$
\State \textbf{Definition: } $\overline{\log K} = \dfrac{1}{2} + \sum_{i=2}^{K}\dfrac{1}{i}$, $n_m = \dfrac{1}{\bar{\log K}}\dfrac{T-K}{K + 1 - m}$
\For{For each phase $m=1,..., K-1$}
\State For each $i \in B_{m}$, select arm $i$ for $n_m - n_{m-1}$ timesteps.
\State Let $B'_{m}$ be the set that contains arms in decreasing order of their sample means $\hat{r}_{i},\forall i\in B_{m}$ such that $\hat{r}_{B'_{m}(1),n_m} \geq \hat{r}_{B'_{m}(2),n_m} \geq \ldots \geq \hat{r}_{B'_{m}(K+1-m),n_m}$.
\State Define the new empirical gaps $\forall i\in B'_{m}$ such that for $1\leq r\leq K+1-m$,
\begin{align*}
\Delta_{B'_{m}(r),n_m} =
    \begin{cases}
      \hat{r}_{B'_{m}(r),n_m} - \hat{r}_{B'_{m}(p(m)+1),n_m}, & \text{ if }\ r\leq p(m+1) \\
      \hat{r}_{B'_{m}(p(m)),n_m}  - \hat{r}_{B'_{m}(r),n_m}, & \text{ if }\ r> p(m)+1
    \end{cases}
%\Delta_{B'_{m}(r),n_m} = \bigg\lbrace^{\hat{r}_{B'_{m}(r),n_m} - \hat{r}_{B'_{m}(p(m)+1),n_m} \text{, if $r\leq p(m)+1$}}_{\hat{r}_{\hat{r}_{B'_{m}(p(m)),n_m}  - \hat{r}_{B'_{m}(r),n_m} \text{, if $r> p(m)+1$}}}
\end{align*}
\State Let $i_m\in \argmax_{i\in B'_m} \hat{\Delta}_{i,n_m}$, then $B_{m+1} = B_m\setminus i_m$ (deactivate $i_m$ with ties broken arbitrarily).
\State If $\hat{r}_{i_m,n_m} > \hat{r}_{B'_{m}(p(m)+1)}$, then accept $i_m$ and set $p(m+1)=p(m)-1$, $J_{p - p(m+1)}=i_m$.
\State $m:=m+1 $
\EndFor
\State Output the $p$-accepted arms $J_1,J_2,\ldots, J_p$.
\end{algorithmic}
\end{algorithm}

\citet{bubeck2013multiple} introduced the  Successive Accept Reject (SAR) algorithm, which is an extension of the SR algorithm; SAR is a round based algorithm whereby at the end of each round an arm is either accepted or rejected (based on certain confidence conditions) until the top $p$ arms are suggested at the end of the budget with high probability. Like SR algorithm, the SAR algorithm divides the budget into rounds where in each round it pulls all the arms equal number of times that is for $n_m - n_{m-1}$ timesteps. At the end of the round it orders the arms into a decreasing sequence of their empirical means in $B'_{m}$ and computes for the $p(m)$ empirical best arms in $B'_{m}$ the distance (in terms of their empirical means) to the $(p(m) + 1)$-th empirical best arm in $B'_{m}$. Again for the arms not in the $p(m)$ empirical best arms, SAR computes the distance to the $p(m)$-th empirical best arm. Finally, SAR deactivates the arm $i_m$ that has the maximum empirical distance. If $i_m$ is the empirical best arm in the $m$-th round, then SAR accepts $i_m$ and sets $p(m + 1) = p(m) - 1$ and $J_p - p(m+1) = i_m$, or otherwise SAR rejects  $i_m$. 

    A similar combinatorial setup was explored in \citet{chen2014combinatorial} where the authors propose the Combinatorial Successive Accept Reject (CSAR) algorithm, which is similar in concept to SAR but with a more general setup. 

 
\subsection{Fixed Confidence setting} 

In this setting, the learning algorithm has to suggest the best arm(s) with a fixed confidence (given as input) with as fewer number of attempts as possible. The single best arm identification has been studied in \citet{even2006action}, while for the combinatorial setup \citet{kalyanakrishnan2012pac} have proposed the LUCB algorithm which, on termination, returns  $m$ arms which are at least $\epsilon$ close to the true top-$m$ arms with probability at least $1-\delta$. For a detailed survey of this setup, we refer the reader to \citet{jamieson2014best}. 

\subsection{Unified Setting}
Apart from these two settings some unified approaches has also been suggested in \citet{gabillon2012best} which proposes the algorithms UGapEb and UGapEc which can work in both the fixed budget setting and fixed confidence  setting. 



    
    


\subsection{Our Contribution}
\label{contribution}
In this paper we propose the Efficient-UCB-Variance (henceforth referred to as EUCBV) algorithm for the stochastic MAB setting. EUCBV combines the approach of UCB-Improved, CCB \citep{liu2016modification} and UCBV algorithms. EUCBV, by virtue of taking into account the empirical variance of the arms, exploration parameters  and non-uniform arm selection (as opposed to UCB-Improved), performs significantly better than the existing algorithms in the stochastic MAB setting. EUCBV outperforms UCBV \citep{audibert2009exploration} which also takes into account empirical variance but is less powerful than EUCBV because of the usage of exploration regulatory factor by EUCBV. Also, we carefully design the confidence interval term with the variance estimates along with the pulls allocated to each arm to balance the risk of eliminating the optimal arm against excessive optimism. Theoretically we refine the analysis of \citet{auer2010ucb} and prove that for $T\geq K^{2.4}$ our algorithm is order optimal and achieves a worst case gap-independent regret bound of $O\left( \sqrt{KT} \right)$ which is same as that of MOSS and OCUCB but better than that of UCBV, UCB1 and UCB-Improved. Also, the gap-dependent regret bound of EUCBV is better than UCB1, UCB-Improved and MOSS but is poorer than OCUCB. However, EUCBV's gap-dependent bound matches OCUCB in the worst case scenario when all the gaps are equal. Through our theoretical analysis we establish the exact values of the exploration parameters for the best performance of EUCBV. Our proof technique is highly generic and can be easily extended to other MAB settings. An illustrative table containing the bounds is provided in Table \ref{tab:comp-bds}. 


\begin{table}[t]
\caption{Regret upper bound of different algorithms}
\label{tab:comp-bds}
\begin{center}
\begin{tabular}{|p{5em}|p{12em}|p{7em}|}
\hline
Algorithm  &   \hspace*{1mm}Gap-Dependent & Gap-Independent \\
\hline
\hline
EUCBV		& $O\left( \dfrac{K\sigma_{\max}^{2}\log (\frac{T\Delta^2}{K})}{\Delta}\right)$ & $O\left(\sqrt{KT}\right)$\\
\hline
\hline
UCB1        & $O\left( \dfrac{K\log T}{\Delta} \right)$ & $O\left(\sqrt{KT\log T}\right)$ \\%\midrule
\hline
\hline
UCBV        & $O\left( \dfrac{K\sigma_{\max}^{2}\log T}{\Delta} \right)$ & $O\left(\sqrt{KT\log T}\right)$ \\
\hline
\hline
UCB-Imp 		& $O\left( \dfrac{K\log (T\Delta^2)}{\Delta} \right)$ & $O\left(\sqrt{KT\log K}\right)$ \\%\midrule
\hline
\hline
MOSS	     	& $O\left( \dfrac{K^2\log (T\Delta^2 /K)}{\Delta}\right)$ & $O\left(\sqrt{KT}\right)$\\%\midrule
\hline
\hline
OCUCB     	& $O\left( \dfrac{K\log (T/ H_{i})}{\Delta}\right)$ & $O\left(\sqrt{KT}\right)$\\\midrule
\end{tabular}
\end{center}
%\vspace*{-2em}
\end{table}


Empirically, we show that EUCBV, owing to its estimating the variance of the arms, exploration parameters and non-uniform arm pull, performs significantly better than MOSS, OCUCB, UCB-Improved, UCB1, UCBV, TS, BU, DMED, KLUCB and Median Elimination algorithms. Note that except UCBV, TS, KLUCB and BU (the last three with Gaussian priors) all the aforementioned algorithms do not take into account the empirical variance estimates of the arms. Also, for the optimal performance of TS, KLUCB and BU one has to have the prior knowledge of the type of distribution, but EUCBV requires no such prior knowledge. EUCBV is the first arm-elimination algorithm that takes into account the variance estimates of the arm for minimizing cumulative regret and thereby answers an open question raised by \citet{auer2010ucb}, where the authors conjectured that an UCB-Improved like arm-elimination algorithm can greatly benefit by taking into consideration the variance of the arms. Also, it is the first algorithm that follows the same proof technique of UCB-Improved and achieves a gap-independent regret bound of $O\left( \sqrt{KT} \right)$ thereby, closing the gap of UCB-Improved which achieved a gap-independent regret bound of $O\left( \sqrt{KT\log K} \right)$. 
	
	The rest of the paper is organized as follows. In section~\ref{sec:eucbv} we present the  EUCBV algorithm. Our main theoretical results are stated in section~\ref{sec:results}, while the proofs are established in   section \ref{sec:proofTheorem}. Section~\ref{sec:expt} contains results and discussions from our numerical experiments. We draw our conclusions in section \ref{sec:conc} and section \ref{sec:app} is Appendix (supplementary material).
	
	%discuss about future works. 
	
	%The section \ref{sec:app} containing further proofs is given as supplementary.
	
	
	
%
%\section{Notation Used and Assumptions}
%\label{notation}
%To benefit the reader, we again recall the notations we stated in Chapter \ref{chap:SMAB} and also a few additional notations. $\mathcal{A}$ denotes the set of arms, and $|\mathcal{A}|=K$ is the number of arms in $\mathcal{A}$. For arm $i\in\mathcal{A}$, we use $r_{i}$ to denote the true mean of the distribution from which the rewards are sampled, while $\hat{r}_{i}(t)$ denotes the estimated mean at time $t$. Formally, using $z_i(t)$ to denote the number of times arm $i$ has been pulled until time $t$, we have $\hat{r}_{i}(t)=\frac{1}{z_{i}(t)}\sum_{b=1}^{z_i(t)} X_{i,b}$, where $X_{i,b}$ is the reward sample received when arm $i$ is pulled for the $b$-th time. %
Similarly, we use $\sigma_{i}^{2}$ to denote the true variance of the reward distribution corresponding to arm $i$, while $\hat{v}_{i}(t)$ is the estimated variance, i.e., $\hat{v}_{i}(t)=\frac{1}{z_i(t)}\sum_{b=1}^{z_{i}(t)}(X_{i,b}-\hat{r}_{i})^{2}$. Whenever there is no ambiguity about the underlying  time index $t$, for simplicity we neglect $t$ from the notations and simply use  $\hat{r}_i, \hat{v}_i,$ and $z_i, $ to denote the respective quantities.  Let  $\Delta_{i}=|\tau-r_{i}|$ denote the distance of the true mean from the threshold $\tau$. Also, like Assumption \ref{SMAB:assm:2}, in this setting too we assume that the rewards of all arms are bounded in $[0,1]$.

%%%%%%%%%%
%1-sub-gaussian assumption removed
%%%%%%%%%%
%Along the lines of \cite{locatelli2016optimal} we assume that all the reward distributions are $1$-sub-Gaussian (note that,  $1$-sub-Gaussian includes Gaussian distributions with variance less than $1$, distributions supported on an interval of length less than 2, etc).


%
%
\vspace*{-1em}
\section{Augmented-UCB Algorithm}
\label{algorithm}
\subsection{Proposed Algorithms}

In this section, we first introduce the three changepoint detection algorithms CPD(\ref{alg:CPD1}), CPD(\ref{alg:CPD2}) and CPD(\ref{alg:CPD3}) which uses three different confidence intervals which are carefully constructed using three different approaches.  CPD(\ref{alg:CPD1}) uses a simple union bound using Chernoff-Hoeffding inequality whereas CPD(\ref{alg:CPD2}) uses the peeling trick and CPD(\ref{alg:CPD3}) uses the Laplace method which results in a confidence interval that is valid uniformly over time.

Then we introduce the single expert changepoint detection algorithm in Algorithm \ref{alg:SECPD1} which calls one of this CPD algorithms at every timestep while running an expert bandit algorithm which is restarted once a changepoint is detected. Finally, we introduce in Algorithm \ref{alg:SECPD2} which uses the UCB-Improved \citep{auer2010ucb}  style phases where at the end of each phase the algorithm calls one of the CPD to detect the changepoints. Naturally, SECPD2 results in more speedup than SECPD1 which employs the changepoint detection algorithms at every timestep at the cost of only additional logarithmic regret.  

\begin{algorithm}[!ht]
\caption{Changepoint-Detection-1($t_0$, $t_p$) (CPD1)}
\label{alg:CPD1}
\begin{algorithmic}
\For{$k=1,..,K$}
\For{$t' = t_0 ,..,t_p$}
\State \If{$\hat{\mu}_{k,t_0:t'} + \sqrt{\dfrac{\log(\frac{t_p}{\sqrt{\delta}})}{n_{k,t_0:t'}}} < \hat{\mu}_{k,t'+1:t_p} - \sqrt{\dfrac{\log(\frac{t_p}{\sqrt{\delta}})}{n_{k,t'+1:t_p}}}$}
\State Return True
\Else{$\hat{\mu}_{k,t_0:t'} - \sqrt{\dfrac{\log(\frac{t_p}{\sqrt{\delta}})}{n_{k,t_0:t'}}} > \hat{\mu}_{k,t'+1:t_p} + \sqrt{\dfrac{\log(\frac{t_p}{\sqrt{\delta}})}{n_{k,t'+1:t_p}}}$}
\State Return True
\EndIf
\EndFor
\EndFor
\end{algorithmic}
\end{algorithm}

\begin{algorithm}[!ht]
\caption{Changepoint-Detection-2($t_0$, $t_p$) (CPD2)}
\label{alg:CPD2}
\begin{algorithmic}
\State {\bf Input:} Exploration parameter $B>1$
\For{$k=1,..,K$}
\For{$t' = t_0 ,..,t_p$}
\State \If{$\hat{\mu}_{k,t_0:t'} + \sqrt{\dfrac{B\log(\frac{t_p}{\sqrt{\delta}})}{n_{k,t_0:t'}}} < \hat{\mu}_{k,t'+1:t_p} - \sqrt{\dfrac{B\log(\frac{t_p}{\sqrt{\delta}})}{n_{k,t'+1:t_p}}}$}
\State Return True
\Else{$\hat{\mu}_{k,t_0 , t'} - \sqrt{\dfrac{B\log(\frac{t_p}{\sqrt{\delta}})}{n_{k,t_0:t'}}} > \hat{\mu}_{k,t'+1:t_p} + \sqrt{\dfrac{B\log(\frac{t_p}{\sqrt{\delta}})}{n_{k,t'+1:t_p}}}$}
\State Return True
\EndIf
\EndFor
\EndFor
\end{algorithmic}
\end{algorithm}

\begin{algorithm}[!ht]
\caption{Changepoint-Detection-3($t_0$, $t_p$) (CPD3)}
\label{alg:CPD3}
\begin{algorithmic}
\For{$k=1,..,K$}
\For{$t' = t_0 ,..,t_p$}
\State \If{$\hat{\mu}_{k,t_0:t'} + \sqrt{\frac{(n_{i,t_0:t'}+1)\log(\frac{(n_{i,t_0:t'}+1)}{\sqrt{\delta}})}{2n_{i,t_0:t'}^2}} < \hat{\mu}_{k,t'+1:t_p} - \sqrt{\frac{(n_{i,t'+1:t}+1)\log(\frac{(n_{i,t'+1:t}+1)}{\sqrt{\delta}})}{2n_{i,t'+1:t}^2}}$}
\State Return True
\Else{$\hat{\mu}_{k,t_0 , t'} - \sqrt{\frac{(n_{i,t_0:t'}+1)\log(\frac{(n_{i,t_0:t'}+1)}{\sqrt{\delta}})}{2n_{i,t_0:t'}^2}} > \hat{\mu}_{k,t'+1:t_p} + \sqrt{\frac{(n_{i,t'+1:t}+1)\log(\frac{(n_{i,t'+1:t}+1)}{\sqrt{\delta}})}{2n_{i,t'+1:t}^2}}$}
\State Return True
\EndIf
\EndFor
\EndFor
\end{algorithmic}
\end{algorithm}


%\begin{algorithm}[!ht]
%\caption{Changepoint-Detection-4($t_0$, $t_p$) (CPD4)}
%\label{alg:CPD4}
%\begin{algorithmic}
%\State \textbf{Initialization: } $m=0$, $\epsilon_0 = 1$, $B>1$
%\State \textbf{Definition: } $n_m = \dfrac{B\log(\frac{t}{{\delta}})}{\epsilon_m}$
%\State \If{$tp == n_{m+1}$}
%\For{$k=1,..,K$}
%\For{$t' = t_0 ,..,t_p$}
%\State \If{$\hat{\mu}_{k,t_0:t'} + \sqrt{\dfrac{\log(\frac{t}{{\delta}} )}{2t}} < \hat{\mu}_{k,t':t_p} - \sqrt{\dfrac{\log(\frac{t}{{\delta}} )}{2t}}$}
%\State Return True
%\Else{$\hat{\mu}_{k,t_0 , t'} - \sqrt{\dfrac{\log(\frac{t}{{\delta}} )}{2t}} > \hat{\mu}_{k,t':t_p} + \sqrt{\dfrac{\log(\frac{t}{{\delta}} )}{2t}}$}
%\State Return True
%\EndIf
%\EndFor
%\EndFor
%\EndIf
%\end{algorithmic}
%\end{algorithm}

\begin{algorithm}[!ht]
\caption{Single Expert with change-point detection (SECPD-1)}
\label{alg:SECPD1}
\begin{algorithmic}
\State {\bf Input:} Time horizon $T$; 
\State {\bf Initialization:} $t_0 = 1$, $t_p = 1$, $\M=\lbrace 0\rbrace$;
\State {\bf New Expert:} Start a new expert $f_{t_0}$ and add it to $\M$.
\State Pull each arm once
\State \For{$t=K+1,..,T$}
\State Play the arm $i_{t}$ suggested by $f_{t_0}$, observe reward $x_{i_t,t}$.

\If{ (Changepoint-Detection($t_0$, $t_p$)) == True}\hspace*{4em} /* Call CPD\ref{alg:CPD1} or CPD\ref{alg:CPD2} or CPD\ref{alg:CPD3} */
\State {\bf Reset Parameters:} $t_0=1$, $t_p = 1$, $\M=\lbrace 0\rbrace$\hspace*{4em}/* Changepoint detected, forget old expert  */
\State  Start a new expert $f_{t_0}$ and add it to $\M$\hspace*{4em}/* Add a new expert*/
\Else{
\State Update the local model of $f_{t_0}$
\State $t_p = t_p + 1$}
\EndIf
\EndFor
\end{algorithmic}
\end{algorithm}


\begin{algorithm}[!ht]
\caption{Single Expert with change-point detection (SECPD-2)}
\label{alg:SECPD2}
\begin{algorithmic}

\State {\bf Input:} Time horizon $T$, parameter exploration $\delta$; 
\State {\bf Initialization:} $t_0 = 1$, $t_p = 1$, $\M=\lbrace 0\rbrace$, $m=0$, $\epsilon_0 = 1$, $n_0 = \frac{B\log (\frac{t_p}{\sqrt{\delta}})}{\epsilon_0}$, $p_{0}=Kn_0$;
\State {\bf New Expert:} Start a new expert $f_{t_0}$ and add it to $\M$.
\State Pull each arm once
\State \For{$t=K+1,..,T$}
\State Play the arm $i_{t}$ suggested by $f_{t_0}$, observe reward $x_{i_t,t}$.
\State \State Update the local model of $f_{t_0}$
\State $t_p = t_p + 1$
\If{($t_p \neq p_{m}$)}
\State \State Update the local model of $f_{e}$
\State $t_p = t_p + 1$
\Else{
\If{ (Changepoint-Detection($t_0$, $t_p$)) == True}\hspace*{4em} /* Call CPD\ref{alg:CPD1} or CPD\ref{alg:CPD2} or CPD\ref{alg:CPD3} */
\State {\bf Reset Parameters:} $t_0=1$, $t_p = 1$, $\M=\lbrace 0\rbrace$\hspace*{4em}/* Changepoint detected, forget old expert  */
\State Start a new expert $f_{t_0}$ and add it to $\M$\hspace*{4em}/* Add a new expert */
\State $m=0$, $\epsilon_0 = 1$, $n_0 = \dfrac{K B\log (\frac{t_p}{\sqrt{\delta}})}{\epsilon_0}$;
\Else{
\State {\bf Update Parameters:} $\epsilon_{m+1} = \max{\left\lbrace\sqrt{\frac{e}{t_p}},\frac{\epsilon_m}{B}\right\rbrace}$,  $n_{m+1} = \frac{B\log (\frac{t_p}{\sqrt{\delta}})}{\epsilon_{m+1}}$, $p_{m+1}=t_p + Kn_{m+1}$, $m=m+1$;
}
\EndIf}
\EndIf
%\State\If{($t_p \neq n_m$)}
%\State Update the local model of $f_{t_0}$
%\State $t_p = t_p + 1$
%\EndIf
\EndFor

\end{algorithmic}
\end{algorithm}


%\begin{algorithm}[!ht]
%\caption{Aggregate Expert with change-point detection (AECPD-1)}
%\label{alg:AECPD1}
%\begin{algorithmic}
%
%\State {\bf Input:} Time horizon $T$, parameter exploration $\delta$; 
%\State {\bf Initialization:} $t_0 = 1$, $e = 1$, $t_p = 1$, $\M =\lbrace 0\rbrace$, $m=0$, $\epsilon_0 = 1$, $n_0 = \frac{B\log (\frac{t_p}{\sqrt{\delta}})}{\epsilon_0}$, $p_{0}=Kn_{0}$;
%\State {\bf New Expert:} Start a new expert $f_{e}$ and add it to $\M $.
%\State \For{$t=1,..,T$}
%\State Play the arm $i_{t}$ suggested by $f_{e}$, observe reward $x_{i_t,t}$.
%\If{($t_p \neq p_{m}$)}
%\State \State Update the local model of $f_{e}$
%\State $t_p = t_p + 1$
%\Else{
%\If{ (Changepoint-Detection($t_0$, $t_p$)) == True}\hspace*{4em} /* Call CPD\ref{alg:CPD1} or CPD\ref{alg:CPD2} or CPD\ref{alg:CPD3} */
%\State {\bf Reset Parameters:} $t_0=1$, $t_p = 1$, $\M =\lbrace f_{e}(\text{model})\rbrace$\hspace*{0.0em}/*Changepoint detected, store old expert*/
%\State $e=e+1$
%\State Start a new expert $f_{e}$ and add it to $\M $\hspace*{4em}/* Add a new expert */
%\State $m=0$, $\epsilon_0 = 1$, $n_0 = \dfrac{K B\log (\frac{t_p}{\sqrt{\delta}})}{\epsilon_0}$;
%\ElsIf{(Model-Overlap($\M$,$f_e$)==True)}
%\State $f_e = f_{suggest}$
%\Else{
%\State {\bf Update Parameters:} $\epsilon_{m+1} = \max{\left\lbrace\sqrt{\frac{e}{t_p}},\frac{\epsilon_m}{B}\right\rbrace}$,  $n_{m+1} =\frac{B\log (\frac{t_p}{\sqrt{\delta}})}{\epsilon_{m+1}}$, $p_{m+1}=t_p + Kn_{m+1}$, $m=m+1$;
%}
%\EndIf}
%\EndIf
%\EndFor
%
%\end{algorithmic}
%\end{algorithm}
%
%%\If{(Model-Overlap($\M$,$f_e$)==True)}
%%\State $f_e = f_{suggest}$
%%\State \State Update the local model of $f_{e}$
%%\State $t_p = t_p + 1$
%%\Else{
%
%\begin{algorithm}[!ht]
%\caption{Aggregation of Expert with change-point detection (EAggrCPD)}
%\label{alg:EAggrCPD}
%\begin{algorithmic}
%\State {\bf Input:} Time horizon $T$, parameter exploration $\eta$; 
%\State {\bf Initialization:} $t_0 = 1$, $t_p = 1$, $M_{t_0}=\lbrace 0\rbrace$, $\hat{L}_{i,t_0}=0,\forall i\in\A$;
%\State \For{$t=1,..,T$}
%\State {\bf New Expert:} Start a new expert $f_{t_p}$ and add it to $\M_{t_0:t_p}$.
%\State Set $\hat{L}_{f_{t_p},t_p}=0 , w_{f_{t_p},0} = \dfrac{1}{t_p}\sum_{f_j\in M_{t_0:t_p}}w_{f_j,t_p -1}\exp(-\eta\hat{L}_{f_j,t_0:t_p -1})$
%\State \For{$i=1,..,K$}
%\State $H_{i_{f_j,t_p}}=\sum_{f_j\in \M_{t_0:t_p}}(w_{f_j,t_p -1})\mathbb{I}[i_{f_j,t_p}=i] $
%\EndFor
%
%\State \For{$i=1,..,K$}
%\State $\hat{p}_{i,t_p} = \dfrac{H_i\exp(-\eta\hat{L}_{i,t_p -1})}{\sum_{i\in \A}H_i\exp(-\eta\hat{L}_{i,t_0:t_p -1})}$ 
%\EndFor
%
%\State Play the arm $i_{t}$ according to the probability $\hat{p}_{i,t}$, observe reward $x_{i_t,t}$.
%\State $\hat{L}_{i_t,t_0:t_p} = \hat{L}_{i_t,t_0:t_p} + \dfrac{1-x_{i_t,t_p}}{\hat{p}_{i_t,t_p}}$
%
%
%\If{ (Changepoint-Detection($t_0$, $t_p$)) == True}
%\State {\bf Reset Parameters:} $t_0=1$, $t_p = 1$, $M_{t_0}=\lbrace 0\rbrace$, $\hat{L}_{i,t_0}=0,\forall i\in\A$ \hspace*{4em}/*Changepoint Detected*/
%\Else{
%\State {\bf Update Weights:} \For{$j=1,..,|M_{t_0:t_p}|$}
%\State \If{$i_{t}=i_{f_j,t_p}$}
%\State $\hat{L}_{f_j,t_0:t_p} = \hat{L}_{f_j,t_0:t_p} + \dfrac{1-x_{i_t,t}}{\hat{p}_{i_t,t}}$
%\State $w_{f_j,t_p}=\exp(-\eta\hat{L}_{f_j,t_0:t_p})$
%\State Update the local model of $f_j$
%\State $t_p = t_p + 1$
%\EndIf
%\EndFor}
%\EndIf
%\EndFor
%\end{algorithmic}
%\end{algorithm}



%\subsection{Proposed Algorithm 2 (Aggregation of Experts with CPD)

 

%
\vspace{-2mm}
\section{Theoretical Results}
\label{results}
The main result of this chapter is presented in the following theorem, where we establish a regret upper bound for the proposed EUCBV  algorithm. 
% \subsection*{Main Theorem}

\subsubsection{Gap-Dependent bound of EUCBV}

\begin{theorem}[\textbf{\textit{Gap-Dependent Bound}}]
\label{Result:Theorem:1}
For $T\geq K^{2.4}$, $\rho=\frac{1}{2}$ and $\psi=\frac{T}{K^2}$, the regret $R_T$ for EUCBV satisfies
\begin{align*}
\E [R_{T}] \leq &\sum\limits_{i\in \A :\Delta_{i} > b}\bigg\lbrace \dfrac{C_0 K^{4}}{T^{\frac{1}{4}}} + \bigg(\Delta_{i}+\dfrac{320\sigma_i^2\log{(\frac{T\Delta_{i}^{2}}{K})}}{\Delta_{i}}\bigg)\bigg \rbrace\\ 
  & +\sum\limits_{i\in \A :0 < \Delta_{i}\leq b} \dfrac{C_2 K^{4}}{T^{\frac{1}{4}}} + \max_{i\in \A :0 < \Delta_{i}\leq b}\Delta_{i}T.
\end{align*}

for all $b\geq\sqrt{\frac{e}{T}}$ and $C_0, C_2$ are integer constants. 
\end{theorem}

\begin{proof}[Outline]
The proof is along the lines of the technique in \citet{auer2010ucb}. It comprises of three modules. In the first module we prove the necessary conditions for arm elimination within a specified number of rounds. However, here we require some additional technical results (see Lemma~\ref{proofTheorem:Lemma:1} and Lemma~\ref{proofTheorem:Lemma:2}) to bound the length of the confidence intervals. Further, note that our algorithm combines the variance-estimate based approach of \citet{audibert2009exploration} with the arm-elimination technique of \citet{auer2010ucb} (see Lemma~\ref{proofTheorem:Lemma:3}). Also, while \citet{auer2010ucb} uses Chernoff-Hoeffding bound to derive their regret bound whereas in our work we use  Bernstein inequality (as in \citet{audibert2009exploration}) to obtain the bound. To bound the probability of the non-uniform arm selection before it gets eliminated we use Lemma~\ref{proofTheorem:Lemma:4} and Lemma~\ref{proofTheorem:Lemma:5}. In the second module we bound the number of pulls required if an arm is eliminated on or before a particular number of rounds. Note that the number of pulls allocated in a round $m$ for each arm is $n_{m}:=\bigg\lceil\frac{\log{(\psi T\epsilon_{m}^{2})}}{2\epsilon_{m}}\bigg\rceil$ which is much lower than the number of pulls of each arm required by UCB-Improved or Median-Elimination. We introduce the variance term in the most significant term in the bound by Lemma~\ref{proofTheorem:Lemma:6}. Finally, the third module deals with case of bounding the regret, given that a sub-optimal arm eliminates the optimal arm.
% (see Lemma~\ref{proofTheorem:Lemma:9}). The detailed proof is available in Section \ref{sec:proofTheorem:Theorem1}.
\hfill $\blacksquare$
\end{proof}

\begin{discussion}
\label{Result:discussion1}
From the above result we see that the most significant term in the gap-dependent bound is of the order $O\left(\frac{K\sigma^2_{\max}\log{(T\Delta^{2}/K)}}{\Delta}\right)$ which is better than the existing results for UCB1, UCBV, MOSS and UCB-Improved (see Table~\ref{tab:comp-bds}). Also as like UCBV, this term scales with the variance. \citet{audibert2010best} have defined the term $H_1=\sum_{i=1}^{K}\frac{1}{\Delta_i^2}$, which is referred to as the hardness of a problem; \citet{bubeck2012regret} have conjectured that the gap-dependent regret upper bound can match $O\left(\frac{K\log{(T/H_1)}}{\Delta}\right)$. However, in  \citet{lattimore2015optimally} it is proved that the gap-dependent regret bound cannot be lower than $O\left(\sum_{i=2}^{K}\frac{\log\left(T/H_i\right)}{\Delta_i}\right)$, where $H_i=\sum_{j=1}^{K}\min\left\lbrace \frac{1}{\Delta_i^2},\frac{1}{\Delta_j^2}\right\rbrace$ (OCUCB proposed in \citet{lattimore2015optimally} achieves this bound). Further, in \citet{lattimore2015optimally} it is shown that only in the worst case scenario when all the gaps are equal (so that $H_1=H_{i}=\sum_{i=1}^{K}\frac{1}{\Delta^2}$) the above two bounds match. In the latter scenario, considering $\sigma^2_{\max}\leq \frac{1}{4}$ as all rewards are bounded in $[0,1]$, we see that the gap-dependent bound of EUCBV simplifies to $O\left(\frac{K\log{(T/H_1)}}{\Delta}\right)$, thus matching the gap-dependent bound of OCUCB which is order optimal.
\end{discussion}


\subsubsection{Gap-Independent bound of EUCBV}

In this section, we specialize the result of Theorem \ref{Result:Theorem:1} in Corollary \ref{Result:Corollary:1} to  obtain the gap-independent worst case regret bound. %and Corollary \ref{Result:Corollary:2}.


%\subsection*{Corollary 1}

\begin{corollary}[\textbf{\textit{Gap-Independent Bound}}]
\label{Result:Corollary:1}
When the gaps of all the sub-optimal arms are identical, i.e., $\Delta_i =\Delta = \sqrt{\frac{K\log K}{T}}>\sqrt{\frac{e}{T}}, \forall i\in \A$ and $C_3$ being an integer constant, the
regret of EUCBV is upper bounded by the following gap-independent expression:
\begin{align*}
	\E[R_{T}]\leq  \dfrac{C_3 K^5}{T^{\frac{1}{4}}} + 80\sqrt{KT}.
\end{align*}	
\end{corollary}
	
%\begin{proof}
The proof is given in Appendix \ref{App:Corollary:1}.
%\end{proof}

\begin{discussion}
\label{Result:discussion2}
 In the non-stochastic scenario, \citet{auer2002nonstochastic} showed that the bound on the cumulative regret for EXP-4 is $O\left(\sqrt{KT\log K}\right)$. However, in the stochastic case, UCB1 proposed in \citet{auer2002finite} incurred a regret of order of  $O\left(\sqrt{KT\log T}\right)$ which is clearly improvable. From the above result we see that in the gap-independent bound of EUCBV the most significant term is $O\left(\sqrt{KT}\right)$ which  matches the upper bound of MOSS and OCUCB, and is better than UCB-Improved, UCB1 and UCBV (see Table~\ref{tab:comp-bds}).
\end{discussion}

%%%%%%%%%%%%%%%%%%%%%%%%%%%%%%%%%%
% Shifted to Appendix
%%%%%%%%%%%%%%%%%%%%%%%%%%%%%%%%%%

%\begin{proof}
%\label{Proof:Corollary:1}
%From \cite{bubeck2011pure}  we know that the function $x\in [0,1]\mapsto x\exp(-Cx^2)$ is  decreasing on $\left[\frac{1}{\sqrt{2C}},1\right ]$ for any $C>0$. Thus, we take $C=\left\lfloor \frac{T}{e}\right\rfloor$ and choose  $\Delta_{i}=\Delta=\sqrt{\frac{K\log K}{T}}>\sqrt{\frac{e}{T}}$ for all $i$.
%
%First, let us recall the result in Theorem \ref{Result:Theorem:1} below:
%\begin{align*}
%\E [R_{T}] \leq &\sum\limits_{i\in \A :\Delta_{i} > b}\bigg\lbrace 64 K + \bigg(\Delta_{i}+\dfrac{64\log{(\frac{T\Delta_{i}^{2}}{K})}}{\Delta_{i}}\bigg)\bigg \rbrace\\ 
%  & +\sum\limits_{i\in \A :0 < \Delta_{i}\leq b} 32 K + \max_{i\in \A :0 < \Delta_{i}\leq b}\Delta_{i}T  
%\end{align*}
%
%Now,  with  $\Delta_i =\Delta = \sqrt{\frac{K\log K}{T}}>\sqrt{\frac{e}{T}}$ we obtain,
%	\begin{align*}
%	&\sum_{i\in \A :\Delta_{i} > b}\dfrac{64\log{(\frac{T\Delta_{i}^{2}}{K})}}{\Delta_{i}} \leq  \dfrac{64K\sqrt{T}\log{(T\dfrac{K(\log K)}{T K})}}{\sqrt{K\log K}}\\ 
%	&\leq  \dfrac{64\sqrt{KT}\log{(\log K)}}{\sqrt{\log K}}
%	\overset{(a)}{\leq} 64\sqrt{KT} 
%	\end{align*}		
%	where $(a)$ follows from the identity $\dfrac{\log{(\log K)}}{\sqrt{\log K}}\leq 1$ for $K\geq 2$. Thus, the total worst case gap-independent bound is given by
%	\begin{align*}
%	\E[R_{T}]\leq 96 K^2 + 64\sqrt{KT}.
%	\end{align*}	
%\hfill $\blacksquare$	
%\end{proof}

%
%
%\vspace{-1em}
\vspace{-5mm}
\section{Numerical Experiments}
\label{expt}

In this section, we empirically compare the  performance of AugUCB against APT, UCBE, UCBEV, CSAR and the uniform-allocation (UA) algorithms. A brief note about these algorithms are as follows:
%\begin{itemize}

$\bullet$ APT: This algorithm is from \cite{locatelli2016optimal}; we set $\epsilon=0.05$, which is the margin-of-error within which APT suggests the set of good arms.

$\bullet$ AugUCB: This is the Augmented-UCB algorithm proposed in this paper; as in Theorem \ref{tbandit:Result:Theorem:1} we set $\rho=\frac{1}{3}$.

$\bullet$ UCBE: This is a modification of the algorithm in \cite{audibert2009exploration} (as it was originally proposed for the best arm identification problem); here, we set $a=\frac{T-K}{H_1}$, and at each time-step an arm $i\in\argmin\left\lbrace |\hat{r}_{i} -\tau|-\sqrt{\frac{a}{n_{i}}} \right\rbrace$ is pulled.

$\bullet$ UCBEV: This is a modification of the algorithm in \cite{gabillon2011multi} (proposed for the TopM problem); its implementation is identical to UCBE, but with $a = \frac{T-2K}{H_{\sigma,1}}$. As mentioned earlier, note that UCBEV's implementation would not be possible in real scenarios, as it requires computing the problem complexity $H_{\sigma,1}$. However, for theoretical reasons we show the best performance achievable by UCBEV. In experiment 6 we perform further explorations of UCBEV with alternate settings of $a$.

$\bullet$ CSAR:  Modification of the successive-reject algorithm in \cite{chen2014combinatorial}; here, we reject the arm farthest from $\tau$ after each round. 

$\bullet$ UA: The naive strategy where at each time-step an arm is uniformly sampled from $\mathcal{A}$ (the set of all arms); however, UA is known to be optimal if all arms are equally difficult to classify. 
%\end{itemize}


\noindent
Motivated by the settings considered in \cite{locatelli2016optimal}, 
we design six different experimental scenarios that are obtained by varying the arm means and variances.  Across all experiments consists of $K=100$  arms (indexed $i=1,2,\cdots,100$) of which ${S}_\tau=\{6,7,\cdots,10\}$, where we have fixed $\tau=0.5$. In all the experiments, each algorithm is run independently for $10000$ time-steps. At every time-step, the output set,  $\hat{S}_\tau$, suggested by each algorithm is recorded; the output is counted as an error if $\hat{S}_\tau\ne S_\tau$. In Figure~1, for each experiment, we have reported the percentage of error incurred by the different algorithms as a function of time; Error percentage is obtained by repeating each experiment independently  for $500$ iterations, and then respectively computing the fraction of errors. The details of the considered experiments are as follows.

\textbf{Experiment-1:} The reward distributions are Gaussian with  means  $r_{1:4}=0.2+(0:3)\cdot0.05$, $r_{5}=0.45$, $r_{6}=0.55$, $r_{7:10}=0.65+(0:3)\cdot0.05$ and $r_{11:100}=0.4$. Thus, the means of the first $10$ arms follow an arithmetic progression. The remaining arms have identical means; this setting is chosen because now a significant budget is required in exploring these arms, thus increasing the problem complexity.

 The corresponding variances are $\sigma_{1:5}^{2}=0.5$ and $\sigma_{6:10}^{2}=0.6$, while $\sigma_{11:100}^{2}$ is chosen independently and uniform in the  interval $[0.38,0.42]$; note that, the variances of the arms in $S_\tau$ are higher than those of the other arms. The corresponding  results are shown in Figure \ref{Fig:budgetExpt1}, from where we see that UCBEV, which has access to the problem complexity while being variance-aware, outperforms all other algorithm (including UCBE which also has access to the problem complexity but does not take into account the variances of the arms).  Interestingly, the performance of our AugUCB (without requiring any complexity input) is comparable with UCBEV, while it outperforms UCBE, APT and the other non variance-aware algorithms that we have considered. 	

\begin{figure}[th!]
    \centering
    \begin{tabular}{cc}
    \subfigure[0.32\textwidth][Expt-$1$: Arithmetic Progression (Gaussian)]
    {
    		\pgfplotsset{
		tick label style={font=\Large},
		label style={font=\Large},
		legend style={font=\Large},
		}
        \begin{tikzpicture}[scale=0.7]
      	\begin{axis}[
		xlabel={Time-step},
		ylabel={Error Percentage},
		grid=major,
        %clip mode=individual,grid,grid style={gray!30},
        clip=true,
        %clip mode=individual,grid,grid style={gray!30},
  		legend style={at={(0.5,1.3)},anchor=north, legend columns=3} ]
      	% UCB
		\addplot table{Chapter5/results/budgetTestAP/APT12_comp_subsampled.txt};
		\addplot table{Chapter5/results/budgetTestAP/AugUCBV1_comp_subsampled.txt};
		\addplot table{Chapter5/results/budgetTestAP/UCBEM1_comp_subsampled.txt};
		\addplot table{Chapter5/results/budgetTestAP/UCBEMV1_comp_subsampled.txt};
		\addplot table{Chapter5/results/budgetTestAP/SR1_comp_subsampled.txt};
		\addplot table{Chapter5/results/budgetTestAP/UA1_comp_subsampled.txt};
		
      	\legend{APT,AugUCB,UCBE,UCBEV,CSAR,UA}
      	\end{axis}
      	\end{tikzpicture}
  		\label{Fig:budgetExpt1}
    }
    &
    \subfigure[0.32\textwidth][Expt-$2$: Geometric Progression (Gaussian)]
    {
    	\pgfplotsset{
		tick label style={font=\Large},
		label style={font=\Large},
		legend style={font=\Large},
		}
        \begin{tikzpicture}[scale=0.7]
        \begin{axis}[
		xlabel={Time-step},
		ylabel={Error Percentage},
        %clip mode=individual,grid,grid style={gray!30},
		grid=major,
		clip=true,
  		legend style={at={(0.5,1.3)},anchor=north, legend columns=3} ]
        % UCB
		\addplot table{Chapter5/results/budgetTestGP/APT12_comp_subsampled.txt};
		\addplot table{Chapter5/results/budgetTestGP/AugUCBV1_comp_subsampled.txt};
		\addplot table{Chapter5/results/budgetTestGP/UCBEM1_comp_subsampled.txt};
		\addplot table{Chapter5/results/budgetTestGP/UCBEMV1_comp_subsampled.txt};
		\addplot table{Chapter5/results/budgetTestGP/SR1_comp_subsampled.txt};
		\addplot table{Chapter5/results/budgetTestGP/UA1_comp_subsampled.txt};
        \legend{APT,AugUCB,UCBE,UCBEV,CSAR,UA}
      	\end{axis}
      	\label{Fig:budgetExpt2}
        \end{tikzpicture}
    }
    \end{tabular}
\caption{Performances of the various TBP algorithms in terms of error percentage vs. time-step.}
    \label{fig:budgetExpt1}
\end{figure}

\begin{figure}[th!]
    \centering
	\begin{tabular}{cc}
	\centering
    \subfigure[0.32\textwidth][Expt-$3$: Three Group Setting (Gaussian)]
    {
    		\pgfplotsset{
		tick label style={font=\Large},
		label style={font=\Large},
		legend style={font=\Large},
		}
        \begin{tikzpicture}[scale=0.7]
        \begin{axis}[
		xlabel={Time-step},
		ylabel={Error Percentage},
        %clip mode=individual,grid,grid style={gray!30},
       	grid=major,
       	clip=true,
  		legend style={at={(0.5,1.3)},anchor=north, legend columns=3} ]
      	% UCB
		\addplot table{Chapter5/results/budgetTestGR1/APT1_comp_subsampled.txt};
		\addplot table{Chapter5/results/budgetTestGR1/AugUCB1_comp_subsampled.txt};
		\addplot table{Chapter5/results/budgetTestGR1/UCBEM1_comp_subsampled.txt};
		\addplot table{Chapter5/results/budgetTestGR1/UCBEMV1_comp_subsampled.txt};
		\addplot table{Chapter5/results/budgetTestGR1/SR1_comp_subsampled.txt};
		\addplot table{Chapter5/results/budgetTestGR1/UA1_comp_subsampled.txt};
        \legend{APT,AugUCB,UCBE,UCBEV,CSAR,UA}
      	\end{axis}
      	\end{tikzpicture}
   		\label{Fig:budgetExpt3} 
    }
    &
    \subfigure[0.32\textwidth][Expt-$4$: Two Group Setting (Gaussian) ]
    {
    	\pgfplotsset{
		tick label style={font=\Large},
		label style={font=\Large},
		legend style={font=\Large},
		}
        \begin{tikzpicture}[scale=0.7]
        \begin{axis}[
		xlabel={Time-step},
		ylabel={Error Percentage},
        %clip mode=individual,grid,grid style={gray!30},
		grid=major,
		clip=true,
  		legend style={at={(0.5,1.3)},anchor=north, legend columns=3} ]
        % UCB
		\addplot table{Chapter5/results/budgetTestGR2/APT1_comp_subsampled.txt};
		\addplot table{Chapter5/results/budgetTestGR2/AugUCBV1_comp_subsampled.txt};
		\addplot table{Chapter5/results/budgetTestGR2/UCBEM1_comp_subsampled.txt};
		\addplot table{Chapter5/results/budgetTestGR2/UCBEMV1_comp_subsampled.txt};
		\addplot table{Chapter5/results/budgetTestGR2/SR1_comp_subsampled.txt};
		\addplot table{Chapter5/results/budgetTestGR2/UA1_comp_subsampled.txt};
        \legend{APT,AUgUCB,UCBE,UCBEV,CSAR,UA}
        %\legend{APT,AugUCB,UCBE,UCBEV,CSAR,Unif Alloc}
      	\end{axis}
      	\label{Fig:budgetExpt4}
        \end{tikzpicture}
    }
    \end{tabular}
    \caption{Performances of the various TBP algorithms in terms of error percentage vs. time-step.}
    \label{fig:budgetExpt2}
\end{figure}

\begin{figure}[th!]
    \centering
	\begin{tabular}{cc}
    \subfigure[0.32\textwidth][Expt-$5$: Two Group Setting (Advance) ]
    {
    	\pgfplotsset{
		tick label style={font=\Large},
		label style={font=\Large},
		legend style={font=\Large},
		}
        \begin{tikzpicture}[scale=0.7]
        \begin{axis}[
		xlabel={Time-step},
		ylabel={Error Percentage},
        %clip mode=individual,grid,grid style={gray!30},
		grid=major,
		clip=true,
  		legend style={at={(0.5,1.3)},anchor=north, legend columns=3} ]
        % UCB
		\addplot table{Chapter5/results/budgetTestGR4/APT1_comp_subsampled.txt};
		\addplot table{Chapter5/results/budgetTestGR4/AugUCB1_comp_subsampled.txt};
		\addplot table{Chapter5/results/budgetTestGR4/UCBEM1_comp_subsampled.txt};
		\addplot table{Chapter5/results/budgetTestGR4/UCBEMV1_comp_subsampled.txt};
		\addplot table{Chapter5/results/budgetTestGR4/SR1_comp_subsampled.txt};
		\addplot table{Chapter5/results/budgetTestGR4/UA1_comp_subsampled.txt};
        \legend{APT,AUgUCB,UCBE,UCBEV,CSAR,UA}
      	\end{axis}
      	\label{Fig:budgetExpt5}
        \end{tikzpicture}
    }
    &
    \subfigure[0.32\textwidth][Expt-$6$: Two Group Setting (Advance) ]
    {
    	\pgfplotsset{
		tick label style={font=\Large},
		label style={font=\Large},
		legend style={font=\Large},
		}
        \begin{tikzpicture}[scale=0.7]
        \begin{axis}[
		xlabel={Time-step},
		ylabel={Error Percentage},
        %clip mode=individual,grid,grid style={gray!30},
		grid=major,
		clip=true,
  		legend style={at={(0.5,1.3)},anchor=north, legend columns=2} ]
        % UCB
		\addplot table{Chapter5/results/budgetTestGR3/testUCBEMV1_0.25_comp_subsampled.txt};
		\addplot table{Chapter5/results/budgetTestGR4/AugUCB1_comp_subsampled.txt};
		\addplot table{Chapter5/results/budgetTestGR3/testUCBEMV1_256_comp_subsampled.txt};
		\addplot table{Chapter5/results/budgetTestGR4/UCBEMV1_comp_subsampled.txt};
        \legend{UCBEV($0.25$), AugUCB, UCBEV($256$), UCBEV($1$)}
      	\end{axis}
      	\label{Fig:budgetExpt6}
        \end{tikzpicture}
    }
    \end{tabular}
    \caption{Performances of the various TBP algorithms in terms of error percentage vs. time-step.}
    \label{fig:budgetExpt3}
\end{figure}

	
\textbf{Experiment-2:} We again consider  Gaussian reward distributions. However, here the means of the first $10$ arms constitute a geometric progression. Formally, the reward means are $r_{1:4}=0.4-(0.2)^{1:4}$, $r_{5}=0.45$, $r_{6}=0.55$, $r_{7:10}=0.6+(0.2)^{5-(1:4)}$ and $r_{11:100}=0.4$; the arm variances are as in experiment-$1$. The corresponding results are shown in Figure \ref{Fig:budgetExpt2}.  We again observe AugUCB outperforming the other algorithms, except UCBEV. 
	

\textbf{Experiment-3:} Here, the first
$10$ arms are partitioned into three groups, with all arms in a group being assigned the same mean; the reward distributions are again Gaussian. Specifically, the reward means are $r_{1:3}=0.1$, $r_{4:7}=\lbrace 0.35, 0.45, 0.55, 0.65\rbrace$ and $r_{8:10}=0.9$; as before,  $r_{11:100}=0.4$ and all the variances are as in Experiment-$1$. The results for this scenario are presented in Figure \ref{Fig:budgetExpt3}. The observations are inline with those made in the previous experiments. 


	
\textbf{Experiment-4:} The setting is similar to that considered in Experiment-3, but with the first $10$ arms partitioned into two groups; the respective means are $r_{1:5}=0.45$, $r_{6:10}=0.55$. The corresponding results are shown in Figure \ref{Fig:budgetExpt4}, from where the good performance of AugUCB is again validated.


\textbf{Experiment-5:} This is again the two group setting involving Gaussian reward distributions. The reward means are as in Experiment-4, while the variances are  $\sigma_{1:5}^{2}=0.3$ and $\sigma_{6:10}^{2}=0.8$;  $\sigma_{11:100}^{2}$ are independently and uniformly chosen in the interval $[0.2,0.3]$.  The corresponding results are shown in Figure \ref{Fig:budgetExpt5}.
 We refer to this setup as \emph{Advanced} because here the chosen variance values are such that only  variance-aware algorithms will perform well.Hence, we see that UCBEV performs very well in comparison with the other algorithms. However,  it is interesting to note that the performance of  AugUCB catches-up with UCBEV as the time-step increases, while significantly outperforming the other non-variance aware algorithms.


\textbf{Experiment-6:} We use the same setting as in Experiment-5, but conduct more exploration of UCBEV with different values of the exploration parameter $a$. The corresponding results are shown in Figure \ref{Fig:budgetExpt6}. As studied in \cite{locatelli2016optimal}, we implement UCBEV with $ a_{i} = 4^{i} \frac{T-2K}{H_{\sigma,1}}$ for $i = -1,0,4$. Here, $a_{0}$ corresponds to UCBEV($1$) (in Figure \ref{Fig:budgetExpt6}) which is UCBEV run with the optimal choice of $H_{\sigma ,1}$. For other choices of $a_i$ we see that UCBEV($a_i$) is significantly outperformed by AugUCB. 
	
Finally, note that in all the above experiments, the CSAR algorithm, although performs well initially, quickly exhausts its budget and saturates at a higher error percentage. This is because it pulls all arms equally in each round, with the round lengths being non-adaptive.






%
%\vspace{-1.2em}
\vspace{-1mm}
\section{Conclusion}
\label{conclusion}
In this chapter, we looked at the stochastic multi-armed bandit (SMAB) setting and discussed how it is important in the general reinforcement learning setup. We also looked at the various state-of-the-art algorithms in the literature for the SMAB setting and discussed the advantages and disadvantages of them. The regret bounds that have been proven for the said algorithms have also been discussed at length and their confidence intervals have also been compared against each other. In the next chapter, we provide our solution to this SMAB setting which achieves an almost order-optimal regret bound.

% Acknowledgments---Will not appear in anonymized version
%\acks{We thank a bunch of people.}

%\clearpage
%\newpage
%\bibliographystyle{named}
%\bibliography{ijcai17}

%\clearpage
%\newpage
%\section{Appendix}
%\label{appendix}
%\appendix
\begin{align*}
& H_{1}^{\sigma}=\sum_{i=1}^{K}\frac{\sigma_{i}+\sqrt{\sigma_{i}^{2}+(16/3)\Delta_{i}}}{\Delta_{i}^{2}}\\
& H_{2}^{\sigma}=\min_{i\in \mathcal{A}} i\tilde{\Delta}_{(i)}^{-2} \text{, where } \tilde{\Delta}_{i}^{-2}=\frac{\sigma_{i}+\sqrt{\sigma_{i}^{2}+(16/3)\Delta_{i}}}{\Delta_{i}^{2}}
%& H_{2}^{\sigma}=\min_{i\in \mathcal{A}} i\frac{12\sigma_{(i)}^{2} + \Delta_{(i)}}{12\Delta_{(i)}^{2}}
\end{align*}

We know that $\sigma_{i}\in [0,1], \forall i\in \mathcal{A}$ and $\Delta_{i}\in [0,1], \forall i\in \mathcal{A}$ and so $\sigma_{i}^{2} \leq \sigma_{i}$ and $\sqrt{\Delta_{i}} \geq \Delta_{i}$.

\begin{align*}
(3\Delta_{i}^{2}).\left(\frac{4\sigma_{i}^{2}+\Delta_{i}+4}{12\sigma_{i}^{2}+\Delta_{i}}\right) & >  \left(\frac{12\Delta_{i}^{2}}{12\sigma_{i}^{2}+\Delta_{i}}\right)\\
& > \left(\frac{12\Delta_{i}^{2}}{12\sigma_{i}^{2}+12\Delta_{i}}\right)\\
& > \left(\frac{\Delta_{i}^{2}}{\sigma_{i}+\Delta_{i}}\right)\\
%& > \left(\frac{\Delta_{i}^{2}}{\sigma_{i}+(\sigma_{i}^{2} + (16/3)\Delta_{i})}\right)\\
& > \left(\frac{\Delta_{i}^{2}}{\sigma_{i}+\sqrt{\sigma_{i}^{2} + (16/3)\Delta_{i}}}\right)\\
& > \left(\frac{1}{\min_{i}i\tilde{\Delta}_{i}^{2}}\right)\\
\end{align*}

Now, from \cite{audibert2010best} we know that,
\begin{align*}
\sum_{i=1}^{K}\tilde{\Delta}_{i}^{-2} = \tilde{\Delta}_{(2)}^{-2} + \sum_{i=2}^{K}\frac{1}{i}i\tilde{\Delta}_{(i)}^{-2} &\leq \bar{\log K}\min_{i}i\tilde{\Delta}_{(i)}^{-2}\\
& \leq \log(2K) H_{2}^{\sigma}, \text{ as $\bar{\log K} \leq \log(2K)$}
\end{align*}

So, $H_{2}^{\sigma} \leq H_{1}^{\sigma} \leq \log(2K) H_{2}^{\sigma}$

\textbf{Regarding union bound}\\
\begin{align*}
\Pb\lbrace\xi_1 \cup \xi_2\rbrace &= \Pb\lbrace\xi_1\rbrace + \Pb\lbrace\xi_2\rbrace - \Pb\lbrace\xi_1 \cap \xi_2\rbrace\\
 &\leq \Pb\lbrace\xi_1\rbrace + \Pb\lbrace\xi_2\rbrace
\end{align*}
So,
\begin{align*}
1-\Pb\lbrace\xi_1 \cup \xi_2\rbrace \geq 1-\Pb\lbrace\xi_1\rbrace + \Pb\lbrace\xi_2\rbrace \geq \E[\Ls(T)] \end{align*}

\section{Cumulative Regret Bound}

\begin{theorem}
\label{proofTheorem:Prop:1}
The regret $R_T$ for AugUCB satisfies
\begin{align*}
&\E [R_{T}]\\
& \leq \sum\limits_{i\in A:\Delta_{i} > b}\bigg\lbrace \dfrac{T\Delta_{i}}{( \frac{3}{2} T\Delta_i^{2})^{\psi_{m_i}}}
  + \left( \Delta_i +\dfrac{22\Delta_i\psi_{m_i}\log( \frac{3}{2} T\Delta_i^2)}{ \Delta_i}\right)\\
  & \bigg(\dfrac{4T^{1-\psi_{m_i}}2^{\psi_{m_i}-\frac{1}{2}}}{\Delta_{i}^{\psi_{m_i}-\frac{1}{2}}} \bigg)\bigg \rbrace +\sum_{i\in A: 0 < \Delta_{i} \leq b}\bigg(\dfrac{4T^{1-\psi_{m_i}}2^{\psi_{m_i}-\frac{1}{2}}}{b^{\psi_{m_i}-\frac{1}{2}}} \bigg) \\
  & + \max_{\substack{i\in A: \\ \Delta_{i}\leq b}}\Delta_{i}T, \text{  for all $b\geq\sqrt{\frac{e}{T}}$}. 
\end{align*} 
\end{theorem}


\begin{proof}
\textbf{\textit{Case a: A sub-optimal arm i is not eliminated on or before round $m_{i}$ with $ * \in B_{m_i}$}}
\newline
%%%%%%%%%%%%%
For any arm $i\in A$, if it is eliminated from active set $B_{m_{i}}$ then the below two events have to come true,
\begin{align}
\hat{r}_{i} + s_{i} < \tau - s_{i}, \label{eq:armelim-var-casea1}\\
\hat{r}_{i} - s_{i} > \tau + s_{i}, \label{eq:armelim-var-caseb1}
\end{align}

From Theorem \ref{Result:Theorem:1}, we know that a sub-optimal arm $i\in\mathcal{A}$, the probability that it is not correctly eliminated in the $m_i$-th round (or before) is also bounded by $4\exp\left(- \dfrac{3\rho \psi_{m_i}}{2} \left(\dfrac{\sigma_{i}^{2}+\sqrt{\rho\epsilon_{m_{i}}}+1}{3\sigma_{i}^{2}+\sqrt{\rho \epsilon_{m_{i}}}}\right) \log( T\epsilon_{m_{i}}) \right)$.

Summing over all arms in $\mathcal{A}^{'}$ and trivially bounding the regret for each arm $i\in \mathcal{A}^{'}$,

\begin{align*}
&\sum_{i\in A^{'}}T\Delta_{i}\exp\left(- \dfrac{3\rho \psi_{m_i}}{2} \left(\dfrac{\sigma_{i}^{2}+\sqrt{\rho\epsilon_{m_{i}}}+1}{3\sigma_{i}^{2}+\sqrt{\rho \epsilon_{m_{i}}}}\right) \log( T\epsilon_{m_{i}}) \right)\\
&\overset{(a)}{\le}  \sum_{i\in A^{'}}T\Delta_{i} \exp\left(- \dfrac{3\rho \psi_{m_i}}{2} \log( T\epsilon_{m_{i}}) \right)\\
&\overset{(b)}{\le}  \sum_{i\in A^{'}}T\Delta_{i} \exp\left(-\psi_{m_i} \log( T\epsilon_{m_{i}}) \right)\\
&\overset{(c)}{\le}  \sum_{i\in A^{'}}\dfrac{T\Delta_{i}}{( T\dfrac{\Delta_i^{2}}{\rho})^{\psi_{m_i}}} \leq \sum_{i\in A^{'}}\dfrac{T\Delta_{i}}{( \frac{3}{2} T\Delta_i^{2})^{\psi_{m_i}}}
\end{align*}

In the above formulation, $(a)$ happens because  $\left(\dfrac{\sigma_{i}^{2}+\sqrt{\rho\epsilon_{m_{i}}}+1}{3\sigma_{i}^{2}+\sqrt{\rho \epsilon_{m_{i}}}}\right) \geq 1$. For $(b)$ we put $\rho=\dfrac{2}{3}$ and for $(c)$ we take the help of the inequality $\sqrt{\rho \epsilon_{m_i}} < \Delta_i$ in the $m_i$-th round.

\textbf{\textit{Case b: A sub-optimal arm i is either eliminated on or before round $m_{i}$ or there is no $ * \in B_{m_i}$}}


\textbf{\textit{Case b1: A sub-optimal arm i is in $B_{m_i}$}}


A sub-optimal arm is in $B_{m_i}$ and hence pulled no more than,
\begin{align*}
\left\lceil \dfrac{2\psi_{m_i}\log( T\epsilon_{m_i})}{\epsilon_{m_i}} \right\rceil\leq \left\lceil \dfrac{32\psi_{m_i}\log( \frac{3}{2} T\Delta_i^2)}{\frac{3}{2} \Delta_i^2}\right\rceil
\end{align*}

Hence, the total contribution to the expected regret is,
\begin{align*}
&\sum_{i\in A^{'}} \left\lceil \dfrac{32\Delta_i\psi_{m_i}\log( \frac{3}{2} T\Delta_i^2)}{\frac{3}{2} \Delta_i^2}\right\rceil\\
&\leq \sum_{i\in A^{'}} \left( \Delta_i +\dfrac{22\Delta_i\psi_{m_i}\log( \frac{3}{2} T\Delta_i^2)}{ \Delta_i}\right)
\end{align*}

\textbf{\textit{Case b2: Optimal arm ${*}$ is eliminated by a sub-optimal arm}}

	Firstly, if conditions of Case $a$ holds then the optimal arm ${*}$ will not be eliminated in round $m=m_{*}$ or it will lead to the contradiction that $r_{i}>r^{*}$. In any round $m_{*}$, if the optimal arm ${*}$ gets eliminated then for any round from $1$ to $m_{j}$ all arms ${j}$ such that $m_{j}< m_{*}$ were eliminated according to assumption in Case $a$. Let the arms surviving till $m_{*}$ round be denoted by $A^{'}$. This leaves any arm $a_{b}$ such that $m_{b}\geq m_{*}$ to still survive and eliminate arm ${*}$ in round $m_{*}$. Let such arms that survive ${*}$ belong to $A^{''}$. Also maximal regret per step after eliminating ${*}$ is the maximal $\Delta_{j}$ among the remaining arms ${j}$ with $m_{j}\geq m_{*}$.  Let $m_{b}=\min\lbrace m|\sqrt{\rho\epsilon_{m}}<\dfrac{\Delta_{b}}{2}\rbrace$. Hence, the maximal regret after eliminating the arm ${*}$ is upper bounded by, 
\begin{align*}
&\sum_{m_{*}=0}^{max_{j\in A^{'}}m_{j}}\sum_{i\in A^{''}:m_{i}>m_{*}}\bigg(\dfrac{1}{(  T\epsilon_{m_{*}})^{\psi_{m_i}}} \bigg).T\max_{j\in A^{''}:m_{j}\geq m_{*}}{\Delta}_{j}\\
&\leq\sum_{m_{*}=0}^{max_{j\in A^{'}}m_{j}}\sum_{i\in A^{''}:m_{i}>m_{*}}\bigg(\dfrac{1}{(  T\epsilon_{m_{*}})^{\psi_{m_i}}} \bigg).T.2\sqrt{\rho\epsilon_{m_{*}}}\\
&\leq\sum_{m_{*}=0}^{max_{j\in A^{'}}m_{j}}\sum_{i\in A^{''}:m_{i}>m_{*}}\bigg(\dfrac{4T^{1-\psi_{m_i}}}{\epsilon_{m_{*}}^{\psi_{m_i}-\frac{1}{2}}} \bigg)\\
&\leq\sum_{i\in A^{''}:m_{i}>m_{*}}\sum_{m_{*}=0}^{\min{\lbrace m_{i},m_{b}\rbrace}}\bigg(\dfrac{4T^{1-\psi_{m_i}}}{2^{-(\psi_{m_i}-\frac{1}{2})m_{*}}} \bigg)\\
&\leq\sum_{i\in A^{'}}\bigg(\dfrac{4T^{1-\psi_{m_i}}}{2^{-(\psi_{m_i}-\frac{1}{2})m_{*}}} \bigg)+\sum_{i\in A^{''}\setminus A^{'}}\bigg(\dfrac{4T^{1-\psi_{m_i}}}{2^{-(\psi_{m_i}-\frac{1}{2})m_{b}}} \bigg)\\
&\leq\sum_{i\in A^{'}}\bigg(\dfrac{4T^{1-\psi_{m_i}}*2^{\psi_{m_i}-\frac{1}{2}}}{\Delta_{i}^{\psi_{m_i}-\frac{1}{2}}} \bigg)+\sum_{i\in A^{''}\setminus A^{'}}\bigg(\dfrac{4T^{1-\psi_{m_i}}*2^{\psi_{m_i}-\frac{1}{2}}}{b^{\psi_{m_i}-\frac{1}{2}}} \bigg)\\
&\leq\sum_{i\in A^{'}}\bigg(\dfrac{4T^{1-\psi_{m_i}}2^{\psi_{m_i}-\frac{1}{2}}}{\Delta_{i}^{\psi_{m_i}-\frac{1}{2}}} \bigg)+\sum_{i\in A^{''}\setminus A^{'}}\bigg(\dfrac{4T^{1-\psi_{m_i}}2^{\psi_{m_i}-\frac{1}{2}}}{b^{\psi_{m_i}-\frac{1}{2}}} \bigg)\\
%& = \sum_{i\in A^{'}}\bigg(\dfrac{ C_{2}(\rho_{a}) T^{1-\rho_{a}}}{\Delta_{i}^{4\rho_{a}-1}} \bigg)+\sum_{i\in A^{''}\setminus A^{'}}\bigg(\dfrac{C_{2(\rho_{a})}T^{1-\rho_{a}}}{b^{4\rho_{a}-1}} \bigg) \text{, where } C_2(x) = \frac{2^{2x+\frac{3}{2}}x^{2x}}{\psi^{x}}
\end{align*}

%\text{, since } \sqrt{\rho_{a}\epsilon_{m}}<\dfrac{\Delta_{i}}{2}

 
Summing up \textbf{Case a} and \textbf{Case b}, the total expected regret is given by,
\begin{align*}
 &\sum\limits_{i\in A:\Delta_{i} > b}\bigg\lbrace \dfrac{T\Delta_{i}}{( \frac{3}{2} T\Delta_i^{2})^{\psi_{m_i}}}
  + \left( \Delta_i +\dfrac{22\Delta_i\psi_{m_i}\log( \frac{3}{2} T\Delta_i^2)}{ \Delta_i}\right)\\
  & \bigg(\dfrac{4T^{1-\psi_{m_i}}2^{\psi_{m_i}-\frac{1}{2}}}{\Delta_{i}^{\psi_{m_i}-\frac{1}{2}}} \bigg)\bigg \rbrace +\sum_{i\in A: 0 < \Delta_{i} \leq b}\bigg(\dfrac{4T^{1-\psi_{m_i}}2^{\psi_{m_i}-\frac{1}{2}}}{b^{\psi_{m_i}-\frac{1}{2}}} \bigg)
\end{align*}

% R_{T} \leq &\sum\limits_{i\in A:\Delta_{i}\geq b}\bigg\lbrace\bigg(\dfrac{2^{1+4\rho_{a}}\rho_{a}^{2\rho_{a}}T^{1-\rho_{a}}}{\psi^{\rho_{a}}\Delta_{i}^{4\rho_{a}-1}}\bigg) + \bigg(\Delta_{i}+\dfrac{32\rho_{a}\log{(\psi  T\dfrac{\Delta_{i}^{4}}{16\rho_{a}^{2}})}}{\Delta_{i}}\bigg)\\
%&  +  \bigg(\dfrac{T^{1-\rho_{a}}\rho_{a}^{2\rho_{a}}2^{2\rho_{a}+\frac{3}{2}}}{\psi^{\rho_{a}}\Delta_{i}^{4\rho_{a} -1}} \bigg) \bigg \rbrace+\sum\limits_{i\in A:0\leq\Delta_{i}\leq b}\bigg(\dfrac{T^{1-\rho_{a}}\rho_{a}^{2\rho_{a}}2^{2\rho_{a}+\frac{3}{2}}}{\psi^{\rho_{a}}b^{4\rho_{a} -1}} \bigg) + max_{i:\Delta_{i}\leq b}\Delta_{i}T

\end{proof}

\end{document}

