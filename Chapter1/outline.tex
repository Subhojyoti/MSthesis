In this chapter, we gave an overview of the various types of bandits available in the literature and also discussed about the main objectives of the thesis and our contributions. In this section, we give a general outline of the thesis that is to follow. In chapter \ref{chap:SMAB} we give a detailed overview of the stochastic multi-armed bandit model and the latest available algorithms in this setting. In the next chapter \ref{chap:EUCBV} we introduce our algorithm Efficient UCB Variance (EUCBV) for the stochastic multi-armed bandit model. We give theoretical guarantees on the performance of EUCBV and also show in numerical simulations that it indeed performs very well as compared to the state-of-the-art algorithms. In the subsequent chapter \ref{chap:tbandit1} we introduce a new variant of pure exploration multi-armed stochastic bandit called the thresholding bandit problem. We analyze the connections between thresholding bandit problem and pure exploration problem and also discuss several existing algorithms in both the settings that are relevant to carefully analyze the thresholding bandit problem. Then in chapter \ref{chap:tbandit2} we introduce our solution for the thresholding bandit problem, called the Augmented UCB (AugUCB) algorithm. We analyze our algorithm AugUCB and derive theoretical guarantees for it as well as show in numerical experiments that it indeed outperforms several state-of-the-art algorithms in the thresholding bandit setting. Finally, in chapter \ref{ThesisConc} we conclude by briefly summarizing all the problems covered in the thesis and discussing some future directions in which the stated problems can be further extended.


%Finally, in chapter \ref{chap:psbandit} we introduce the piecewise-stochastic bandit model which is a new variant that strides between the stochastic and adversarial setting. We discuss extensively on this setting and also provide our solution to this setting and show in numerical simulations that our solution is very close to the optimal solution. 