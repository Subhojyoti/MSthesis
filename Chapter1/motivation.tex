The MAB model fits very well in various real-world scenarios that can be modeled as decision-making under uncertainty  problems. Some of which are as follows:-
\begin{enumerate}
\item \emph{Online Shop Domain:} In the online shop domain \citep{ghavamzadeh2015bayesian}, a retailer aims to maximize profit by sequentially suggesting products to online shopping customers. In this scenario, at every timestep, the retailer displays an item to a customer from a pool of items which has the highest probability of being selected by the customer. The one-step interaction ends when the customer selects or does not select a product (which will be considered as a loss to the retailer). This feedback is incorporated by the learner as a feedback from the environment and it modifies its policy for the next suggestion. This process is repeated till a pre-specified number of times with the retailer gathering valuable information regarding the customer from this behaviour and modifying its policy to display other items to different customers. In its simplest form,  this can be modeled as a stochastic MAB problem which is studied in the first part of the thesis.
\item \emph{Medical Treatment Design:} Another interesting domain that MAB model was first studied was for the medical treatment design \citep{thompson1933likelihood},\citep{thompson1935theory}. Here at every timestep, the learner chooses to administer one out of several treatments sequentially on a stream of patients who are suffering from the same ailment (say). Let's also assume that there is a single treatment which will be able to alleviate the patients from their disease. Here, the one-step interaction ends when the patient responds well or does not respond well to the treatment whereby the learner modifies its policy for the suggestion to the next patient. The goal of the learner is to quickly converge on the best treatment so that whenever a new patient comes with the same ailment, the learner can suggest the best treatment which can relieve the patient of its ailment with a high probability. This problem can also be modeled as a stochastic MAB problem.
\item \emph{Financial Portfolio Management:} In financial portfolio management MAB models can also be used. Here, the learner is faced with the choice of selecting the most profitable stock option out of several stock options. The simplest strategy where we can employ a bandit model is this; at the start of every trading session, the learner suggests a stock to purchase worth Re $1$, while at the closing of the trading session it sells off the stock to witness its value after a day's trading. The profit recorded is treated as the reward revealed by the environment and the learner modifies its policy for the next day. Let's assume that no new stock options are being introduced over the considered time horizon and there is a single best stock option which if selected in perpetuity will always give the best returns. Then, the goal of the learner is reduced to identifying the best stock option as quickly as possible. This is another interesting variation which can be modeled as a stochastic MAB problem.
\item \emph{Product Selection:} A company wants to introduce a new product in the market and there is a clear separation of the test phase from the commercialization phase. In this case, the company tries to minimize the loss it might incur in the commercialization phase by testing as much as possible in the test phase. So from the several variants of the product that is in the test phase, the learning learner must suggest the product variant(s) whose qualities are above a particular threshold $\tau$ at the end of the test phase that has the highest probability of minimizing the loss in the commercialization phase. A similar problem has been discussed for single best product variant identification without threshold in \citet{bubeck2011pure}. This problem can be modeled as a stochastic thresholding MAB problem which is studied in the second part of the thesis.
\item \emph{Mobile Phone Channel Allocation:} Another similar problem as above concerns channel allocation for mobile phone communications \citep{audibert2009exploration}. Here there is a clear separation between the allocation phase and communication phase whereby in the allocation phase a learner has to explore as many channels as possible to suggest the best possible set of channel(s) whose qualities are above a particular threshold $\tau$. The threshold may depend on the subscription level of the customer such that with the higher subscription the customer is allowed better channel(s) with the $\tau$ set high. Each evaluation of a channel is noisy and the learning algorithm must come up with the best possible set of suggestions within a very small number of attempts. This setting can also be modeled as a stochastic thresholding MAB problem.
\item \emph{Anomaly Detection and Classification:} MABs can also be used for anomaly detection where the goal is to seek out extreme values in the data. Anomalies may not always be naturally concentrated which was shown in  \citet{steinwart2005classification}. To implement a MAB model the best possible way is to define a cut-off level $\tau$ and classify the samples above this level $\tau$ as anomalous along with a tolerance factor which gives it a degree of flexibility. Such an approach has already been mentioned in \citet{streeter2006selecting} and further studied in \citet{locatelli2016optimal}. Finally, this is also an interesting variation which can be modeled as a stochastic thresholding MAB problem.
\end{enumerate}


%    In all the above examples the MAB model performs well mainly because all of them suffer from \textit{exploration-exploitation dilemma}. This is characterized by action-selection choice faced by the learner where it must decide whether to stay with the action yielding highest reward till now or to explore newer actions which might be more profitable in the long run. MAB's are suited for such scenarios because 
%\begin{enumerate}
%\item They are easy to implement.
%\item The switch between exploration and exploitation is more well defined theoretically.
%\item They perform well empirically.
%\end{enumerate}
