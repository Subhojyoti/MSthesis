In this chapter, we deal with the stochastic multi-armed bandit (SMAB) setting. In its classical form, stochastic MABs represent a sequential learning problem where a learner is exposed to a finite set of actions (or arms) and needs to choose one of the actions at each timestep. After choosing (or pulling) an arm the learner receives a reward, which is conceptualized as an independent random draw from stationary distribution associated with the selected arm. Also, note that in SMAB, the distribution associated with each arm is fixed throughout the entire duration of the horizon denoted by $T$.

\begin{algorithm}[!th]
\caption{SMAB formulation}
\label{alg:SMAB}
\begin{algorithmic}
\State {\bf Input:} Time horizon $T$, $K$ number of arms with unknown parameters of reward distribution
\State \For{ each timestep $t=1,2,\ldots, T$}
\State The learner chooses an arm $i\in\A$, where $\A$ is the set of arms and $|\A|=K$.
\State The learner observes the reward $X_{i,t}\sim^{i.i.d} D_{i}$ where, $D_{i}$ is the distribution associated with the arm $i$. 
\State \EndFor
\end{algorithmic}
\end{algorithm}
	 
The rest of the chapter is organized as follows. We specify all the notations and assumptions in section~\ref{sec:notations}. Then we define the problem statement for the SMAB setting in section~\ref{sec:probDef}. In the next section~\ref{sec:motivation} we discuss the motivations behind the SMAB setting. In section~\ref{sec:related} we discuss extensively on the various state-of-the-art algorithms available for the SMAB setting. Finally, we draw our conclusions in section \ref{chap2:conc}.